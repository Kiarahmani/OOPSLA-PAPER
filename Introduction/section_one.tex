\section{Introduction}

Most of the modern web-based applications are implemented as multiple
agents simultaneously serving users and working on replicated data
objects across geographically distributed machines. Implementation and
reasoning techniques for strongly consistent replicated data stores
(e.g. stores providing linearizability and serializability) have been
studied for decades.  However, these strong notions of consistency
come with the price of availability and low response- time, which are
intrinsically crucial for distributed large-scale applications. Weak
consistency models such as \emph{monotonic reads} and \emph{monotonic
writes} however, liberate the application instances from extensive
synchronization and allows them to be "always-on" despite of possible
network partitionings.  Unfortunately, these weak guarantees suffer
from the lack of standardized definitions and enforcement
methodologies, which forces developers to modify their applications
with ad-hoc fixes in order to enforce their desired levels of
consistency. The ad-hoc implementation approaches and the lack of a
general reasoning framework for weak consistency, has resulted in the
difficulty of proving correctness and optimality properties of such
highly error-prone implementations. To make the matter worse, many of
distributed applications simultaneously require different levels of
consistency for different tasks and even face new consistency
requirements after the development phase is over. 

Let's consider a distributed bulletin board application where users
send requests to servers to post a message or read the current
messages on the board. Since the underlying data store provides
eventual consistency guarantee, we can be sure that every message
submitted to any server, will \emph{eventually} become visible at all
servers.  However, we might also be interested in ensuring users that
if they see a message as a result of a read request, they will also
see that message in their subsequent reads. This requirement which is
known as "monotonic reads", is not guaranteed under eventual
consistency, since users might send read request to different servers
with different sets of available messages.  Version vectors and
similar ad-hoc approaches are usually used to add guards to the code
in order to enforce the required consistency guarantees. This
error-prone task requires developers to also come up their own proofs
of correctness which to make the matter worse, has to be renewed each
time the application is updated or the consistency requirements are
changed (e.g. write operations are also realized to require guards
before executions). 

%The Figure Containing The Bulletin Board and its Modified Version
\begin{figure}[h]
    \centering
    \begin{subfigure}[b]{0.48\textwidth} 
       \lstinputemph[language=Java]{0}{../Introduction/app1.java}
       \label{fig:app1}
    \end{subfigure} ~
       \begin{subfigure}[b]{0.52\textwidth} 
        \lstinputemph[language=Java]{3}{../Introduction/app2.java}
        \label{fig:app2}
    \end{subfigure}\\
    \hrulefill
    \caption{A Bulletin Board Application (left) and the Modified Version Including Guards (right) }
\end{figure}


In this paper, we address these issues by introducing a principled approach to 
derive enforcement mechanisms for various forms of weak consistency 
guarantees. We offer developers with a language to specify their application-level consistency 
requirements as logical formulae expressing allowed relationships between update effects. 
We also introduce a new tool that generates a shim layer on top of the eventually 
consistent data store, and extends it to a key-value store with multiple environments, each of which 
shows the desired behavior specified by the given specifications. Our consistency enforcement 
methodology is independent of the underlying store, and only assumes the well established 
guarantee of "eventual delivery". 
Moreover, our specification language is also powerful enough to express all the known consistency 
guarantees in the context. 

We argue that all the consistency guarantees basically specify, \emph{when} an 
application instance should block a user request and for arrival of 
\emph{what} remote updates it should wait. For obvious reasons, any implementation of these 
consistency guarantees must be \emph{correct} and \emph{optimal}. The former property states that 
all possible behaviors of the implementation are allowed by the given specification and the later 
ensures that application instances do not engage in 
any unnecessary synchronization before responding to a user request (i.e. users are only blocked if 
it is absolutely necessary).
\\ Our technique is based on tracking relationships between update effects, and maintaining multiple 
(logical) caches at the shim layer, each of which is enforced to satisfy a specific level of consistency, 
derived from the given specifications. We believe that our approach is the first ever principled 
reasoning and implementation framework for weak consistency enforcement techniques which 
is also proven to be \emph{correct} and \emph{optimal}. 
By separating all the consistency management procedures from the application level, we are taking 
the non-trivial task of proving soundness criteria for ad-hoc consistency enforcement techniques 
off of the developer's shoulders. 
% Contributions of the paper
\\The paper makes the following contributions:
\begin{itemize}
\item Introducing a new specification language that covers all the known consistency guarantees in 
the context. Our language is arguably simpler than the previous works, and is based on observing 
similarities of different consistency requirements.

\item Proposing a principled consistency enforcement methodology that is 
complete for our specification language and allows deriving multi-purpose consistency enforcement 
tools.

\item Presenting proofs of correctness and optimality of our consistency enforcement methodology 
and introducing the first ever general reasoning framework for weak consistency guarantees. 

\item Introducing our tool called  \tool that realizes our consistency enforcement methods and 
extends an off-the-shelf eventually consistent data store (Cassandra) to a multi purpose key-value 
data store with the ability to guarantee any requested consistency level defined by the 
developers. 

\item Presenting the evaluation results showing the utility of \tool compared to a hard-coded
implementation of a weak consistency guarantee. 
\end{itemize}

% Structure of the paper
The rest of the paper is structured as follows. 
In section 2 we introduce the running example and explain problems associated with it. 
Section 3 introduces our system model, including the distributed underlying database model, our 
multi-purpose shim layer, and the formal definition of our specification language.
In section 4, we introduce our consistency enforcement strategy in detail and introduce the 
formal semantics of the shim layer. In section 5, we present the formal definitions 
and proofs of correctness of optimality properties of our shim. Sections 6 introduces our tool 
including 
the implementation details,  
which is followed by section 7, where the test results from running \tool on 
distributed real world machines are presented. 
We discuss the related works and future directions in the section 8. 
