Here we explain the the core operational semantics of our algorithm, which for simplicity reasons, 
is parametrized over a single contract $\psi$:
\begin{smathpar}
\begin{array}{lcc}
\psi = \forall (a,b). a \xrightarrow{R} b  \Rightarrow a
\xrightarrow{vis} b, & \spc & R=r_1;r_2;...;r_k \\
\end{array}
\end{smathpar}
Before intorducing the semantics, let us define the inverse of a
given relation $R=r_1;r_2;...;r_k$.
The following definitions are parametrized over an execution $E$ and a set of \emph{available} effects V.
The idea is to define the inverse only if the \emph{necessary
information} about the sequence of relations, is available in V; this
way we can capture the reality of the distributed stores, where some
effects, carrying the required info about the relation being computed
might not be present at the moment.
\begin{smathpar}
\begin{array}{ccc}
   b \in (R;r_k)^{-1}_V (a) & \iff & \exists c.(c\in r_k^{-1} (a)) \wedge (b \in R^{-1}
   (c))  \wedge  (r_k^{-1}(a) \subseteq V)
\end{array}
\end{smathpar}
The above definition is based on the following definition of
the inverse of a clause :
\begin{smathpar}
r^{-1}(S) = 
\begin{cases}
\begin{array}{lll}
\bigcup^{}_{e\in S}. \{\eta|(\eta,e) \in E.r \} & if &r\in\{so,vis\} \\ 
r_1^{-1}(S)\cup r_2^{-1}(S) & if & r=r_1\cup r_2\\
G_{r}(S,\emptyset) & if &  r=r^* 
\end{array}
\end{cases}
\end{smathpar}
Where we define the helping function $G$ as follows: 
\begin{smathpar}
G_r(S,R) =
\begin{cases}
\begin{array} {lll}
G_r(r^{-1}(S,R\cup r^{-1}(S))) &if& r^{-1}(S) \neq \emptyset  \\
R  & &  otherwise
\end{array}
\end{cases}
\end{smathpar}
Given a set of effects, we also define its maximal closed subset
under the relation R as follows: 
\begin{smathpar}
\left \lfloor S \right \rfloor_V = S' \spc \mathtt{iff} \spc S'
\subseteq S \; \wedge \;
R_V^{-1}(S') \subseteq S' \; \wedge \; 
\not\exists
S''.((R_V^{-1}(S''))\subseteq S''\wedge |S''|>|S'|)
\end{smathpar}
\begin{figure}[t]
\raggedright
\textbf{Auxiliary Definitions}\\ \vspace{-2mm}
%
\begin{minipage}{0.5\textwidth}
\begin{fmathpar}
\begin{array}{lclcl}
  \multicolumn{5}{c}{
    {op} \in \mathtt{Operation\; Name} \spc \spc
    {v} \in \mathtt{Return\; Value} \spc \spc
    {s} \in \mathtt{Session\; Id} \spc\spc
  } 
  \\ 
  \eff & \in & \mathtt{Effect} & \coloneqq &  (s,op,v)\\
  F_{op} & \in & \mathtt{Op.\,Def.} & \coloneqq & \set{\eff} \mapsto v\\
  \EffSoup & \in & \mathtt{Eff\,Soup}	  & \coloneqq & \set{\eff} \\
  \visZ,\soZ &	\in & \mathtt{Relations} & \coloneqq & \set{(\eff,\eff)} \\
  {\E} 	& \in & \mathtt{Exec\;State}  & \coloneqq & \Exec \\
\end{array}
\end{fmathpar}
\end{minipage}
%

\vspace {3mm}

\textbf{Auxiliary Reduction} \; \\
\fcolorbox{black}{pgrey}{\scriptsize \(\auxred{S} {(\E,op_{<s,i>})} {} {(\E',\eff)}\)}\\
\begin{minipage}{0.9\textwidth}
\vspace{2mm}
\rulelabel{Oper}
\vspace{-2mm}
\begin{fmathpar}
\stretcharraybig
\begin{array}{l}
\RuleTwo
{
%\Theta(\rho \mapsto (v,cache)) \qquad
S \subseteq \EffSoup \qquad F_{op}(S) = v \qquad
\eta \not\in S \qquad
\eff = (s,op,v) \qquad  \\
%\id(\eta) = i \qquad
%\{\eff'\} = \EffSoup_{({\sf SessID}=s,\,{\sf SeqNo}=i-1)}\\
\EffSoup' = \EffSoup \cup \{\eff\}  \qquad
\visZ' = \visZ \cup S \times\{\eff\}\qquad
\soZ' = \soZ \cup \{(\eta',\eta) \,|\, \eta'\in \EffSoup_{({\sf
SessID}=s)}      \}\qquad
%\soZ' = (\soZ^{-1}(\eff') \cup \eff') \times\{\eff\} \cup \soZ
}
{
  \auxred {S} {((\EffSoup,\visZ,\soZ), op_{<s,i>}))}
  {} {((\EffSoup',\visZ',\soZ'),\eta)}
}
\end{array}
\end{fmathpar}
\end{minipage}
\vspace{4mm}\\
\textbf{Operational Semantics} \; \\
  \fcolorbox{black}{pgrey}{\scriptsize \((\E,op_{<s,i>}) \;\xrightarrow{V}\; (\E',\eff)\)}\\
\vspace{2mm}
\begin{minipage}{0.45\textwidth}
\rulelabel{UB Exec}
\vspace{-2mm}
\begin{fmathpar}
\stretcharraybig
\begin{array}{l}
\RuleTwo
{
  \visZ \subseteq r_k \spc
  V \subseteq E.A \spc  
  V'= \left \lfloor V \right \rfloor_V \spc
  \\ %   R^{-1}_{V}(\eta) \subseteq V' \\
  \auxred {V'} {(E, op_{<s,i>}))}
    {} {(E',\eta)} 
}
{
  (\E,op_{<s,i>}) \;\xrightarrow{V}\; (\E', \eff)
}
\end{array}
\end{fmathpar}
\end{minipage}
\hfill
\begin{minipage}{0.45\textwidth}
\rulelabel{LB Exec}
\vspace{-2mm}
\begin{fmathpar}
\stretcharraybig
\begin{array}{l}
\RuleTwo
{
     \visZ \not\subseteq r_k \spc
     V \subseteq E.A \spc  
     R^{-1}_{V}(\eta) \subseteq V \\
  \auxred {V} {(E, op_{<s,i>}))}
    {} {(E',\eta)} 
}
{
  (\E,op_{<s,i>}) \;\xrightarrow{V}\; (\E', \eff)
}
\end{array}
\end{fmathpar}
\end{minipage}
\\
\vspace{5mm}
\hrulefill\\
\caption{Core Operational semantics of a replicated data store.}
\label{fig:semantics}
\end{figure}


\newpage

