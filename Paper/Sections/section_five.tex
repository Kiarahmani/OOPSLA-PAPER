%========================================================================================================================
%========================================================================================================================
%========================================================================================================================
\section{Semantics}
\label{sec:semantics}
In this section we formalize our consistency enforcement algorithm with an
operational semantics, which is also a high-level abstraction of our
tool \tool.
The proposed approach is complete for the specification language defined
in \ref{sec:ctrt_language}
and extends the underlying key-value store to a multi-consistent shim
layer. Here for simplicity reasons we present an operational semantics 
parametrized over a contract with a single prop. However, as we explain
in section \ref{subsec:generalization}, the rules can be easily
generalized to cover multiple consistency levels, each specified by an
arbitrary contract. Therefore, in the rest of this section we will assume a given contract $\psi$ of the
following form:
	\begin{smathpar}
	\begin{array}{lcc}
		\psi = \forall (a,b). a \xrightarrow{R} b  \Rightarrow a
		\xrightarrow{\visZ} b, & \spc & R=r_1;r_2;...;r_k \\
	\end{array}
	\end{smathpar}

The operational semantics is defined via a small-step relation over \emph{execution
states}, which are tuples of the form $\E=(\EffSoup,\visZ ,\soZ)$.
The \emph{effect soup} $\E.\EffSoup$, represents the set of all
effects produced in the system, and  $\E.\visZ$,
$\E.\soZ$ $\subseteq \EffSoup \times \EffSoup$, stand for, respectively the
visibility and session order relations
between such effects. 
We also use the notation $\EffSoup_{(condition)}$
to represent the projection of $\EffSoup$ to a subset
of effects  in $\EffSoup$, that satisfy the specified condition.
The reduction step that will be defined, is of the form
\begin{smathpar}
(\E,\op_{<s,i>}) \;\xrightarrow{\V}\; (\E', \eff),
\end{smathpar}
which can be interpreted as a reduction caused by a replica with a local 
set of effects $\V$, after it executes an operation
$\op$, which is the $i$'th request from the session $s$. 
During this reduction step a new effect $\eff$ is produced and added to
the effect soup, with new relations $\visZ$ and $\soZ$ that are updated
accordingly.


%=============================================================================================================
%--------- Definitions to be used in the semanrics
%=============================================================================================================
\subsection{Preliminaries}
\label{subsec:prelim}
Before intorducing the operational semantics, 
we need to formally define the inverse of a relation, \emph{given a set
of local effects $\V$}.
In correspondence with real
distributed replicas, where at any moment there is only a \emph{subset}
of all produced effects present at a replica, we want to define
the inverse of a relation,
only if the necessary information is present at the replica. For example, assume a
replica wants to compute the set $D$ of all effects satisfying
$\xrightarrow{r_1;r_2}$
relation to an effect $\eff$. This is done in two steps, by first finding
the set $D'$ of all effects satisfying relation $\xrightarrow{r_2}$ to
$\eff$, and only then finding effects satisfying $\xrightarrow{r_1}$
relation to any member of set $D'$. We should formalize our definition of the
inverse relation, so that it captures the fact that the replica cannot
compute $D$, if some mid-level effects in $D'$ are not locally available yet.
In order to do so, we first formally define the inverse of a seed relation
$r$ for a given set of effects $S$ as:
\begin{smathpar}
\rel^{-1}(\Set) = 
\begin{cases}
\begin{array}{lll}
\bigcup^{}_{\eta\in \Set}. \{\eta'|(\eta',\eta) \in \E.\rel \} & \myif
&\rel\in\{\soZ,\visZ\} \\ 
\rel_1^{-1}(\Set)\cup \rel_2^{-1}(\Set) & \myif & \rel=\rel_1\cup \rel_2
\end{array}
\end{cases}
\end{smathpar}
Now, using a helping recursive function $G_r$, we extend the above
definition to also cover the closure of a seed relation:
\begin{smathpar}
(r^{*})^{-1}(S) = G_{r}(S,\emptyset) 
\end{smathpar}
where,
\begin{smathpar}
G_r(S,P) =
\begin{cases}
\begin{array} {lll}
G_r(r^{-1}(S) , P\cup r^{-1}(S)) &if& r^{-1}(S) \neq \emptyset  \\
P  & &  otherwise
\end{array}
\end{cases}
\end{smathpar}
Now we are ready to formally define the inverse of the given relation
$R=r_1;r_2;...;r_k$, according to a set $\V$ of local effects. This is
done by logically specifying the set of effects satisfying the inverse
relation with a given effect $\eff$:
\begin{smathpar}
\begin{array}{ccc}
   \eta' \in (r_1;r_2;...;r_k)^{-1}_V (\eta) & \iff & \exists
   \eta''.(\eta''\in r_k^{-1}
   (\eta))\; \wedge \;(\eta' \in (r_1;r_2;...;r_{k-1})^{-1}
   (\eta''))\;  \wedge \; (r_k^{-1}(\eta) \subseteq V)
\end{array}
\end{smathpar}
Note that the first two conjuncts are the straightforward stepwise definition of
the closure of relations. However, the last conjunct is also added, to
make sure that before going any further in recursive computation, $V$
includes all
effects of the very next step. This is how we capture the
necessaity of the presence of mid-level effects in real world systems for
computing the final set of dependencies. We can now capture the blocking
process in the semantics, by requiring the presence of $R_V^{-1}$ of
operations under LB contracts at the
replicas that they are being executed at.

Lastly, we finish this part, by formally defining the \emph{maximally
closed subset} of a set of effects $S$, given a relation $R$ and a set
of local effects $V$
as follows:
\begin{smathpar}
\left \lfloor S \right \rfloor_V = S' \spc \iff \spc S'
\subseteq S \; \wedge \;
R_V^{-1}(S') \subseteq S' \; \wedge \; 
\not\exists
S''.((R_V^{-1}(S''))\subseteq S''\wedge |S''|>|S'|)
\end{smathpar}
The above statement, simply requires $\left \lfloor S \right \rfloor_V $
to be closed under $R_V^{-1}$ and to be the largest among such closed
subsets. This allows us to introduce filteration process in the
semantics, which only allows the maximally closed subset of available
effects to be witnessed by operations with UB contracts.

%=============================================================================================================
%--------- The operational semantics
%=============================================================================================================
\subsection{Core Operational Semantics}
\begin{figure}[t]
\raggedright
\textbf{Auxiliary Definitions}\\ \vspace{-2mm}
%
\begin{minipage}{0.5\textwidth}
\begin{fmathpar}
\begin{array}{lclcl}
  \multicolumn{5}{c}{
    {op} \in \mathtt{Operation\; Name} \spc \spc
    {v} \in \mathtt{Return\; Value} \spc \spc
    {s} \in \mathtt{Session\; Id} \spc\spc
  } 
  \\ 
  \eff & \in & \mathtt{Effect} & \coloneqq &  (s,op,v)\\
  F_{op} & \in & \mathtt{Op.\,Def.} & \coloneqq & \set{\eff} \mapsto v\\
  \EffSoup & \in & \mathtt{Eff\,Soup}	  & \coloneqq & \set{\eff} \\
  \visZ,\soZ &	\in & \mathtt{Relations} & \coloneqq & \set{(\eff,\eff)} \\
  {\E} 	& \in & \mathtt{Exec\;State}  & \coloneqq & \Exec \\
\end{array}
\end{fmathpar}
\end{minipage}
%

\vspace {3mm}

\textbf{Auxiliary Reduction} \; \\
\fcolorbox{black}{pgrey}{\scriptsize \(\auxred{S} {(\E,op_{<s,i>})} {} {(\E',\eff)}\)}\\
\begin{minipage}{0.9\textwidth}
\vspace{2mm}
\rulelabel{Oper}
\vspace{-2mm}
\begin{fmathpar}
\stretcharraybig
\begin{array}{l}
\RuleTwo
{
%\Theta(\rho \mapsto (v,cache)) \qquad
S \subseteq \EffSoup \qquad F_{op}(S) = v \qquad
\eta \not\in S \qquad
\eff = (s,op,v) \qquad  \\
%\id(\eta) = i \qquad
%\{\eff'\} = \EffSoup_{({\sf SessID}=s,\,{\sf SeqNo}=i-1)}\\
\EffSoup' = \EffSoup \cup \{\eff\}  \qquad
\visZ' = \visZ \cup S \times\{\eff\}\qquad
\soZ' = \soZ \cup \{(\eta',\eta) \,|\, \eta'\in \EffSoup_{({\sf
SessID}=s)}      \}\qquad
%\soZ' = (\soZ^{-1}(\eff') \cup \eff') \times\{\eff\} \cup \soZ
}
{
  \auxred {S} {((\EffSoup,\visZ,\soZ), op_{<s,i>}))}
  {} {((\EffSoup',\visZ',\soZ'),\eta)}
}
\end{array}
\end{fmathpar}
\end{minipage}
\vspace{4mm}\\
\textbf{Operational Semantics} \; \\
  \fcolorbox{black}{pgrey}{\scriptsize \((\E,op_{<s,i>}) \;\xrightarrow{V}\; (\E',\eff)\)}\\
\vspace{2mm}
\begin{minipage}{0.45\textwidth}
\rulelabel{UB Exec}
\vspace{-2mm}
\begin{fmathpar}
\stretcharraybig
\begin{array}{l}
\RuleTwo
{
  \visZ \subseteq r_k \spc
  V \subseteq E.A \spc  
  V'= \left \lfloor V \right \rfloor_V \spc
  \\ %   R^{-1}_{V}(\eta) \subseteq V' \\
  \auxred {V'} {(E, op_{<s,i>}))}
    {} {(E',\eta)} 
}
{
  (\E,op_{<s,i>}) \;\xrightarrow{V}\; (\E', \eff)
}
\end{array}
\end{fmathpar}
\end{minipage}
\hfill
\begin{minipage}{0.45\textwidth}
\rulelabel{LB Exec}
\vspace{-2mm}
\begin{fmathpar}
\stretcharraybig
\begin{array}{l}
\RuleTwo
{
     \visZ \not\subseteq r_k \spc
     V \subseteq E.A \spc  
     R^{-1}_{V}(\eta) \subseteq V \\
  \auxred {V} {(E, op_{<s,i>}))}
    {} {(E',\eta)} 
}
{
  (\E,op_{<s,i>}) \;\xrightarrow{V}\; (\E', \eff)
}
\end{array}
\end{fmathpar}
\end{minipage}
\\
\vspace{5mm}
\hrulefill\\
\caption{Core Operational semantics of a replicated data store.}
\label{fig:semantics}
\end{figure}

 %--- The Figure Containing the Rules

%--- Section intro
In this part we present the reduction rules, representing our
consistency preservation approach.
Figure \ref{fig:semantics} presents the set of rules defining the
auxiliary relation ($\hookrightarrow$) and small-step reduction relation 
($\rightarrow$) over executions. The latter relation is parametrized
over a set $V$,
that represents the set of effects that are available at the replica
that is 
taking the step. Obviously $V$ must be a subset of the effect soup 
of the initial execution, however, there is no other restrictions on $V$,
since we only assume eventual consistency at the underlying store.

%--- The [OPER] rule
Let's now go over the auxiliary reduction rule
\rulelabel{oper},
that represtns the procedure of producing a new effect $\eff$, by witnessing a set
of effects $S$. 
An effect is formally defined as a tuple $\eff=(s,op,v)$, representing the
session and the operation name 
whose execution creates $\eff$, and the state value
that the replica returns as the response to that operation.
The rule explains how the execution state changes after producing an
effect at a replica. Specifially, in the new state, the effect soup
$\EffSoup'$ must
now contain the newly created effect $\eff$, and the relations $\visZ'$
and $\soZ'$ must capture the fact that the set $S$ was made
visible to $\eta$, and all effects from the same session that were
already presenet in the intial execution state, should be in session
order with $\eff$ in the final execution state.


%--- rule for (->) relation
Now we explain the rule for reduction relation $(\xrightarrow{V})$,
namely \rulelabel{exec}, which represents the execution of operations
in a replica that updates the global state and produces a new
effect. The rule is crafted so it would only take place, if the
opertaing replica, contains the necessary effects specified by LB
contracts, and only allows the operations to witness effects that would
not violate the UB contracs.



%--- Explaining the operational rules
The preconditions of the reduction rules, include an auxiliary
reduction, explaining the changes that occur in the execution state $\E$
when an operation is performed. 
Moreover, the rule contains
$R_V^{-1}(\eff)\subseteq V'$ precondition, which basically guarantees
that the operation will witness the set of necessary effects, specified
by an LB contract. Note that this precondition trivially holds for UB
contracts, which only put constraints on the effects that are
\emph{about to} be made
visible to $\eff$, unlike LB contracts that end with an $\soZ$ relation and
refer to an \emph{already defined} set of dependencies. 

Moreover, the we have the precondition $V' = \left \lfloor V \right
\rfloor_V$, which basically filters out the effects from V, that would
violate a UB contract. Similarly this condition is trivially held for LB
contracts, since $V'$ is
already consistent because of the above pre-condition. In this case,
since $V'$ has
to be maximal, it should be set equal to $V$. The \rulelabel{exec}
similarly handles hybrid contracts, by both blocking oprtations and
filtering updates.

%=============================================================================================================
%--------- Theorem on correctness of enforcement
%=============================================================================================================
\subsection{Soundness}
\label{subsec:sound}
Here, we present a meta-theoretic property showing that the given contract would never be
violated, if the execution is in a consistent state and takes a reduction
step. We start by defining what we mean by a consistent state of the
execution: 
%
%Subsection on the correctness of contract 
%
In order to prove a meta-theoretic correctness property for our
semantics, we first define a $\psi$-consistent set of effects $S$ given
a execution state $\E$ as follows:

\begin{equation}
\scriptsize
S \mathtt{\;is\;} \psi \mathtt{-consistent} \iff \forall (\eff \in S).
\forall(a\in \E.\EffSoup). \trunc{R}(a,\eta)
\Rightarrow a \in S
\end{equation}
%Where the definition of $R$ is based on the definition of $R^{-1}$ from
%subsection 
%\ref{subsec:prelim}:
%\begin{equation}
%\scriptsize
%R(a,b) \iff a \in R_{E.A}^{-1}(b)  
%\end{equation}
\begin{theorem}
\label{theorem:one}
For all reduction steps 
$
\scriptsize
\; (\E,op_{<s,i>}) 
    \xrightarrow{V}
  (\E',\eff)  
$,
\begin{fmathpar}
\begin{array}{ll}
(i) & \mathtt{if\;} V \mathtt{\;is\;} \psi \mathtt{-consistent\;} \mathtt{under\;} \E,
\mathtt{\;then\;}  V \cup \{\eta\} \mathtt{\;is\;} \psi\mathtt{-consistent \; under\;} E'  \\
(ii) & E' \models \psi[\eta/\hat{\eta}]
\end{array}
\end{fmathpar}



\end{theorem}
\begin{proof}
Appendix \ref{app:proof1}
\end{proof}







%=============================================================================================================
%--------- Theorem on maxVis and minWait
%=============================================================================================================
\subsection{Optimality}
\label{subsec:opt}
Now we will present two theorems, regarding the optimality of our
approach. Firstly, we show that there
cannot be a larger subset of local effects in a replica to be made
visible to an operation, than what
our reduction rules specify. Next, we will prove the liveness property
of our semantics, which guarantees that all replicas will take a step,
given the proper set of local effects. Considering the eventual delivery of all
updates, which is guaranteed by the underlying store, this theorem
guarantees  that our system would never get permanently stuck. 
%
%Subsection of the maximality of the set made visible and liveness
%properties 
%
\begin{theorem}
\label{theorem:two}
For all reduction steps
{\footnotesize $
(\E,op_{<s,i>}) 
    \xrightarrow{V}
  (\E',\eff) 
$},
the set of effects made visible to $\eta$ is maximal. i.e. for all
 {\footnotesize $a \in V$}, if 
 {\footnotesize $ \SC {\trunc{R}} a V$}, then 
\begin{fmathpar}
(a,\eta) \not\in \E'.\visZ \Rightarrow 
(\E'.A,\E'.\visZ \cup \{a,\eta\}, \E'.so) \not\models \psi[\eta/\hat{\eta}]
\end{fmathpar}
\end{theorem}


%
% THE MINIIMAL WAIT THEOREM
%

\begin{theorem}
\label{theorem:three}
For all execution states $E$, if there
exists ($S\subseteq E.A$) such that: 
\begin{smathpar}
\auxred{S} {(\E,op_{<s,i>})} {} {(\E',\eff)} \spc \wedge \spc (\mathtt{S \cup \{\eta\} \; is 
\;} \psi \mathtt{-consistent \; under \;} E' )  
\end{smathpar}
then there exist  $E''$, $\eff'$ and $V\subseteq E.A$ such that:
$((\E,op_{<s,i>})\;\xrightarrow{V}\;(\E'',\eff'))$
\end{theorem}
\begin{proof}
The proof is given by choosing set $V$ to be equal to $S$, and then
considering two cases, where either $S$ or $\left \lfloor S \right
\rfloor_S$ are made visible to operations, if the contract is respectively
waiting and non-waiting. In both cases, all premises of taking an step
are satisfied.
A detailed proof can be found in appendix
\ref{app:proof3}.
\\
\end{proof}



%=============================================================================================================
%--------- How it can be generalized for all contracts
%=============================================================================================================
\subsection{Generalization}
\label{subsec:generalization}
We finish this section by extendeding our ideas in two dimentions. 
We will first explain how to handle an arbitrary
contract $\psi$ of the following form:  
\begin{smathpar}
\psi = \pi_1 \wedge \pi_2 \wedge ... \wedge \pi_m \qquad \qquad 
\pi_i = \forall (a,b). a \xrightarrow{R_i} b \Rightarrow a
\xrightarrow{\visZ} b
\end{smathpar}
Later, we will
explain how to maintain multiple levels of consistency simultaneously,
each of which is defined for a different operation name. We will assume an arbitrary contract
$\psi_{\op}$ for every user-defined operation $\op$, and explain how to
modify our system model to preserve them all.

To begin with, as we mentioned earlier, all propositions in our specification language,
either put a maximal or a minimal bound on the subset of local effects 
to be made visibe to each opreation. 
This simply means that when the system is given a conjunction
of propositions, it should define the such subsets in a way, so it would not violate
\emph{any} of them. 
Therefore, by a few modifications we can extend the system to support
all contracts. Firstly, the single premise $R_V^{-1}(\eta) \subseteq V$
in the reduction rule should be replaced with the following
conjunction: 
\begin{smathpar}
\bigwedge_{1 \leq i \leq m} (R_i)_V^{-1}\subseteq V
\end{smathpar}
Secondly, the definition of the maximal closed subset of local effects
must also be modified to a subset that is closed under \emph{all} given
relations:
%---------------------------------------------------------------------
% I am not sure if we should include the formal definition here. It is
% unnecessarily complex
\begin{smathpar}
\left \lfloor S \right \rfloor_V = S' \spc \iff \spc S'
\subseteq S \; \wedge \;
\bigwedge(R_i)_V^{-1}(S') \subseteq S' \; \wedge \; 
\not\exists
S''.(\bigwedge ((R_i)_V^{-1}(S''))\subseteq S''\wedge |S''|>|S'|)
\end{smathpar}
%---------------------------------------------------------------------

Moreover, for modifying the system to handle multiple contracts
simultaneously, we can
extend the local effect set $V$, to a sequence
of sets $V_{\op_i}$, each maintaning  the consistency level for an
operation type $\op_i$. Now we define the modified form of execution steps as
follow:
\begin{smathpar}
(\E,\op_{<s,i>}) 
    \;\xrightarrow{V_{\op}}\;
  (\E',\eff) 
\end{smathpar}
The local effect set $V$ must also be replaced with $V_{\op_i}$ in the 
premises of the reduction rules, so each operation of type $\op_i$ would
only witness the associated subset for its own consistency requirements.
This abstractly represents our implementation, in the sense that all operations
work only on a specific subset of available effects at any replica. The subset, is
maintained according to the contract assosiated with each operation, and
is guaranteed to preserve the consistency requirements following the
theorems of sections \ref{subsec:sound} and \ref{subsec:opt}. 

