%
%=========================================================================
%
\section{Semantics}
In this section we explain the core operational semantics of our
consistency enforcement algorithm. 
Our approach is complete for the specification language defined earlier,
and extends the key-value store to multiple different consistency
environments. Here for simplicity reasons, we present the semantics 
parametrized over a single contract,
however as we explain in section \ref{subsec:generalization}, it can be easily
generalized to cover all contracts and multiple consistency levels. In
the rest of this section we will assume a given contract $\psi$ of the
following form:
	\begin{smathpar}
	\begin{array}{lcc}
		\psi = \forall (a,b). a \xrightarrow{R} b  \Rightarrow a
		\xrightarrow{vis} b, & \spc & R=r_1;r_2;...;r_k \\
	\end{array}
	\end{smathpar}

We now define the execution state $\E$ 
as a tuple $(\EffSoup,\visZ ,\soZ)$. The \emph{Effect Soup} $\EffSoup$ represents the set of all
effects produced by all application instances, and  $\visZ$ and
$\soZ$ respectively stand for the visibility and session order relation
between them. The reduction that we will define is of the form
\begin{smathpar}
(\E,op_{<s,i>}) \;\xrightarrow{V}\; (\E', \eff)
\end{smathpar}
which can be interpreted as the step on the execution state $\E$, where a replica
that contains a local set of effects $V$, executes an operation $op$
which is the $i$'th request from the session $s$. During this reduction step 
by producing a new effect $\eff$, the replica has taken the execution to
a new state $\E'$.


%=================================================
%--------- Definitions to be used in the semanrics
%=================================================
\subsection{Preliminaries}
Before intorducing the semantics, given an execution state $E$ and a
set $V$ of \emph{available} effects, we need to define the inverse of a
relation. In correspondence with real
distributed replicas, where at any given moment only a subset of all
effects is present, we want to define the inverse of the relation,
\emph{only
if} the necessary information (which is carried by effects themselves) is present.
To do so, let's first formally define the inverse of a seed relation
$r$ for a given set of effects $S$:
\begin{smathpar}
r^{-1}(S) = 
\begin{cases}
\begin{array}{lll}
\bigcup^{}_{\eta\in S}. \{\eta'|(\eta',\eta) \in E.r \} & if &r\in\{so,vis\} \\ 
r_1^{-1}(S)\cup r_2^{-1}(S) & if & r=r_1\cup r_2
%\\G_{r}(S,\emptyset) & if &  r=r^* 
\end{array}
\end{cases}
\end{smathpar}
Now, using a helping fixpoint function $G_r$, we extend the above
definition to cover also the closure of a seed relation:
\begin{smathpar}
(r^{*})^{-1}(S) = G_{r}(S,\emptyset) 
\end{smathpar}
where,
\begin{smathpar}
G_r(S,P) =
\begin{cases}
\begin{array} {lll}
G_r(r^{-1}(S) , P\cup r^{-1}(S)) &if& r^{-1}(S) \neq \emptyset  \\
P  & &  otherwise
\end{array}
\end{cases}
\end{smathpar}
{\bf Definition:} For a given relation $R=r_1;r_2;...;r_k$, we
recursively define $R^{-1}$ as follows: 
\begin{smathpar}
\begin{array}{ccc}
   \eta' \in (r_1;r_2;...;r_k)^{-1}_V (\eta) & \iff & \exists
   \eta''.(\eta''\in r_k^{-1}
   (\eta))\; \wedge \;(\eta' \in (r_1;r_2;...;r_{k-1})^{-1}
   (\eta''))\;  \wedge \; (r_k^{-1}(\eta) \subseteq V)
\end{array}
\end{smathpar}
Note that the first two conjuncts are the straightforward stepwise definition of
the closure of relations. However, the last conjunct also require $V$ to include all
effects of the next step, before going further. This captures the
reality of the real systems where the dependencies cannot be computed
when the necessary information is not present at the replica. 

Now, we finish this part, by defining the maximally closed subset of a set
of effects $S$, given a relation $R$ and a set of available effects $V$
as follows:
\begin{smathpar}
\left \lfloor S \right \rfloor_V = S' \spc \iff \spc S'
\subseteq S \; \wedge \;
R_V^{-1}(S') \subseteq S' \; \wedge \; 
\not\exists
S''.((R_V^{-1}(S''))\subseteq S''\wedge |S''|>|S'|)
\end{smathpar}
The above statement, simply requires $\left \lfloor S \right \rfloor_V $
to be closed under $R_V^{-1}$ and to be the largest among such closed
subsets.



%=================================================
%--------- The operational semantics
%=================================================
\subsection{Core Operational Semantics}

%--- The Figure Containing the Rules
\begin{figure*}[h]
\raggedright
%
\textbf{Auxiliary Definitions}\\
%
\begin{minipage}{\columnwidth}
\begin{smathpar}
\stretcharraybig
\begin{array}{lclcl}
  \multicolumn{5}{c}{
    \hspace{-8mm}
    {op} \in \mathtt{Operation\; Name} \spc \spc
    {s} \in \mathtt{Session\; Id} \spc\spc
    {i} \in \mathtt{Effect\; Id} \spc\spc
  }\\
  \eff & \in & \mathtt{Effect} & \coloneqq &  (s,i,op,v)\\
  F_{op} & \in & \mathtt{Op.\,Def.} & \coloneqq & \set{\eff} \rightarrow \eta\\
  \EffSoup & \in & \mathtt{Eff\,Soup}	  & \coloneqq & \set{\eff} \\
  \visZ,\soZ &	\in & \mathtt{Relations} & \coloneqq & \set{(\eff,\eff)} \\
  {\E} 		& \in & \mathtt{Exec\;State}  & \coloneqq & \Exec \\
\end{array}
\end{smathpar}
\end{minipage}
%

\begin{smathpar}
\begin{array}{c}
\ssn(s,\_,\_,\_) = s \spc\spc
\id(\_,j,\_,\_) = j \spc\spc
\oper(\_,\_,op,\_) = op \spc\spc
\rval(\_,\_,\_,v) = v\\
\end{array}
\end{smathpar}

\textbf{Auxiliary Reduction} \;
  \fbox{\(\auxred{S} {(\E,op_{<s,i>})} {} {(\E',\eff)}\)}\\

\begin{minipage}{\textwidth}
\rulelabel{Oper}
\begin{smathpar}
\stretcharraybig
\begin{array}{l}
\RuleTwo
{
%\Theta(\rho \mapsto (v,cache)) \qquad
F_{op}(S) = \eta \qquad
\ssn(\eta) = s \qquad 
\id(\eta) = i \qquad
%\{\eff'\} = \EffSoup_{({\sf SessID}=s,\,{\sf SeqNo}=i-1)}\\
\EffSoup' = \EffSoup \cup \{\eff\} \qquad S \subseteq \EffSoup\\
\visZ' = \visZ \cup S \times\{\eff\}\qquad
\soZ' = \soZ \cup \{(\eta',\eta) \,|\, \eta'\in \EffSoup \conj 
    \ssn(\eta')=s \conj \id(\eta')<i\}\qquad
%\soZ' = (\soZ^{-1}(\eff') \cup \eff') \times\{\eff\} \cup \soZ
}
{
  \auxred {S} {((\EffSoup,\visZ,\soZ), op_{<s,i>}))}
  {} {((\EffSoup',\visZ',\soZ'),\eta)}
}
\end{array}
\end{smathpar}
\end{minipage}
\vspace{4mm}\\
\textbf{Operational Semantics} \;
  \fbox{\((\E,op_{<s,i>}) \;\xrightarrow{V}\; (\E',\eff)\)}\\
\begin{minipage}{2.8in}
\rulelabel{vis Exec}
\begin{smathpar}
\stretcharraybig
\begin{array}{l}
\RuleTwo
{
     V \subseteq E.A \spc  R^{-1}_{V}(\{\eta\}) \subseteq V \spc  (vis \subseteq r_k) \spc 
  V'= \left \lfloor V \right \rfloor_V \\
  \auxred {V'} {(E, op_{<s,i>}))}
    {} {(E',\eta)} 
}
{
  (\E,op_{<s,i>}) \;\xrightarrow{V}\; (\E', \eff)
}
\end{array}
\end{smathpar}
\end{minipage}
\hspace{12 mm}
\begin{minipage}{2.3in}
\rulelabel{non-vis Exec}
\begin{smathpar}
\stretcharraybig
\begin{array}{l}
\RuleTwo
{
 
  V \subseteq E.A \spc  R^{-1}_{V}(\{\eta\}) \subseteq V\spc (vis \nsubseteq r_k )\\
  \auxred {V} {(E,op_{<s,i>}))}
    {} {(E',\eta)}
}
{
  (\E,op_{<s,i>}) 
    \;\xrightarrow{V}\;
  (\E',\eff) 
}
\end{array}
\end{smathpar}
\end{minipage}


\caption{Core Operational semantics of a replicated data store.}
\label{sem:oper}
\end{figure*}



%--- Section intro
In this section we present the reduction reules, representing our
consistency preservation technique.
Figure \ref{fig:semantics} represents the set of rules defining the
auxiliary relation ($\hookrightarrow$), and reduction relation 
($\rightarrow$) over executions. The latter is parametrized over a set $V$,
that stands for the set of effects that were present at the replica
taking the step. Obviously $V$ must be a subset of all effects produced
in the system at any moment, however, there is no other restrictions,
since we only assume eventual consistency in the underlying store.

%--- The [OPER] rule
Let's now go over the auxiliary reduction 
\rulelabel{oper},
that represtns the procedure of producing a new effect $\eff$, by witnessing a set
of effects $S$. 
We formally define an effect to be a tuple $(s,op,v)$, representing the
session and the operation name that $\eff$ is the result of its execution and the value
that the replica returned as a responce to the operation.
The rule explains how the execution state changes after producing an
effect at a replica. Specifially, in the new state, the effect soup
$\EffSoup'$ must
now contain the newly created effect $\eff$, and the relations $\visZ'$
and $\soZ'$ must capture the fact that the set $S$ was made
visible to $\eta$, and all effects in the initial state from the same
sesison, are in session order with $\eff$.


%--- Why there is two different rules for (->) relation
Now we explain the rules for reduction relation $(\xrightarrow{V})$,
namely  \rulelabel{vis exec} and
\rulelabel{non-vis exec}, which represent taking a step respectively
under a waiting and under a non-waiting contract. The seperation in
these rules, comes from the inherent differences in the enforcement of
waiting and non-waiting contracts. In waiting contracts, where the last
($k$'th) edge is $\soZ$, we are simply required to make sure that the
set of dependency effects, is
present at the replica, before execution the operation and if not, block
the operation temporarily. In other words, these contracts only specify
the \emph{minimum} set of effects that must be made visible to an
operation. On the other hand, the non-waiting contracts, that end with a
$\visZ$ edge, allow us to pick any subset of the  effects at the
replica and make them visible to the operation as long as it does not
violate the contracts' logical formula. In an extreme case, we can
always pick the empty set and make it visible to operations, which
would trivially guarantee the presevation of the contract, since no effect
would satisfy its left hand side. 

%--- Explaining the operational rules
The precondition of both reduction rules, include an auxiliary
reduction, explaining the changes that occur in the execution state $\E$
when an operation is performed. Note that for non-waiting contracts
(i.e. when $vis\subseteq r_k$), 
not all effects available at the replica are shown to the operation, but
only a subset of it $V'$ is shown, which is defined to be the  maximally
closed subset of $V$ under $R$.
Moreover, both rules contain
$R_V^{-1}(\eff)\subseteq V$ which is basically guarantees the
preservation of the required consistency level by taking the step. Note
that for non-waiting executions, this is trivially satisfied, since the
subset $V'$ that is made visible to the contract, by definition includes
the inverse of relation $R$. However, conisdering the stepwise
definition of $R_V^{-1}$, this condition requires the presence of
$\soZ^{-1}(\eff)$ at the replica for waiting operations, which can be
interpreted as \emph{blocking} the operation temporarily.







%=================================================
%--------- Theorem on correctness of enforcement
%=================================================
\subsection{Soundness}
Here, we present a theorem showing that the contract would never be
violated, if the store is in a consistent state and takes a reduction
step. We start by defining what we mean a consistent state of the
execution: 
\begin{definition}
For a given contract ($\psi=\forall(a,b). a
\xrightarrow{Q} b \Rightarrow a \xrightarrow{vis} b $), we define a set
of effects S, to be  $\psi$-consistent under an execution $E$, if and
only if: 
$\forall (\eff \in S). \forall(a\in E.\EffSoup). Q(a,\eta)
\Rightarrow a \in S$ 
\\(We use definition of $Q^{-1}$ from the previous section to formally
define: $Q(a,b) \iff a \in Q_{E.A}^{-1}(b)  $)
\end{definition}

\begin{theorem}
For all operation executions of the form: 
$$
(\E,op_{<s,i>}) 
    \;\xrightarrow{V}\;
  (\E',\eff) 
,$$ 
if $V$ is $\psi$-consistent then $(V\cup\{\eta\})$ is also
$\psi$-consistent.
\end{theorem}
\begin{proof}
Sine $\eff$ is the only effect added to $V$, we only need to check the
following property for $\eta$, in order to prove $\psi$-consistency of
$V\cup\{\eta\}$:
$$\forall (a \in E.A). Q(a,\eta) \Rightarrow vis(a,\eta)  $$
\\
(1) $Q(a,\eta) \spc$ (hypothesis)
\\(2) $a \in Q^{-1}_V (\eta) \spc$  (we need to prove this)
\\(3) $a \in V \spc$ (premise of the rules and (2))
\\(4) $V$ is $\psi$-consistent so from (3) and definition of
consistency, the property holds for $a$
\paragraph{} In order to prove (2), we \emph{probably} have to destruct the contract and
show that if it ends with $so$, the property trivially holds since the
vis set of $\eta$ is fixed. For the other case the premise of the rule
[VIS Exec] will have the enought information to prove.
\end{proof}
\newpage



%=================================================
%--------- Theorem on maxVis and minWait
%=================================================
\subsection{Optimality}
Here we will present two theorems, regarding the optimality of our
approach. Firstly, we show that there
cannot be a larger set to be made visible to an operation than what
our reduction rules specify. Moreover, we also prove the liveness property
of our semantics, which guarantees that all replicas will take a step,
given the proper set of effect. Considering the eventual delivery of all
updates that is guaranteed by the underlying store, the theorem shows that the system would never get
stuck.
%
%Subsection of the maximality of the set made visible and liveness
%properties 
%
For all operation executions of the form: 
$(\E,op_{<s,i>}) 
    \;\xrightarrow{V}\;
  (\E',\eff) 
,$
the set of effects made visible to $\eta$ is maximal. i.e. for all
execution states $E''$ such that:
$(\E,op_{<s,i>})\;\xrightarrow{V}\;(\E'',\eff)$
then    
$\small (\E''.vis^{-1}(\eta)) \subseteq
(\E'.vis^{-1}(\eta))$
\begin{proof}
\begin{footnotesize}
\vspace{-2mm}\rule{\textwidth}{1pt}\\ \vspace{0mm} \\
We prove the theorem by contradiction:
\begin{fmathpar}
\begin{array}{ll}
H_0: & (\E,op_{<s,i>}) \;\xrightarrow{V}\; (\E',\eff) \\
H_1: & a \in V \\ 
H_2: & (a,\eta) \not\in E'.\visZ\\
H_3: & (E'.A,E'.\visZ \cup \{(a,\eta)\},E'.\soZ) \models  \psi[\eta/\hat{\eta}]\\
H_4: & (\trunc{R}^{-1}_V(a) = \trunc{R}^{-1}_{E.A}(a)) \\
G_0: & \bot
\end{array}
\end{fmathpar}
Now we call {\scriptsize $(E'.A,E'.\visZ \cup \{(a,\eta)\},E'.\soZ)$} as
{\scriptsize $E''$} and derive
the following from $H_3$:
\begin{fmathpar}
\begin{array}{ll}
H_5: & E'' \models \forall x. x\xrightarrow{R}\eta \Rightarrow x
\xrightarrow{\visZ} \eta
\end{array}
\end{fmathpar}
because $E''$ defines the universe of quantification (and since
$E''.A=E'.A$), we get the
following:
\begin{fmathpar}
\begin{array}{ll}
H_6: & \forall (x\in E'.A). E'' \models x\xrightarrow{R}\eta \Rightarrow x
\xrightarrow{\visZ} \eta
\end{array}
\end{fmathpar}
and is rewritten as the following:
\begin{fmathpar}
\begin{array}{ll}
H_7: & \forall (x\in E'.A). (E'' \models x\xrightarrow{R}\eta) \Rightarrow
(E'' \models x
\xrightarrow{\visZ} \eta)
\end{array}
\end{fmathpar}
Now by inversion on $H_0$ we get two cases, one of which is trivial. We
skip the formal proof for it but it is easy to see that in [LB exec]
case, ALL effects in $V$ are made visible to $\eta$, so the set is
trivially 
maximal, i.e. $H_1$ and $H_2$ yield $\bot$. For the other case (UB
exec), we get the following:
\begin{fmathpar}
\begin{array}{ll}
H_8: & V' = \left \lfloor V  \right \rfloor_{\mathtt{max}}\\
H_9: & \auxred{V'} {(\E,op_{<s,i>})} {} {(\E',\eff)}
\end{array}
\end{fmathpar}
by inversion on $H_9$ we get $H_{10}$ and from that and from $H_2$, following a similar argument from the proof of
theorem 1, we get $H_{11}$: 
\begin{fmathpar}
\begin{array}{ll}
H_{10}: & \visZ' = \visZ \cup V' \times \{\eta\}\\
H_{11}: & a \not\in V'
\end{array}
\end{fmathpar}
Now by denoting the interpretation of $R$ under $E''$ as $R''$,
$H_7$ can be rewritten as follows: 
\begin{fmathpar}
\begin{array}{ll}
H_{12}: &  \forall(x\in E'.A). R''(x,\eta) \Rightarrow \visZ''(x,\eta)\\
\end{array}
\end{fmathpar}
Now by inversion on $H_8$, we get the following:
\begin{fmathpar}
\begin{array}{lll}
& H_{13}: & V' \in \left \lfloor V  \right \rfloor \\
& H_{14}: & \not\exists V'' \in \left \lfloor V  \right \rfloor.
|V''|>|V'|\\
(\mathtt{from \; H_{13}}) & H_{15}: & V' \subseteq V \; \wedge \;
(\trunc{R})^{-1}_V(V') \subseteq V' \; \wedge \\ & & (\trunc{R})^{-1}_V(V')  =
(\trunc{R})^{-1}_{E.A}(V') 
\end{array}
\end{fmathpar}
Now we can destruct $R$, where we get multiple cases, only two of which
are non-trivial, ($R=\visZ$) and ($R=\trunc{R};\visZ$)
\begin {itemize}
\item{\bf Case1}($R=\visZ$):\\
$\trunc{R}=\nullR$, thus V itself satisfies the requirements in $H_{15}$
and we get that {\scriptsize $(V=\left \lfloor V  \right \rfloor_{\mathtt{max}})$} and
the following holds:
\begin{fmathpar}
\begin{array}{ll}
H_{16}: & V = V'
\end{array}
\end{fmathpar}
which results in contradiction from $H_1$ and $H_{11}$.
\\
\item{\bf Case2}($R=\trunc{R};\visZ$):\\
Since {\scriptsize $|V'\cup\{a\}|>|V'|$} we have the following:
\begin{fmathpar}
\begin{array}{ll}
H_{17}: & (V'\cup\{a\}) \not\in \left \lfloor V  \right \rfloor 
\end{array}
\end{fmathpar}
which based on the definition yields that the conditiones for holding
the above relation are not true, i.e.
\begin{fmathpar}
\begin{array}{ll}
H_{18}: &  \neg ((V'\cup\{a\}) \subseteq V \; \wedge \;
(\trunc{R})^{-1}_V(V'\cup\{a\}) \subseteq (V'\cup\{a\}) \; \wedge \\ & 
(\trunc{R})^{-1}_V(V'\cup\{a\})  =
(\trunc{R})^{-1}_{E.A}(V'\cup\{a\}))
\end{array}
\end{fmathpar}
or equally:
\begin{fmathpar}
\begin{array}{ll}
H_{19}: &   (V'\cup\{a\}) \not\subseteq V\; \vee \\
        &  (\trunc{R})^{-1}_V(V'\cup\{a\}) \not\subseteq
	(V'\cup\{a\}) \; \vee \\ & 
(\trunc{R})^{-1}_V(V'\cup\{a\})  \not=
(\trunc{R})^{-1}_{E.A}(V'\cup\{a\})
\end{array}
\end{fmathpar}
By inversion on the above, we get three cases, two of which are
trivial. The last conjunct can't hold because of $H_4$ and the first one
also contradics with $H_1$ and $H_{15}$. Thus, we are left with only one
case: 
\begin{fmathpar}
\begin{array}{ll}
H_{20}: & (\trunc{R})^{-1}_V(V'\cup\{a\}) \not\subseteq
	(V'\cup\{a\})
\end{array}
\end{fmathpar}
Now, from the second conjunct in $H_{15}$ we know that it should be the
case that: 
\begin{fmathpar}
\begin{array}{lll}
\hspace{-30mm} (\mathtt{\tiny from\; H_{15}: (\trunc{R})^{-1}_V(V')
\subseteq V' }) \quad & H_{21}: & ((\trunc{R})^{-1}_V(a) \not\subseteq
	(V'\cup\{a\})) 
\end{array}
\end{fmathpar}
The above hypothesis yields the existance of $c\not= a$ such that:
\begin{fmathpar}
\begin{array}{ll}
H_{22}: & c \in (\trunc{R})^{-1}_V(a)\\
H_{23}: & c \not\in V'
\end{array}
\end{fmathpar}
Now, by rewriting ({\scriptsize $R=\trunc{R};\visZ$}) in $H_{12}$ we get $H_{24}$, which can be
rewritten again into $H_{25}$ from the definition:
\begin{fmathpar}
\begin{array}{lrll}
H_{24}: & \forall (x\in E'.A). ((\trunc{R};\visZ)''(x,\eta) & \Rightarrow
& \visZ''(x,\eta)) \\
H_{25}: & \forall (x\in E'.A). (\exists b.\trunc{R}''(x,b) \; \wedge & & \\
& \visZ''(b,\eta) & \Rightarrow & \visZ''(x,\eta))
\end{array}
\end{fmathpar}
Now, we instantiate $H_{25}$ with $x=c$:
\begin{fmathpar}
\begin{array}{ll}
H_{26}: & \exists b.\trunc{R}''(c,b) \; \wedge \;
 \visZ''(b,\eta)  \Rightarrow  \visZ''(c,\eta)
\end{array}
\end{fmathpar}
we can replace $\trunc{R}''$ with $\trunc{R}'$ in above definition,
since from $H_3$, the only difference in interpretation under $E'$ and
$E''$ is the extra element $(a,\eta)$ in $E''.\visZ$ which does not
effect $\trunc{R}''(c,b)$:
\begin{fmathpar}
\begin{array}{ll}
H_{27}: & \exists b.\trunc{R}'(c,b) \; \wedge \;
 \visZ''(b,\eta)  \Rightarrow  \visZ''(c,\eta)
\end{array}
\end{fmathpar}
Moreover, since $c\not= a$, we can replace $\visZ''(c,\eta)$ with
$\visZ'(c,\eta)$:
\begin{fmathpar}
\begin{array}{ll}
H_{28}: & \exists b.\trunc{R}'(c,b) \; \wedge \;
 \visZ''(b,\eta)  \Rightarrow  \visZ'(c,\eta)
\end{array}
\end{fmathpar}
From $H_{15}$ and $H_{22}$ we get $H_{29}$, and $H_{30}$ also holds
trivially from $H_3$:
\begin{fmathpar}
\begin{array}{ll}
H_{29}: & \trunc{R}'(c,a) \\
H_{30}: & \visZ''(a,\eta)
\end{array}
\end{fmathpar}
which can be used in instantiation of  $H_{28}$ with $b=a$ and derive the
following:
\begin{fmathpar}
\begin{array}{ll}
H_{31}: & \visZ' (c,\eta)
\end{array}
\end{fmathpar}
However, we know -from the previously explained argument- that $H_{31}$
results in $H_{32}$, which results in contradiction with $H_{23}$.
\begin{fmathpar}
\begin{array}{ll}
H_{32}: & c \in V'
\end{array}
\end{fmathpar}
\fcolorbox{red}{white}{\color{black} \footnotesize QED.}

\end{itemize}
\end{footnotesize}

\end{proof}
%%%




%
% THE MINIIMAL WAIT THEOREM
%

\begin{theorem}
\label{theorem:three}
For all execution states $E$, if there
exists ($S\subseteq E.A$) such that: 
\begin{smathpar}
\auxred{S} {(\E,op_{<s,i>})} {} {(\E',\eff)} \spc \wedge \spc (\mathtt{S \cup \{\eta\} \; is 
\;} \psi \mathtt{-consistent \; under \;} E' )  
\end{smathpar}
then there exists an execution state $E''$, $\eff'$ and $V\subseteq E.A$ such that:
$((\E,op_{<s,i>})\;\xrightarrow{V}\;(\E'',\eff'))$
\end{theorem}
\begin{proof}
The proof is given by choosing set $V$ to be equal to $S$, and then
considering two cases, where either $S$ or $\left \lfloor S \right
\rfloor_S$ is shown to operations, if the contract is respectively
waiting and non-waiting.
A detailed proof can be found in appendix
\ref{app:proof3}.
\\
\end{proof}




%=================================================
%--------- How it can be generalized for all contracts
%=================================================
\subsection{Generalization}
\label{subsec:generalization}
Here we explain how our idea can be easily extended to cover all
contracts (we previously omitted the conjunction of props). Moreover, we
explain how multiple levels of consistency can be maintained
simultaneously. 

Firstly, we argue that since all props, basically specify the
dependencies of operations, therefore the dependencies defined by a
conjunct of props is simply the union of their individual dependencies.
Therefore, by replacing the single premise $R_V^{-1}(\eta) \subseteq V$
in the operational rules with the following conjunction, we extend the
dependency checks, to make sure that ALL contracts are preserved, before
taking a step:
\begin{smathpar}
\bigwedge_{1 \leq i \leq k} (R_i)_V^{-1}
\end{smathpar}

Moreover, we can extend the visible set $V$ of the rules, to a sequence
of sets, each corresponding a consistency level: $V_{\psi_i}$. Now by
simply extending the notion of operations to also include a contract, we
can derive the following form of execution steps: 
\begin{smathpar}
(\E,op_{<s,i,\psi>}) 
    \;\xrightarrow{V_{\psi}}\;
  (\E',\eff) 
\end{smathpar}
This represents our implementation, in the sense that all operations
work on a subset of available effects at the replica. The subset, is
maintained according to the contract assosiated with the operation, and
the is guaranteed to preserve the consistency requirements. 

