\section {Specification Language} 
\label{sec:ctrt_language}
\begin{wrapfigure}{r}{0.48\textwidth}
\begin{minipage}{0.5\textwidth}
\centering
  \begin{smathpar}
  \begin{array}{lclcl}
		r & \in & \texttt{rel.seed} & \coloneqq & \visZ \ALT \soZ \ALT r \cup r \\
               R & \in & \texttt{relation} & \coloneq &  r \ALT r^*  \ALT r;R \ALT r^*;R   \\
	     \pi & \in & \texttt{prop} & \coloneqq & \forall(\eff,\eff').
      ~\eff \xrightarrow{R} \eff' ~\Rightarrow~ \eff \xrightarrow{\visZ} \eff'\\
		\psi & \in & \texttt{spec} & \coloneqq & \pi \ALT \pi \conj \pi\\
  \end{array}
  \end{smathpar}
\caption{Syntax of the Specification Language}
\label{fig:ctrt_syntax}
\end{minipage}
\end{wrapfigure}

In this section we complete the introduction of our tool, by formally
presenting the specification language used by developers to fine-tune
the consistency requirements for each \tool's environement.

Formal syntax of our specification language is presented in
figure \ref{fig:ctrt_syntax}.
The language allows the definition of  \propS{}s, which are FOL formulae, where a 
the visibility relation between two effects, is set as a logical
consequence of a certain dependency relation $\mathtt{R}$ holding between them.
By a combination of filtering and blocking
techniques on each environment, \tool guarantees the preservation of any
given \propS{} in this language.
The type \relationS{} which is used to define dependencies between
effects in \propS{}s, is a sequence of \seedS{}s where each of them is a $\visZ$ or 
$\soZ$ relation or the union of them both over the set of effects. In order to
configure each environment, the developers are required to write a
\specS{} for each of them, that is consisted of conjunctions of \propS{}s, which
simply allows developers to eliminiate multiple types of anomalous
behavior from each environment. For example, the developers of the
comment management application, can simply write $\psi_1\wedge \psi_2$
for their \readC{} operation and prevent \emph{both} anomalies explained
in the previous section.

Now we syntactically  classify propositions into \emph{lower bound} (LB) and
\emph{upper bound} (UB) and hybrid types, and show that they completely
align with types of
anomalies previously mentioned. 
\begin{itemize}
\item {\bf LB}: We define a \propS{} to be of type LB, 
if its dependency relation $R$, ends with an $\soZ$ relation, i.e. is of
the following from: $\forall(\eff,\eff'). \eff \xrightarrow{r_1;r_2;...;\soZ} \eff'
\Rightarrow \eff \xrightarrow{\visZ} \eff'$.
Observe that these contracts simply put a lower bound on the set of
effects each operation must witness, by defining a certain set of
dependency for each effect, that must be visible to it, which makes the
blocking technique very suitable for easily maintaining them.
\item {\bf UB}:  These propositions are similarly defined as \propS{}s
with dependency relations ending with $\visZ$, or of the following form: 
$\forall(\eff,\eff'). \eff \xrightarrow{r_1;r_2;...;\visZ} \eff'
\Rightarrow \eff \xrightarrow{\visZ} \eff'$.
This type of \propS{}s, put an upper bound on the set of effects 
a replica should make visible to each operation when executing it, by
enforcing that if an effect is made visible, certain set of dependent
effects must also be made visible. This clearly resembles the
filteration technique, where only a subset of available effects can enter
the consistent environments.
\item {\bf Hybrid}: The \propS{}s whose dependency relation ends with
$\visZ\cup \soZ$, which require both blocking and filteration to be
maintained. 
\end{itemize}
We can extend the above definitions to \specS{}s also, by defining an LB
(UB) \specS{} to contain only LB (UB) \propS{}s. Hybrid \specS{}s are also
defined as the ones that are neither LB or UB.

We finish this section by presenting weak consistency guarantees from
Terry et. al. in figure \ref{fig:ctrt_rep}, which shows the generality
of our simple specification language. We also present a simple way of
representing contracts as graphs where the left-hand-side of the
contracts are depicted as the sequence of edges relating two effects,
and the right-hand-sinde (or what must be enforced by replicas at the
execution time) is represented as the single dashed $\visZ$ edge.
\begin{figure}[t]
        \centering
	\begin{subfigure}[b]{0.3\textwidth}
	\caption{Read My Writes: \\
	$\forall(\eff,\eff'). \eff \xrightarrow{\soZ} \eff \Rightarrow
	\eff \xrightarrow{\visZ} \eff'$}
	\includegraphics[scale=0.5]{Figures/rmw.pdf}
	\end{subfigure}
	%
	\vrule
	\hfill
	\begin{subfigure}[b]{0.3\textwidth}
	\caption{Monotonic Reads: 
	\\	$\forall(\eff,\eff'). \eff \xrightarrow{\visZ;\soZ} \eff \Rightarrow
	\eff \xrightarrow{\visZ} \eff' $}
	\includegraphics[scale=0.5]{Figures/mr.pdf}
	\end{subfigure}
	%
	\vrule
	\hfill
	\begin{subfigure}[b]{0.25\textwidth}
	\caption{Monotonic Writes:
	\\	$\forall(\eff,\eff'). \eff \xrightarrow{\soZ;\visZ} \eff \Rightarrow
	\eff \xrightarrow{\visZ} \eff' $}
	\includegraphics[scale=0.5]{Figures/mw.pdf}
	\end{subfigure}
	%
%	\begin{subfigure}[b]{0.24\textwidth}
%	\centering
%	\includegraphics[scale=0.38]{Figures/wfr.pdf}
%	\caption{Writes Follow Reads}
%	\end{subfigure}
\\ \hrulefill

\caption{Known weak consistency guarantees written in our specification
language}
\label{fig:ctrt_rep}
\end{figure}



