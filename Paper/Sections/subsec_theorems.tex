%
%Subsection on the correctness of contract 
%
\begin{definition}
For a given contract ($\psi=\forall(a,b). a
\xrightarrow{R} b \Rightarrow a \xrightarrow{vis} b $), we define a set
of effects $S$, to be  $\psi$-consistent under an execution $\E$, if and
only if: 
$\forall (\eff \in S). \forall(a\in \E.\EffSoup). R(a,\eta)
\Rightarrow a \in S$ 
\\Where the definition of $R(a,b)$ is based on the definition of $R^{-1}$ from
\ref{subsec:prelim}, as follows: $R(a,b) \iff a \in R_{E.A}^{-1}(b)  $
\end{definition}

\begin{theorem}
\label{theorem:one}
For all reduction steps 
$
\; (\E,op_{<s,i>}) 
    \xrightarrow{V}
  (\E',\eff)  
$,
if $V$ is $\psi$-consistent under $\E$, then $(V\cup\{\eta\})$ is
$\psi$-consistent under $\E'$.
\end{theorem}
\begin{proof}
The proof is followed by the fact that the premises of the execution
step, contains $R_V^{-1}(\eta)\subseteq V$, which basically states that the set made visible to
the operation, must not violate the contract $\psi$. A more detailed proof can
be found in appendix \ref{app:proof1}.
\\
\end{proof}





