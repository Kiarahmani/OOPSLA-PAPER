\begin{abstract}
Linearizability and Sequential Consistency are well-understood
correctness criteria for concurrent applications, for which equally
well-understood implementation techniques (e.g., 2PL) exist for
enforcement. These criteria are however too strong for
distributed applications operating over replicated data. The inherent
tradeoff between consistency and availability in a distributed setting
makes it impossible for these applications to enforce strong
consistency, thus requiring them to adopt weak alternatives.
Unfortunately, the lack of standardized definitions and enforcement
techniques for weak consistency has incentivized programmers to adopt
\emph{ad hoc} implementation techniques for their enforcement, which
don't lend themselves to a uniform reasoning technique, making it
hard to reason about their correctness.  

In this paper, we describe a principled approach to derive the
enforcement mechanisms for various forms of weak consistency. Our
approach is parametric over the declarative semantics of weak
consistency, and generates a provably optimal enforcement mechanism
for any given consistency semantics. We implement our approach in a
tool called \tool that accepts a declarative specification of a weak
consistency guarantee, and extends an eventually consistent data store
with the capability to enforce the guarantee.  Our method thus
works for  any off-the-shelf key-value store. We demonstrate the
applicability of our approach by automatically deriving enforcement
mechanisms for various well-known weak consistency guarantees in the
literature. Experiments show that the performance of the derived
mechanisms is comparable to (utmost 100\% worse than) their
corresponding hand-written implementations, thus attesting to the
practical utility of \tool.
\end{abstract}
