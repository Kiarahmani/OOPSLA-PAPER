\section{Formalization}
In this chapter we first present our specification language and then
introduce the core opeational semantics of our tool.
\subsection{Specification Language}
Following is the formal syntax of contract language in our system:
\begin{wrapfigure}{r}{0.48\textwidth}
\begin{minipage}{0.5\textwidth}
\centering
  \begin{smathpar}
  \begin{array}{lclcl}
		r & \in & \texttt{rel.seed} & \coloneqq & \visZ \ALT \soZ \ALT r \cup r \\
               R & \in & \texttt{relation} & \coloneq &  r \ALT r^*  \ALT r;R \ALT r^*;R   \\
	     \pi & \in & \texttt{prop} & \coloneqq & \forall(\eff,\eff').
      ~\eff \xrightarrow{R} \eff' ~\Rightarrow~ \eff \xrightarrow{\visZ} \eff'\\
		\psi & \in & \texttt{spec} & \coloneqq & \pi \ALT \pi \conj \pi\\
  \end{array}
  \end{smathpar}
\caption{Syntax of the Specification Language}
\label{fig:ctrt_syntax}
\end{minipage}
\end{wrapfigure}

\\ The language is general enough to cover all  the known consistency
levels in the context:
\begin{figure}[h]
  \begin{smathpar}
  \begin{array}{ll}
		\texttt{ Read My Write (RMW): }  & \forall(a,b). a \xrightarrow{so} b ~\Rightarrow~ a \xrightarrow{\visZ} b\\
		\texttt{ Monotonic Reads (MR): }  & \forall(a,b). a \xrightarrow{so;vis} b ~\Rightarrow~ a \xrightarrow{\visZ} b\\
		\texttt{ Monotonic Writes (MW): }  & \forall(a,b). a \xrightarrow{vis;so} b ~\Rightarrow~ a \xrightarrow{\visZ} b\\
		\texttt{ Writes Follow Reads (WFR): }  & \forall(a,b). [a \xrightarrow{vis;vis} b ~\Rightarrow~ a \xrightarrow{\visZ} b] \wedge  
		[a \xrightarrow{vis;so;vis} b ~\Rightarrow~ a \xrightarrow{\visZ} b]\\


  \end{array}
  \end{smathpar}
\caption{Well-known consistency requirements, written in our specification language}
\label{fig:ctrt}
\end{figure}

\\ Contracts written in our language, can be visualized by simple graphs. For example, \texttt{(MR)} contract from the above figure can be represented as:

\begin{figure}[h]
\includegraphics[scale=0.6]{../Figures/MR.pdf}
\caption{Well-known consistency requirements, written in our specification language}
\label{fig:ctrt}
\end{figure}



\newpage
\subsection{Operational Semantics}
Our system is consisted of a number of replicas, each of which itself is consisted of an array of caches. Each cache is consistent according to a part of the contract:
\begin{figure}[h]
\includegraphics[scale=0.6]{../Figures/Caches.pdf}
\caption{Replica model in our system, where each replica maintains a sequense of logical caches according to the given contract}
\end{figure}

\\ Here we explain the the core operational semantics of our algorithm, which for simplicity reasons, 
is parametrized over a single contract 
$\psi = \forall (a,b). a \xrightarrow{r_1;r_2;...;r_k} b  \Rightarrow a \xrightarrow{vis} b$. 
However this can be easily  genralized to maintain different levels of consistency. 
\begin{figure*}[t]
\raggedright
%
\textbf{Auxiliary Definitions}\\
%
\begin{minipage}{\columnwidth}
\begin{smathpar}
\stretcharraybig
\begin{array}{lclcl}
  \multicolumn{5}{c}{
    {\delta} \in \mathtt{Replicated\; Data\; Type} \spc\spc
    {v} \in \mathtt{Value}\spc\spc
    {op} \in \mathtt{Operation\; Name}
  }\\
  \multicolumn{5}{c}{
    {s} \in \mathtt{Session\; Id} \spc\spc
    {i} \in \mathtt{Effect\; Id} \spc\spc
    {r} \in \mathtt{Replica\; Id}
  }\\
  \eff & \in & \mathtt{Effect} & \coloneqq &  (s,i,op,v)\\
  F_{op} & \in & \mathtt{Op.\,Def.} & \coloneqq & \delta \rightarrow \eta\\
  \EffSoup & \in & \mathtt{Eff\,Soup}	  & \coloneqq & \set{\eff} \\
  \visZ,\soZ &	\in & \mathtt{Relations} & \coloneqq & \set{(\eff,\eff)} \\
  {\E} 		& \in & \mathtt{Exec\;State}  & \coloneqq & \Exec \\
  \Theta  & \in & \mathtt{Store}      & \coloneqq & r \mapsto (\delta,\set{\eff}) \\
  {\sigma} 	& \in & \mathtt{Session} 					 	& \coloneqq & \cdot \ALT op::\sigma \\
  \Sigma 		& \in & \mathtt{Session\;Soup}   	 	& \coloneqq &
        \langle s, i, \sigma \rangle \pll \Sigma \ALT \emptyset \\
\end{array}
\end{smathpar}
\end{minipage}
%

\begin{smathpar}
\begin{array}{c}
\ssn(s,\_,\_,\_) = s \spc\spc
\id(\_,j,\_,\_) = j \spc\spc
\oper(\_,\_,op,\_) = op \spc\spc
\rval(\_,\_,\_,n) = n\\
\end{array}
\end{smathpar}

\vspace{5mm}
\textbf{Auxiliary Reduction} \;
  \fbox{\(\auxred{\Theta} {(\E,\langle s,i,op \rangle)} {r} {(\E',\eff)}\)}\\

\begin{minipage}{\textwidth}
\rulelabel{Oper}
\begin{smathpar}
\stretcharraybig
\begin{array}{l}
\RuleTwo
{
\Theta(r \mapsto (v,S)) \qquad
F_{op}(v) = \eta \qquad
\ssn(\eta) = s \qquad 
\id(\eta) = i \qquad
%\{\eff'\} = \EffSoup_{({\sf SessID}=s,\,{\sf SeqNo}=i-1)}\\
\EffSoup' = \EffSoup \cup \{\eff\} \\
\visZ' = \visZ \cup S \times\{\eff\}\qquad
\soZ' = \soZ \cup \{(\eta',\eta) \,|\, \eta'\in \EffSoup \conj 
    \ssn(\eta')=s \conj \id(\eta')<i\}\qquad
%\soZ' = (\soZ^{-1}(\eff') \cup \eff') \times\{\eff\} \cup \soZ
}
{
  \auxred {\Theta} {((\EffSoup,\visZ,\soZ), \langle s,i,op \rangle}
  {r} {((\EffSoup',\visZ',\soZ'),\eta)}
}
\end{array}
\end{smathpar}
\end{minipage}


\vspace{5mm}
\textbf{Operational Semantics} \;
  \fbox{\((\E,\Theta,\Sigma) \;\xrightarrow{\eff}\; (\E',\Theta',\Sigma')\)}\\

\begin{minipage}{3in}
\rulelabel{EffVis}
\begin{smathpar}
\stretcharraybig
\begin{array}{l}
\RuleTwo
{
  \eta \in \E.\EffSoup \spc
  \Theta(r) = (v,S) \spc
  \eta \not\in S \\
  {\sf deps}_{\psi}(\eta) \subseteq  S \spc
  \Theta' = \Theta[r \mapsto (apply \;\eta \;v, \{\eta\} \cup S)]\\
}
{
  (\E,\Theta,\Sigma) \;\xrightarrow{\eff}\; (\E, \Theta', \Sigma)
}
\end{array}
\end{smathpar}
\end{minipage}
\begin{minipage}{2.3in}
\rulelabel{Exec}
\begin{smathpar}
\stretcharraybig
\begin{array}{l}
\RuleTwo
{
  \auxred{\Theta} {(\E,\langle s,i,op \rangle)} {r} {(\E',\eta)} \\
  \Theta(r) = (v,S) \spc {\sf deps}_{\psi}(\eta) \subseteq   S \spc
}
{
  (\E,\Theta,\langle s,i,op::\sigma \rangle \pll \Sigma) 
    \;\xrightarrow{\eff}\;
  (\E',\Theta,\langle s,i+1,\sigma \rangle \pll \Sigma) 
}
\end{array}
\end{smathpar}
\end{minipage}


\caption{Operational semantics of a replicated data store.}
\label{sem:oper}
\end{figure*}


