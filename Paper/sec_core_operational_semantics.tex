Here we explain the the core operational semantics of our algorithm, which for simplicity reasons, 
is parametrized over a single contract 
$\psi = \forall (a,b). a \xrightarrow{Q} b  \Rightarrow a
\xrightarrow{vis} b$, where $Q=q_1;q_2;...;q_k$. 
\\Before intorducing the semantics, let's define the inverse of the
given composition $Q=q_1;q_2;...;q_k$ according to a given
set of available effects $P$, as a partial function from $\set{\eff}$ to
$\set{\eff}$. The idea is to define the function only if the \emph{necessary information about the relations beetween effects, is present in P}.
\\ $
(q_1;...;q_k)^{-1}_P (S)= (q_1;...;q_{k-1})^{-1}_P (q^{-1}_{k}(S)), \qquad 
if \; (q_k)^{-1}(S) \subseteq P$
\\Trivially, the above definition is based on the following definition of
the inverse of a clause :\\
$q^{-1}(S) = 
\begin{cases}
\bigcup^{}_{e\in S}. \{\eta|(\eta,e) \in E.q \}, \hspace{8.5 mm} if \;
q\in\{so,vis\} \\ 
r_1^{-1}(S)\cup r_2^{-1}(S), \qquad \spc \qquad if \; q=r_1\cup r_2\\
G_{r}(S,\emptyset), \quad \hspace{17.25 mm} \qquad if \; q=r^* 
\end{cases}
$
\\ Where we define the helping function $G$ as follows: 
$G_r(S,R) =
\begin{cases} 
G_r(r^{-1}(S,R\cup r^{-1}(S))), \qquad if r^{-1}(S) \neq \emptyset  \\
R,                          \hspace{36.1mm} otherwise
\end{cases}$ \\
We also define the relation $\succ$ over clauses as follows: $q \succ
p \iff \forall (a,b). (q(a,b) \rightarrow p(a,b)) $
\begin{figure}[t]
\raggedright
\textbf{Auxiliary Definitions}\\ \vspace{-2mm}
%
\begin{minipage}{0.5\textwidth}
\begin{fmathpar}
\begin{array}{lclcl}
  \multicolumn{5}{c}{
    {op} \in \mathtt{Operation\; Name} \spc \spc
    {v} \in \mathtt{Return\; Value} \spc \spc
    {s} \in \mathtt{Session\; Id} \spc\spc
  } 
  \\ 
  \eff & \in & \mathtt{Effect} & \coloneqq &  (s,op,v)\\
  F_{op} & \in & \mathtt{Op.\,Def.} & \coloneqq & \set{\eff} \mapsto v\\
  \EffSoup & \in & \mathtt{Eff\,Soup}	  & \coloneqq & \set{\eff} \\
  \visZ,\soZ &	\in & \mathtt{Relations} & \coloneqq & \set{(\eff,\eff)} \\
  {\E} 	& \in & \mathtt{Exec\;State}  & \coloneqq & \Exec \\
\end{array}
\end{fmathpar}
\end{minipage}
%

\vspace {3mm}

\textbf{Auxiliary Reduction} \; \\
\fcolorbox{black}{pgrey}{\scriptsize \(\auxred{S} {(\E,op_{<s,i>})} {} {(\E',\eff)}\)}\\
\begin{minipage}{0.9\textwidth}
\vspace{2mm}
\rulelabel{Oper}
\vspace{-2mm}
\begin{fmathpar}
\stretcharraybig
\begin{array}{l}
\RuleTwo
{
%\Theta(\rho \mapsto (v,cache)) \qquad
S \subseteq \EffSoup \qquad F_{op}(S) = v \qquad
\eta \not\in S \qquad
\eff = (s,op,v) \qquad  \\
%\id(\eta) = i \qquad
%\{\eff'\} = \EffSoup_{({\sf SessID}=s,\,{\sf SeqNo}=i-1)}\\
\EffSoup' = \EffSoup \cup \{\eff\}  \qquad
\visZ' = \visZ \cup S \times\{\eff\}\qquad
\soZ' = \soZ \cup \{(\eta',\eta) \,|\, \eta'\in \EffSoup_{({\sf
SessID}=s)}      \}\qquad
%\soZ' = (\soZ^{-1}(\eff') \cup \eff') \times\{\eff\} \cup \soZ
}
{
  \auxred {S} {((\EffSoup,\visZ,\soZ), op_{<s,i>}))}
  {} {((\EffSoup',\visZ',\soZ'),\eta)}
}
\end{array}
\end{fmathpar}
\end{minipage}
\vspace{4mm}\\
\textbf{Operational Semantics} \; \\
  \fcolorbox{black}{pgrey}{\scriptsize \((\E,op_{<s,i>}) \;\xrightarrow{V}\; (\E',\eff)\)}\\
\vspace{2mm}
\begin{minipage}{0.45\textwidth}
\rulelabel{UB Exec}
\vspace{-2mm}
\begin{fmathpar}
\stretcharraybig
\begin{array}{l}
\RuleTwo
{
  \visZ \subseteq r_k \spc
  V \subseteq E.A \spc  
  V'= \left \lfloor V \right \rfloor_V \spc
  \\ %   R^{-1}_{V}(\eta) \subseteq V' \\
  \auxred {V'} {(E, op_{<s,i>}))}
    {} {(E',\eta)} 
}
{
  (\E,op_{<s,i>}) \;\xrightarrow{V}\; (\E', \eff)
}
\end{array}
\end{fmathpar}
\end{minipage}
\hfill
\begin{minipage}{0.45\textwidth}
\rulelabel{LB Exec}
\vspace{-2mm}
\begin{fmathpar}
\stretcharraybig
\begin{array}{l}
\RuleTwo
{
     \visZ \not\subseteq r_k \spc
     V \subseteq E.A \spc  
     R^{-1}_{V}(\eta) \subseteq V \\
  \auxred {V} {(E, op_{<s,i>}))}
    {} {(E',\eta)} 
}
{
  (\E,op_{<s,i>}) \;\xrightarrow{V}\; (\E', \eff)
}
\end{array}
\end{fmathpar}
\end{minipage}
\\
\vspace{5mm}
\hrulefill\\
\caption{Core Operational semantics of a replicated data store.}
\label{fig:semantics}
\end{figure}




