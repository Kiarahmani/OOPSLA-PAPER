Here we explain the the core operational semantics of our algorithm, which for simplicity reasons, 
is parametrized over a single contract 
$\psi = \forall (a,b). a \xrightarrow{Q} b  \Rightarrow a
\xrightarrow{vis} b$, where $Q=q_1;q_2;...;q_k$. 
\\Before intorducing the semantics, let's define the inverse of the
given composition $Q=q_1;q_2;...;q_k$ according to a given
set of available effects $P$, as a partial function from $\set{\eff}$ to
$\set{\eff}$. The idea is to define the function only if the \emph{necessary information about the relations beetween effects, is present in P}.
\\ $
(q_1;...;q_k)^{-1}_P (S)= (q_1;...;q_{k-1})^{-1}_P (q^{-1}_{k}(S)), \qquad 
if \; (q_k)^{-1}(S) \subseteq P$
\\Trivially, the above definition is based on the following definition of
the inverse of a clause :\\
$q^{-1}(S) = 
\begin{cases}
\bigcup^{}_{e\in S}. \{\eta|(\eta,e) \in E.q \}, \hspace{8.5 mm} if \;
q\in\{so,vis\} \\ 
r_1^{-1}(S)\cup r_2^{-1}(S), \qquad \spc \qquad if \; q=r_1\cup r_2\\
G_{r}(S,\emptyset), \quad \hspace{17.25 mm} \qquad if \; q=r^* 
\end{cases}
$
\\ Where we define the helping function $G$ as follows: 
$G_r(S,R) =
\begin{cases} 
G_r(r^{-1}(S,R\cup r^{-1}(S))), \qquad if r^{-1}(S) \neq \emptyset  \\
R,                          \hspace{36.1mm} otherwise
\end{cases}$ \\
We also define the relation $\succ$ over clauses as follows: $q \succ
p \iff \forall (a,b). (q(a,b) \rightarrow p(a,b)) $
\begin{figure*}[h]
\raggedright
%
\textbf{Auxiliary Definitions}\\
%
\begin{minipage}{\columnwidth}
\begin{smathpar}
\stretcharraybig
\begin{array}{lclcl}
  \multicolumn{5}{c}{
    \hspace{-8mm}
    {op} \in \mathtt{Operation\; Name} \spc \spc
    {s} \in \mathtt{Session\; Id} \spc\spc
    {i} \in \mathtt{Effect\; Id} \spc\spc
  }\\
  \eff & \in & \mathtt{Effect} & \coloneqq &  (s,i,op,v)\\
  F_{op} & \in & \mathtt{Op.\,Def.} & \coloneqq & \set{\eff} \rightarrow \eta\\
  \EffSoup & \in & \mathtt{Eff\,Soup}	  & \coloneqq & \set{\eff} \\
  \visZ,\soZ &	\in & \mathtt{Relations} & \coloneqq & \set{(\eff,\eff)} \\
  {\E} 		& \in & \mathtt{Exec\;State}  & \coloneqq & \Exec \\
\end{array}
\end{smathpar}
\end{minipage}
%

\begin{smathpar}
\begin{array}{c}
\ssn(s,\_,\_,\_) = s \spc\spc
\id(\_,j,\_,\_) = j \spc\spc
\oper(\_,\_,op,\_) = op \spc\spc
\rval(\_,\_,\_,v) = v\\
\end{array}
\end{smathpar}

\textbf{Auxiliary Reduction} \;
  \fbox{\(\auxred{S} {(\E,op_{<s,i>})} {} {(\E',\eff)}\)}\\

\begin{minipage}{\textwidth}
\rulelabel{Oper}
\begin{smathpar}
\stretcharraybig
\begin{array}{l}
\RuleTwo
{
%\Theta(\rho \mapsto (v,cache)) \qquad
F_{op}(S) = \eta \qquad
\ssn(\eta) = s \qquad 
\id(\eta) = i \qquad
%\{\eff'\} = \EffSoup_{({\sf SessID}=s,\,{\sf SeqNo}=i-1)}\\
\EffSoup' = \EffSoup \cup \{\eff\} \qquad S \subseteq \EffSoup\\
\visZ' = \visZ \cup S \times\{\eff\}\qquad
\soZ' = \soZ \cup \{(\eta',\eta) \,|\, \eta'\in \EffSoup \conj 
    \ssn(\eta')=s \conj \id(\eta')<i\}\qquad
%\soZ' = (\soZ^{-1}(\eff') \cup \eff') \times\{\eff\} \cup \soZ
}
{
  \auxred {S} {((\EffSoup,\visZ,\soZ), op_{<s,i>}))}
  {} {((\EffSoup',\visZ',\soZ'),\eta)}
}
\end{array}
\end{smathpar}
\end{minipage}
\vspace{4mm}\\
\textbf{Operational Semantics} \;
  \fbox{\((\E,op_{<s,i>}) \;\xrightarrow{V}\; (\E',\eff)\)}\\
\begin{minipage}{2.8in}
\rulelabel{vis Exec}
\begin{smathpar}
\stretcharraybig
\begin{array}{l}
\RuleTwo
{
     V \subseteq E.A \spc  R^{-1}_{V}(\{\eta\}) \subseteq V \spc  (vis \subseteq r_k) \spc 
  V'= \left \lfloor V \right \rfloor_V \\
  \auxred {V'} {(E, op_{<s,i>}))}
    {} {(E',\eta)} 
}
{
  (\E,op_{<s,i>}) \;\xrightarrow{V}\; (\E', \eff)
}
\end{array}
\end{smathpar}
\end{minipage}
\hspace{12 mm}
\begin{minipage}{2.3in}
\rulelabel{non-vis Exec}
\begin{smathpar}
\stretcharraybig
\begin{array}{l}
\RuleTwo
{
 
  V \subseteq E.A \spc  R^{-1}_{V}(\{\eta\}) \subseteq V\spc (vis \nsubseteq r_k )\\
  \auxred {V} {(E,op_{<s,i>}))}
    {} {(E',\eta)}
}
{
  (\E,op_{<s,i>}) 
    \;\xrightarrow{V}\;
  (\E',\eff) 
}
\end{array}
\end{smathpar}
\end{minipage}


\caption{Core Operational semantics of a replicated data store.}
\label{sem:oper}
\end{figure*}


\newpage
\subsubsection{Theorems and proofs}
\begin{definition}
For a given contract ($\psi=\forall(a,b). a
\xrightarrow{Q} b \Rightarrow a \xrightarrow{vis} b $), we define a set
of effects S, to be  $\psi$-consistent under an execution $E$, if and
only if: 
$\forall (\eff \in S). \forall(a\in E.\EffSoup). Q(a,\eta)
\Rightarrow a \in S$ 
\\(We use definition of $Q^{-1}$ from the previous section to formally
define: $Q(a,b) \iff a \in Q_{E.A}^{-1}(b)  $)
\end{definition}

\begin{theorem}
For all operation executions of the form: 
$$
(\E,op_{<s,i>}) 
    \;\xrightarrow{V}\;
  (\E',\eff) 
,$$ 
if $V$ is $\psi$-consistent then $(V\cup\{\eta\})$ is also
$\psi$-consistent.
\end{theorem}
\begin{proof}
Sine $\eff$ is the only effect added to $V$, we only need to check the
following property for $\eta$, in order to prove $\psi$-consistency of
$V\cup\{\eta\}$:
$$\forall (a \in E.A). Q(a,\eta) \Rightarrow vis(a,\eta)  $$
\\
(1) $Q(a,\eta) \spc$ (hypothesis)
\\(2) $a \in Q^{-1}_V (\eta) \spc$  (we need to prove this)
\\(3) $a \in V \spc$ (premise of the rules and (2))
\\(4) $V$ is $\psi$-consistent so from (3) and definition of
consistency, the property holds for $a$
\paragraph{} In order to prove (2), we \emph{probably} have to destruct the contract and
show that if it ends with $so$, the property trivially holds since the
vis set of $\eta$ is fixed. For the other case the premise of the rule
[VIS Exec] will have the enought information to prove.
\end{proof}
\newpage

