In this section we introduce the complete behavior of our multi-consistenct shim layer as a new set of operational semantics. 
The rules are parametrized over a given map of operation names to
contracts, $\Psi :  op
\mapsto \psi $ and we assume each contract is of the form:
$\Psi(op)=\forall (a,b). a \xrightarrow{Q_{op}} \Rightarrow a \xrightarrow{vis} b.$ 
We use $Q_{op}[m]$ notation to refer to the m'th clause in the composition.
We define a pool to be a set of effects and a value that contains all
the effects that arrive to the replica. 
Our shim layer, also maintains a set of k caches (one cache for each
given contract). Each cache is maintained to be the largest
subset of pool that is closed under its
contract.  i.e. $\forall \eff \in \Cache(op). Q_{op}^{-1}(\eff)
\subseteq \Cache(op) $ 
\\We also maintain a level of availability for each effect
according to an operation and the current pool. If the availability of an effect for
$op$ is $x$, it means
that the effect satisfies the suffix of size $x$ of contract $\Psi(op)$.
An effect $\eta$ satisfies a suffix of size $x\geq 1$ of contaract
$\psi=\forall (a,b). a \xrightarrow{q_1;q_2;...;q_n} b$,
written as $ (\eta \models_{\Pool}^{x} \psi )$, if and only if
$(q_{n-(x-1)};...;q_{n} )^{-1}_{\Pool}(\{\eta\}) \subseteq \Pool$. The
initial availability value for all effects is 0, and increases when more
dependent effects arrive. 
\\
Following are the formal definitions: 
\\
\begin{minipage}{\columnwidth}
\begin{smathpar}
\stretcharraybig
\begin{array}{lclcl}
  \multicolumn{5}{c}{
    {\delta} \in \mathtt{Replicated\; Data\; Type} \spc\spc
    {v} \in \mathtt{Value}\spc\spc
    {op} \in \mathtt{Operation\; Name}
  }\\
  \multicolumn{5}{c}{
    {s} \in \mathtt{Session\; Id} \spc\spc
    {i} \in \mathtt{Effect\; Id} \spc\spc
    {\rho} \in \mathtt{Replica\; Id}
  }\\
  \eff & \in & \mathtt{Effect} & \coloneqq &  (s,i,op,v)\\
   {\Pool} & \in & \mathtt {Pool} & \coloneqq & (v,\set {\eff}) \\
   {\Cache} & \in & \mathtt {Cache} & \coloneqq & op \mapsto (v,\set{\eff})\\
   \Avail & \in & \mathtt {Avail} & \coloneqq & op \mapsto (\eff \mapsto
\{0,1,...,k-1\}) \\
  F_{op} & \in & \mathtt{Op.\,Def.} & \coloneqq & v \rightarrow \eta\\
  \EffSoup & \in & \mathtt{Eff\,Soup}	  & \coloneqq & \set{\eff} \\
  \visZ,\soZ &	\in & \mathtt{Relations} & \coloneqq & \set{(\eff,\eff)} \\
  {\E} 		& \in & \mathtt{Exec\;State}  & \coloneqq & \Exec \\
  \Theta  & \in & \mathtt{Store}      & \coloneqq & \rho \mapsto
  (\Pool, {\Cache}, \Avail) \\
  {\sigma} 	& \in & \mathtt{Session} 					 	& \coloneqq & \cdot \ALT op::\sigma \\
  \Sigma 		& \in & \mathtt{Session\;Soup}   	 	& \coloneqq &
        \langle s, i, \sigma \rangle \pll \Sigma \ALT \emptyset \\
\end{array}
\end{smathpar}
\end{minipage}
%
\begin{smathpar}
\begin{array}{c}
\ssn(s,\_,\_,\_) = s \spc\spc
\id(\_,j,\_,\_) = j \spc\spc
\oper(\_,\_,op,\_) = op \spc\spc
\rval(\_,\_,\_,n) = n\\
\end{array}
\end{smathpar}



\begin{figure*}[h]
\raggedright
%

\textbf{Auxiliary Definitions}\\
%
\begin{minipage}{\columnwidth}
\begin{smathpar}
\stretcharraybig
\begin{array}{lclcl}
  \multicolumn{5}{c}{
    {\delta} \in \mathtt{Replicated\; Data\; Type} \spc\spc
    {v} \in \mathtt{Value}\spc\spc
    {op} \in \mathtt{Operation\; Name}
  }\\
  \multicolumn{5}{c}{
    {s} \in \mathtt{Session\; Id} \spc\spc
    {i} \in \mathtt{Effect\; Id} \spc\spc
    {\rho} \in \mathtt{Replica\; Id}
  }\\
  \eff & \in & \mathtt{Effect} & \coloneqq &  (s,i,op,v)\\
   {\Pool} & \in & \mathtt {Pool} & \coloneqq & (v,\set {\eff}) \\
   {\Cache} & \in & \mathtt {Cache} & \coloneqq & (v,\set{\eff})\\
  F_{op} & \in & \mathtt{Op.\,Def.} & \coloneqq & v \rightarrow \eta\\
  \EffSoup & \in & \mathtt{Eff\,Soup}	  & \coloneqq & \set{\eff} \\
  \visZ,\soZ &	\in & \mathtt{Relations} & \coloneqq & \set{(\eff,\eff)} \\
  {\E} 		& \in & \mathtt{Exec\;State}  & \coloneqq & \Exec \\
  \Theta  & \in & \mathtt{Store}      & \coloneqq & \rho \mapsto (\Pool,\Cache) \\
  {\sigma} 	& \in & \mathtt{Session} 					 	& \coloneqq & \cdot \ALT op::\sigma \\
  \Sigma 		& \in & \mathtt{Session\;Soup}   	 	& \coloneqq &
        \langle s, i, \sigma \rangle \pll \Sigma \ALT \emptyset \\
\end{array}
\end{smathpar}
\end{minipage}
%

\begin{smathpar}
\begin{array}{c}
\ssn(s,\_,\_,\_) = s \spc\spc
\id(\_,j,\_,\_) = j \spc\spc
\oper(\_,\_,op,\_) = op \spc\spc
\rval(\_,\_,\_,n) = n\\
\end{array}
\end{smathpar}

\vspace{5mm}
\textbf{Auxiliary Reduction} \;
  \fbox{\(\auxred{v} {(\E,\langle s,i,op \rangle)} {} {(\E',\eff)}\)}\\

\begin{minipage}{\textwidth}
\rulelabel{Oper}
\begin{smathpar}
\stretcharraybig
\begin{array}{l}
\RuleTwo
{
%\Theta(\rho \mapsto (v,cache)) \qquad
F_{op}(v) = \eta \qquad
\ssn(\eta) = s \qquad 
\id(\eta) = i \qquad
%\{\eff'\} = \EffSoup_{({\sf SessID}=s,\,{\sf SeqNo}=i-1)}\\
\EffSoup' = \EffSoup \cup \{\eff\} \\
\visZ' = \visZ \cup S \times\{\eff\}\qquad
\soZ' = \soZ \cup \{(\eta',\eta) \,|\, \eta'\in \EffSoup \conj 
    \ssn(\eta')=s \conj \id(\eta')<i\}\qquad
%\soZ' = (\soZ^{-1}(\eff') \cup \eff') \times\{\eff\} \cup \soZ
}
{
  \auxred {v} {((\EffSoup,\visZ,\soZ), \langle s,i,op \rangle}
  {} {((\EffSoup',\visZ',\soZ'),\eta)}
}
\end{array}
\end{smathpar}
\end{minipage}


\vspace{5mm}
\textbf{Operational Semantics} \;
  \fbox{\((\E,\Theta,\Sigma) \;\xrightarrow{\eff}\; (\E',\Theta',\Sigma')\)}\\

\begin{minipage}{3in}
\rulelabel{Pool Refresh}
\begin{smathpar}
\stretcharraybig
\begin{array}{l}
\RuleTwo
{ 
  \eta \in \E.\EffSoup \spc
  \Theta(\rho) = (\Pool,\Cache) \spc
  \eta \not\in \Pool_e \\
  \Pool' = (apply \; \eta\; \Pool_v,\Pool_e \cup \{\eta\}) \spc \\
  \Theta' = \Theta[\rho \mapsto (\Pool',\Cache)]\\
}
{
  (\E,\Theta,\Sigma) \;\xrightarrow{\eff}\; (\E, \Theta', \Sigma)
}
\end{array}
\end{smathpar}
\end{minipage}

\vspace{5mm}
\begin{minipage}{2.8in}
\rulelabel{Cache Refresh}
\begin{smathpar}
\stretcharraybig
\begin{array}{l}
\RuleTwo
{
  \Theta(\rho) = (\Pool,\Cache) \spc \eta \in \Pool_e
  \\ \eta \not\in \Cache_e \spc
  \Cache'=(apply \; \eta \; \Cache_v ,\Cache_e \cup \{\eta\}) \\ \psi^{-1}
  (\eta) \subseteq \Cache_e
 \spc  \Theta' = \Theta[\rho \mapsto (\Pool,\Cache')]\\
}
{
  (\E,\Theta,\Sigma) \;\xrightarrow{\eff}\; (\E, \Theta', \Sigma)
}
\end{array}
\end{smathpar}
\end{minipage}
\hspace{12 mm}
\vspace{3mm}
\begin{minipage}{2.3in}
\rulelabel{Exec}
\begin{smathpar}
\stretcharraybig
\begin{array}{l}
\RuleTwo
{
  \auxred{\CacheFinder_{\psi}(\Theta(\rho))} {(\E,\langle s,i,op \rangle)} {} {(\E',\eta)}  \\
  \Theta(\rho) = (\Pool,\Cache) \spc \psi^{-1}(\eta) \subseteq \Pool_e  \spc
}
{
  (\E,\Theta,\langle s,i,op::\sigma \rangle \pll \Sigma) 
    \;\xrightarrow{\eff}\;
  (\E',\Theta,\langle s,i+1,\sigma \rangle \pll \Sigma) 
}
\end{array}
\end{smathpar}
\end{minipage}


\caption{Operational semantics of a replicated data store.}
\label{sem:oper}
\end{figure*}




