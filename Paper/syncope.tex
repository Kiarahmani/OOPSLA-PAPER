%% For double-blind review submission
\documentclass[acmlarge,anonymous]{acmart}\settopmatter{printfolios=true} % review will add line numbers -- anonymous will make it anonymous
%% For single-blind review submission
%\documentclass[acmlarge,review]{acmart}\settopmatter{printfolios=true}
%% For final camera-ready submission
%\documentclass[acmlarge]{acmart}\settopmatter{}
\usepackage{wrapfig}
\usepackage[inline]{enumitem}
\usepackage{bussproofs}
\usepackage{bussproofs}
\usepackage{dsfont}
\usepackage{xspace}
\usepackage{mathtools}
\usepackage{amsthm}
\usepackage{mathpartir}
\usepackage{listings}
\usepackage{color}

\usepackage[T1]{fontenc}
\usepackage[utf8]{inputenc}
\definecolor{pblue}{rgb}{0.13,0.13,1}
\definecolor{pgreen}{rgb}{0,0.5,0}
\definecolor{pred}{rgb}{0.9,0,0}
\definecolor{pgrey}{rgb}{0.46,0.45,0.48}
\definecolor{mygray}{rgb}{0.5,0.5,0.5}

\usepackage{listings}
\lstset{language=Java,
  showspaces=false,
  showtabs=false,
  breaklines=true,
  showstringspaces=false,
  breakatwhitespace=false,
  commentstyle=\color{pgreen},
  keywordstyle=\color{pblue},
  stringstyle=\color{pred},
  basicstyle=\footnotesize\tt,
  tabsize=2,
  numbers=left,
  numbersep=0pt,
  numberstyle=\tiny\color{mygray},
  moredelim=[il][\textcolor{pgrey}]{$$},
  moredelim=[is][\textcolor{pgrey}]{\%\%}{\%\%}
}



%% Note: Authors migrating a paper from PACMPL format to traditional
%% SIGPLAN proceedings format should change 'acmlarge' to
%% 'sigplan,10pt'.

\usepackage{lipsum}  
%% Some recommended packages.
\usepackage{booktabs}   %% For formal tables:
                        %% http://ctan.org/pkg/booktabs
\usepackage{subcaption} %% For complex figures with subfigures/subcaptions
                        %% http://ctan.org/pkg/subcaption


\makeatletter\if@ACM@journal\makeatother
%% Journal information (used by PACMPL format)
%% Supplied to authors by publisher for camera-ready submission
\acmJournal{PACMPL}
\acmVolume{1}
\acmNumber{1}
\acmArticle{1}
\acmYear{2017}
\acmMonth{1}
\acmDOI{10.1145/nnnnnnn.nnnnnnn}
\startPage{1}
\else\makeatother
%% Conference information (used by SIGPLAN proceedings format)
%% Supplied to authors by publisher for camera-ready submission
\acmConference[OOPSLA'17]{The ACM SIGPLAN conference on Systems, Programming, Languages and Applications}{January 01--03, 2017}{New York, NY, USA}
\acmYear{2017}
\acmISBN{978-x-xxxx-xxxx-x/YY/MM}
\acmDOI{10.1145/nnnnnnn.nnnnnnn}
\startPage{1}
\fi


%% Copyright information
%% Supplied to authors (based on authors' rights management selection;
%% see authors.acm.org) by publisher for camera-ready submission
\setcopyright{none}             %% For review submission
%\setcopyright{acmcopyright}
%\setcopyright{acmlicensed}
%\setcopyright{rightsretained}
%\copyrightyear{2017}           %% If different from \acmYear


%% Bibliography style
\bibliographystyle{ACM-Reference-Format}
%% Citation style
%% Note: author/year citations are required for papers published as an
%% issue of PACMPL.
\citestyle{acmauthoryear}   %% For author/year citations


%%% New Definitions %%%%%%%
%% -----------------------
\newcommand{\tool}[0]{{\sc syncope}\xspace} % short for SYNthesizing
% COnsistency Property Enforcement.


\begin{document}

%% Title information
\title{\tool: Automatic Enforcement of Distributed Consistency Guarantees}         %% [Short Title] is optional;
                                        %% when present, will be used in
                                        %% header instead of Full Title.
%\titlenote{with title note}             %% \titlenote is optional;
                                        %% can be repeated if necessary;
                                        %% contents suppressed with 'anonymous'
%\subtitlenote{with subtitle note}       %% \subtitlenote is optional;
                                        %% can be repeated if necessary;
                                        %% contents suppressed with 'anonymous'


%% Author information
%% Contents and number of authors suppressed with 'anonymous'.
%% Each author should be introduced by \author, followed by
%% \authornote (optional), \orcid (optional), \affiliation, and
%% \email.
%% An author may have multiple affiliations and/or emails; repeat the
%% appropriate command.
%% Many elements are not rendered, but should be provided for metadata
%% extraction tools.

% FIRST AUTHOR
%% Author with single affiliation.
\author{Kia Rahmani}
%\authornote{with author1 note}          %% \authornote is optional;
                                        %% can be repeated if necessary
\orcid{nnnn-nnnn-nnnn-nnnn}             %% \orcid is optional
\affiliation{
  \position{PhD Student}
  \department{Computer Science}              %% \department is recommended
  \institution{Purdue University}            %% \institution is required
  \streetaddress{Street1 Address1}
  \city{City1}
  \state{State1}
  \postcode{Post-Code1}
  \country{Country1}
}
\email{rahmank@purdue.edu}          %% \email is recommended
%------------------------------------------------------------------------------------
%SECOND AUTHOR
\author{Gowtham Kaki}
%\authornote{with author1 note}          %% \authornote is optional;
                                        %% can be repeated if necessary
\orcid{nnnn-nnnn-nnnn-nnnn}             %% \orcid is optional
\affiliation{
  \position{PhD Student}
  \department{Computer Science}              %% \department is recommended
  \institution{Purdue University}            %% \institution is required
  \streetaddress{Street1 Address1}
  \city{City1}
  \state{State1}
  \postcode{Post-Code1}
  \country{Country1}
}
\email{rahmank@purdue.edu}       


%------------------------------------------------------------------------------------
%THIRD AUTHOR
\author{Suresh Jagannathan}
%\authornote{with author1 note}          %% \authornote is optional;
                                        %% can be repeated if necessary
\orcid{nnnn-nnnn-nnnn-nnnn}             %% \orcid is optional
\affiliation{
  \position{PhD Student}
  \department{Computer Science}              %% \department is recommended
  \institution{Purdue University}            %% \institution is required
  \streetaddress{Street1 Address1}
  \city{City1}
  \state{State1}
  \postcode{Post-Code1}
  \country{Country1}
}
\email{rahmank@purdue.edu}   
%------------------------------------------------------------------------------------


%% Paper note
%% The \thanks command may be used to create a "paper note" ---
%% similar to a title note or an author note, but not explicitly
%% associated with a particular element.  It will appear immediately
%% above the permission/copyright statement.
%\thanks{with paper note}                %% \thanks is optional
                                        %% can be repeated if necesary
                                        %% contents suppressed with 'anonymous'


%% Abstract
%% Note: \begin{abstract}...\end{abstract} environment must come
%% before \maketitle command

\begin{abstract}

  Modern-day distributed systems often use replication to improve
  communication latency and fault tolerance among geographically
  distributed servers and clients.  But, replication compromises
  simplicity because it forces applications to weigh the correctness
  benefits of guaranteeing strong consistency against a potentially
  significant cost in performance.  Because not all operations on a
  replicated data type require strongly consistent behavior, however,
  library designers ought be able to direct implementations to provide
  precisely the level of consistency needed, eschewing the costs of
  strong consistency except when absolutely necessary.  Unfortunately,
  implementations typically support only a small set of consistency
  levels.  Consequently, developers must either be content with
  mapping application requirements to a supported consistency level
  that may well enforce stronger guarantees than required, or hand
  weave a specialized protocol that more accurately matches
  application needs.  The former approach sacrifices performance,
  while the latter introduces unwanted complexity.

  In this paper, we describe a lightweight runtime system for weakly
  consistent distributed systems that dynamically customizes a
  consistency protocol based on a declarative axiomatic specification
  that reflects the necessary constraints any correct implementation
  must satisfy.  Our technique generates a provably optimal runtime
  enforcement mechanism that imposes no additional communication or
  blocking overhead beyond what is required to satisfy the
  specification, thus freeing developers from writing bespoke and
  complex consistency protocols without having to sacrifice
  availability and performance by not doing so.

  We demonstrate the applicability of our approach by automatically
  deriving enforcement mechanisms for various well-known weak
  consistency instantiations.  Experimental results show that the
  performance of our automically derived mechanisms is comparable to
  specialized hand-written protocols, providing strong evidence of its
  practical utility.
\end{abstract}


%% 2012 ACM Computing Classification System (CSS) concepts
%% Generate at 'http://dl.acm.org/ccs/ccs.cfm'.
%\begin{CCSXML}
%<ccs2012>
%<concept>
%<concept_id>10011007.10011006.10011008</concept_id>
%<concept_desc>Software and its engineering~General programming languages</concept_desc>
%<concept_significance>500</concept_significance>
%</concept>
%<concept>
%<concept_id>10003456.10003457.10003521.10003525</concept_id>
%<concept_desc>Social and professional topics~History of programming languages</concept_desc>
%<concept_significance>300</concept_significance>
%</concept>
%</ccs2012>
%\end{CCSXML}

%\ccsdesc[500]{Software and its engineering~General programming languages}
%\ccsdesc[300]{Social and professional topics~History of programming languages}
%% End of generated code


%% Keywords
%% comma separated list
%\keywords{Fine Grained Consistency , keyword2, keyword3}  %% \keywords is optional


%% \maketitle
%% Note: \maketitle command must come after title commands, author
%% commands, abstract environment, Computing Classification System
%% environment and commands, and keywords command.
\maketitle

%================================ SECTION ONE: Introduction
\section{Introduction}
\label{sec:intro}

Modern web-based applications are typically implemented as multiple
agents simultaneously serving clients, operating over shared data
objects replicated across geographically distributed machines.
Historically, replication transparency (i.e. requiring distributed
systems to \emph{appear} as a single compute and storage server to
users), has been the \emph{de facto} system abstraction used to
program in these environments.  This abstraction has resulted in the
development of standardized implementation and reasoning techniques
around \emph{strongly consistent} (SC) distributed stores.  Although
strong notions of consistency, such as \emph{linearizability} and
\emph{serializability}, are ideal to develop and reasoning about
distributed applications, they come at the price of availability and
low-latency.  Extensive synchronization overhead often necessary to
realize strong consistency is unacceptable for web-scale applications
that wish to be ``always-on'' despite network partitions.  Such
applications are therefore usually designed to tolerate certain
inconsistencies, allowing them to adopt weaker notions of consistency
that impose less synchronization overhead. An extreme example is
\emph{eventual consistency} (EC), where the application responds to
user requests using just the local state of the server to which the
client connects; this state is \emph{some} subset of the global state
(i.e., it includes an unspecified subset of writes submitted to the
application in an unspecified order).  Applications that may not
tolerate the level of inconsistency imposed by EC strengthen it as
needed, resulting in various instantiations that are stronger than EC,
but weaker than SC. The term \emph{weak consistency} is a catch-all
term used to refer to such application-specific weak consistency
guarantees.

Unfortunately, the \emph{ad hoc} nature of weak consistency confounds
standardization, with different implementations defining different
mechanisms for achieving weakly consistent behavior.  Oftentimes,
implementations are closely tied to application logic, complicating
maintainability and reuse.  To illustrate, consider a web application
that stores user passwords (encrypted or otherwise) in an
off-the-shelf EC data store (e.g., Cassandra~\cite{cassandra}). The
application allows an authenticated user to change her password,
following which the current authentication expires, and the user is
required to re-login.  Now, consider the scenario shown in
Fig.~\ref{fig:rmw-anomaly} where Alice changes her password, and
subsequently tries to login with the new password. This involves a
write of a new password to the store, followed by a read during
authentication.  However, because of transient system properties
(e.g., load balancing, or network partitions), Alice's write and the
read could be served by different replicas of the store, say $R_1$ and
$R_2$ (resp.), where $R_2$ may not (yet) contain the latest writes
from $R_1$. Consequently, Alice login attempt fails, even though she
types the correct password.

To preempt the scenario described above, applications might want to
enforce a stronger consistency guarantee that ensures reads from a
client session witness previous writes from the same session. The
consistency guarantee, known as \emph{Read-My-Writes/Read-Your-Writes}
(RMW/RYW), is one of several well-understood session
guarantees~\cite{terry-pdis94}, yet the methods used for its
enforcement are often store -and application- dependent. For instance,
Oracle's replicated implementation of Berkeley DB suggests application
developers implement RMW by querying various metadata associated with
writes~\cite{oracle-ryw}.  Each successful write to the store returns
a commit token, which is then passed with the subsequent reads to help
the store identify the last write preceding the read. The read
succeeds only if the write is present at the replica serving the read,
failing which the application has to retry the read, preferably after
some delay.  Fig.~\ref{fig:rmw-oracle} illustrates this idea.

The RMW implementation described above already requires considerable
re-engineering of the application (to store and pass commit tokens for
each object accessed), and conflates application logic with concerns
orthogonal to its semantics. On stores that do not admit metadata
queries (e.g., Cassandra), the implementation is even more complicated
as we describe in Sec.~\ref{sec:motivation}. Moreover, applications
sometimes require different consistency guarantees for different
objects.  In such cases different enforcement mechanisms must be
developed, forcing developers to simultaneously reason about their
respective properties \emph{in conjunction with} the application
state. This is clearly an onerous task.  Other alternatives such as
forgoing application integrity, or resorting to strong consistency,
sacrifice correctness or availability, both unappealing options.

In this paper, we propose an alternative to the aforementioned
approaches that overcomes their weaknesses.  \tool is a lightweight
runtime system for Haskell that allows application developers to take
advantage of weak consistency without having to re-engineer their code
to accommodate consistency enforcement logic.  The key insight that
drives \tool's design is that the hardness of reasoning about the
integrity of a distributed application stems from conflating
application logic with the consistency enforcement logic, reasoning
about both \emph{operationally}.  By separating application semantics
from consistency enforcement semantics, admitting operational
reasoning for the former, and declarative reasoning for the latter
frees programmers from having to worry about implementation details of
consistency guarantees, and instead focus on reasoning about
application semantics under the assumption that specified consistency
guarantees are automatically enforced by the data store runtime.  Our
approach admits declarative reasoning for consistency enforcement via
a specification language that lets programmers formally specify the
consistency requirements of their application. The design of our
specification language is based on the observation that various forms
of weak consistency guarantees differ only in terms of how and what
they mark as \emph{dependencies} among operations.  When all
dependencies are present on a replica to an operation, then the effect
of the operation is guaranteed to be correct with respect to the
specification.  For example, RMW marks all previous writes from the
same session as dependencies of subsequent read operations, so an RMW
read succeeds only if all the previous writes are visible (i.e.,
present on the replica on which the read is performed) . A different
consistency guarantee (e.g., \emph{Monotonic Reads} (MR)) imposes a
different set of dependencies, as do various combinations of (e.g.,
MR+RMW).  By allowing a runtime system to monitor dependencies defined
by consistency specifications, we realize a generic weak consistency
enforcement mechanism framework.  Such a runtime, working in tandem
with a consistency specification language, contributes to the novelty
of our approach.


% here users send requests to a cloud of servers
% to post a message or
% to read the current messages on the board. Each user request is sent
% to an available server, which itself is working on top of an instance
% of an off-the-shelf data store (e.g. Facebook's Cassandra).  Since the
% underlying stores, usually satisfy the {\bf eventual consistency}
% model, developers are guaranteed  that every write to a local instance
% of the data store will eventually  be delivered at all other
% instances. However, most of the desirable application-level properties
% are not met under this model.          For example, assume the
% developers wants to make sure that all the $\mathtt{read}$ requests
% from a user, would necessarily include prior writes by the same user.
% This guarantee which is known as {\bf read my writes (RMW)} can be
% violated in eventually consistent stores.  
% \begin{figure}
\centering
\begin{subfigure}[b]{0.3\textwidth}
\includegraphics[scale=0.2]{../Figures/System_example.pdf}
\label{fig:rmw_falsified}
\caption{A behavior allowd in EC stores that falsifies RMW guarantee}
\end{subfigure}\quad
\begin{subfigure}[b]{0.3\textwidth}
A figure showing tokens being passed around for enforcing RMW
\label{fig:addhoc_impl}
\caption{An ad-hoc implementation of RMW consistency guarantee, using
tokens and guards}
\end{subfigure}\quad
\begin{subfigure}[b]{0.3\textwidth}
A figure showing how consistency management can be separated from the
applications and underlying stores,using shim layers (our shims are
synthesized according to the given contract)
\label{fig:shim_impl}
\caption{Consistency management shim layer that guarantees desired
levels of consistency, based on the given specifications}
\end{subfigure}~
\caption{Running applications on EC store, ad-hoc consistency
implementations, general-purpose shim layers}



\end{figure}


% Here, developers are forced to come up with their own implementation of
% RMW. They must modify client and server applications to generate, send
% and receive special tokens, to maintain the set of updates available at
% each server, and block user requests, if some necessary updates are
% still missing. Now, in order to have a correct application, the
% developer must prove certain non-trivial safety and liveness
% properties for this implementation. To make the matter worse, the
% proof will becmoe obsolete after each modification in the application or
% the consistency requirements. For example, after users reporting some unaccepatble
% behaviors, in order to disallow those executions, developers 
% implement and add another consistency guarantee to RMW; a task that
% would clearly make the original correctness proofs obsolete. 


% In this paper, we address these issues by introducing a compositional, principled approach to 
% derive enforcement mechanisms for various forms of weak consistency 
% guarantees. 
% We offer developers with a tool that generates a shim layer that works
% with an eventually consistent store and extends it to a key-value store
% with multiple environment, each of which shows a certain level of
% consistency, specified by the developers. 
% We also introduce a language for specifying application-level consistency 
% requirements as logical formulae that express legal execution states of the distributed store. 
% Our consistency enforcement methodology is independent of the underlying store 
% and only assumes the well-established guarantee of "eventual delivery". 
% Moreover, our enforcement tool is complete for the specification
% language, which itself is powerful enough to express all the known consistency 
% guarantees in the literature.

% We argue that all the consistency guarantees basically specify, \emph{when} an 
% application instance should block a user request and for arrival of 
% \emph{what} remote updates it should wait. For obvious reasons, any implementation of these 
% consistency guarantees must be \emph{correct} and \emph{optimal}. The former property states that 
% all possible behaviors of the implementation are allowed by the given specification and the later 
% ensures that application instances do not engage in 
% any unnecessary synchronization before responding to a user request (i.e. users are only blocked if 
% it is absolutely necessary).
% \\ Our technique is based on tracking relationships between update effects, and maintaining multiple 
% (logical) caches at the shim layer, each of which is enforced to satisfy a specific level of consistency, 
% derived from the given specifications. We believe that our approach is the first ever principled 
% reasoning and implementation framework for weak consistency enforcement techniques which 
% is also proven to be \emph{correct} and \emph{optimal}. 
% By separating all the consistency management procedures from the application level, we are taking 
% the non-trivial task of proving soundness criteria for ad-hoc consistency enforcement techniques 
% off of the developer's shoulders. 
% Contributions of the paper
A summary of our contributions is given below:
\begin{itemize}
  \item We propose a consistency specification language that lets
    programmers express the consistency requirements of their
    applications in terms of the dependencies between operations.

  \item We describe a generic consistency enforcement runtime that
    analyzes an operation's consistency specification, and ensures
    that its dependencies are satisfied before it is executed. We
    formalize the operational semantics of the runtime, and prove its
    correctness and optimality guarantees. Optimality includes
    \emph{minimum wait}, which guarantees that an operation waits (on
    arriving communication) only until its dependencies are satisfied,
    and \emph{minimum staleness}, which guarantees that among various
    states that satisfy an operation's dependencies, operation
    witnesses the latest state.
  
  \item We describe an implementation of our specification language and
    consistency enforcement runtime in a tool called \tool, which
    works on top of an off-the-shelf EC data store. We evaluate \tool
    over realistic applications and microbenchmarks, and present
    results that demonstrate the performance benefits of making
    fine-grained distinctions between consistency guarantees, and the
    ease of doing so via our specification language.

\end{itemize}

The remainder of the paper is organized as follows.  The next section
presents a system model that describes key notions of consistency and
replication.  Sec.~\ref{sec:motivation} provides additional
motivation.  Sec.~\ref{sec:contract-lang} introduces an abstract store
model that forms the basis of our specification language, along with
the language itself. We describe the semantis of our consistency
enforcement runtime, and formalize its correctness and optimality
guarantees in Sec.~\ref{sec:formalization}.  Sec.~\ref{sec:algorithm}
elaborates on the algorithmic aspects of our runtime that is key to
its efficient realization.  Sec.~\ref{sec:evaluation} describes \tool,
an implementation of the specification language and enforcement
mechanism, and evaluates its applicability and practical utility.
Related work and conclusions are presented in
Sec.~\ref{sec:related-work}.


%% . While there are a few well-defined
%% models of weak consistency~\cite{terry-pdis94}, the list is not
%% exhaustive as applications often define consistency models to suit
%% their needs. Furthermore, unlike the standardized implementation
%% techniques, such as 2PL~\cite{2pl}, to enforce strong consistency,
%% enforcement of weak consistency guarantees, including the well-defined
%% ones, is often done via \emph{ad hoc} implementation techniques that
%% are strongly coupled with the the application logic. 
% weak consitency is often enforced using \emph{ad hoc} implementation
% techniques that are tailor-made Even though weak models of
% consistency (e.g. session guarantees from Terry et. al.) have been
% known for more than two decades, they suffer from the lack of
% standardized definitions and enforcement methodologies, which forces
% developers to modify their applications with ad-hoc fixes in order
% to enforce their desired levels of consistency. The lack of a
% general reasoning framework for weak consistency models, has
% resulted in the difficulty of proving correctness and optimality
% properties of such highly error-prone implementations. To make the
% matter worse, many of distributed applications require different
% \emph{combinations} of known consistency guarantees, or might even
% face new consistency requirements after the developement phase is
% over.


%================================ SECTION TWO: System Model
%==========================================================
%--- What programs are and how they are written in our tool
%==========================================================
\section{System Model}
\label{sec:sys_model}

A data store in our system model is a collection of \emph{replicas}
(\replO{1},\replO{2},...), each of which maintains a copy of a set of
replicated \emph{data object} (\xO,\yO,...).  Each data object includes
and maintains a \emph{state value} (\vO,\vO',...) and is equipped with a set of \emph{operations}
(\opO,\opO',...). Operations may read the state of an object
residing in a replica, and modify it by generating \emph{update
effects} ($\eff$,$\eff'$,...).
These effects are then asynchronously sent to all other replicas where
they are applied to the state of the object instance at the recipient
replica using a user-supplied function.
Figs.~\ref{fig:sys_model1} and \ref{fig:sys_model2} illustrate this
process.

Because there is no direct synchronization between replicas when an
operation is executed, concurrent and possibly conflicting updates can
be generated at different replicas.  Conflict resolution is handled at
the point when an effect is applied to the current state of the
object, and must be designed to ensure that all replicas eventually
converge to the same value, assuming generated effects quiesce.  This
model admits all inconsistencies and anomalies associated with
eventual consistency \cite{quelea,terry}.

Clients in our model interact with the store by invoking operations on
objects.  A \emph{session} is a sequence of operations invoked by a
particular client. Consequently, operations (and effects) can be
uniquely identified by the \emph{session id} that invoked them, and
their \emph{sequence number} in that particular session, which is used
by replicas to record the set of all updates that are locally applied.
Since, the data store may be concurrently accessed by a large number
of clients, operations (even from the same session) might be routed to
different replicas to improve latency (see Figs.~\ref{fig:sys_model1}
and \ref{fig:sys_model3}).

Lastly, we define two relations over effects created in the store.
\emph{Session order} ($\soZ$) is an irreflexive, transitive relation
that relates an effect to all subsequent effects from the same
session.  \emph{Visibility} (${\visZ}$) is an
irreflexive and assymetric relation that relates an effect to all
others that are influenced by it (i.e., witnesses its update) at the time of
their generation.  For example, in Fig.~\ref{fig:sys_model3} $\visO
(\eta,\eta')$ holds, since $\eta$ (the effect of $\opO$) has already
been delivered and applied to the replica~\replO{4}, when $\opO'$ is
executed and thus has influenced generation of $\eff'$.
\begin{figure}[t]
    \centering
    \begin{subfigure}[t]{0.3\textwidth}
    \centering
        \includegraphics[scale=0.32]{Figures/system_model1.pdf}
        \caption{\scriptsize A client submits an operation $\op$ to the store, which
	is routed to the replica \texttt{\#1}.}
        \label{fig:sys_model1}
    \end{subfigure}
    \hfill
    \vline
    \hfill
    \begin{subfigure}[t]{0.3\textwidth}
        \centering
	\includegraphics[scale=0.32]{Figures/system_model2.pdf}
        \caption{\scriptsize The state of the replica \texttt{\#1} is updated, an
	effect is created and is being propagated}
        \label{fig:sys_model2}
    \end{subfigure}
    \hfill
    \vline
    \hfill
    \begin{subfigure}[t]{0.3\textwidth}
        \centering
	\includegraphics[scale=0.32]{Figures/system_model3.pdf}
        \caption{\scriptsize Second operation $\op'$ is submitted to the store, which
	is routed to the replica \texttt{\#4}}
        \label{fig:sys_model3}
    \end{subfigure}
    \caption{system model of \tool}\label{fig:system_model}
\end{figure}




















%==========================================================




























































\begin{comment}

%
% Initial Paragraph explaining the running example
%

Let's consider the developement of a modern-day large scale \emph{Bulletin
Board} web application, that allows managers to initiate multiple boards for their
organization, where employees can either write something on the
board, or request to read all the messages on it.
In order to achieve high availablity, developers decide to implement the
application as distributed servers, working
with replicated data objects on top of an 
Eventually Consistent Data Store ({ECDS}). The data model here dictates
each row at the data base level, to include a board identifier
and a set of messages 
(Fig. \ref{fig:simple_bb}).
In this fashion, employees can start a session to send requests to geographically distributed
servers and make updates to the rows, and the underlying store,
would guarantee the eventual delivery of all updates at every server.
\begin{figure}[h]
    \begin{subfigure}[b]{0.48\textwidth}
        \centering
	\includegraphics[scale = 0.5]{Figures/SimpleBulletinApp.pdf}
        \caption{Simple Data Model}
    \end{subfigure} 
    \begin{subfigure}[b]{0.48\textwidth}
        \centering
	\includegraphics[scale=0.5]{Figures/ModifiedBulletinBoard.pdf}
	\caption{Modified Data Model}
    \end{subfigure}
    \\ \hrulefill \\
\caption{Low-Level Data Model of a Simple Bulletin Board Application
(Left) and the Modified Version Including Session Counters to be Used
for Providing }

\end{figure}

%
%===========================================================================SUBSECTION
% Paragraph 2: Explaining the possible anomaly under EC
%
\subsection{Ad-hoc Anomaly Prevention Mechanisms}

Developers cannot solely rely on the eventual consistency guarantee that
is provided by the underlying data stores for application correctness. There are application integrity anomalies
that must be thought of at the developement stage and be handled. For
example, assume an employee signs into the system, writes some
messages on the board, and immediately refreshes her browser hoping to see
her messages on the board, which are however not there. This is obviously
not desirable and as mentioned in
the previous section,  can occure if
the write messages were sent to a server, and the subsequent read to
another, where her previous write updates are  not available yet. 
\begin{wrapfigure}{l}{0.3\textwidth}
	\centering
	\includegraphics[scale = 0.4]{Figures/metadatapresent.pdf}
\vspace{1 mm} 
\hrule
\caption{Incorrect implementation of RMW. The meta data referring to the
first operation of session A can be present at server \#2, when the
actuall message is not}
\end{wrapfigure}

%
% Paragraph 3: Ad-hoc prevention methods
%

To prevent this anomaly, developers must provide RMW consistency
guarantee for their application, where user requests sent to a server are guaranteed to see
all the previous updates made by the same client and session. The
conventional technique for acheiving this,
is to tag each session and its contained requests, respectively with unique identifiers and
sequence numbers and to include this meta-data in the database, to
record what requests have been seen by the server at any time. 
Moreover, there must be some mechanisms to temporarily block user requests, when the database
does not include the updates from previous requests.
Since developers are not interested in synchronized and direct
communication between servers (which obviously contradics the idea of using
an ECDS), they have to record the meta data in the key value store and rely on
the provided delivery mechanisms. 
%
%Paragraph 4: The problem with simply adding the metadata on top of the
%implementation
%

However, the desired behavior cannot
be achieved by simply creating meta data "add-ons" to the current implementation.
The reason is that, in EC stores, no order of delivery is guaranteed,
and some servers might receive the meta-data regarding some updates that
are not present at the server yet. 
Developers need to make sure that the meta-data recording sequence
number of a request, becomes available at a server, at the same time
when the corresponding update arrives. This can only be done by
modifying the low-level initial data models, to also record the sequence
numbers seen from each client. 
%
% Paragraph 5: The difficulties of low-level modifications: complex
% and redundant
%

Now developers are inevitebaly required to modify the origianl low level
data model to also include the required meta data, so the changes in the
data and meta-data would arrive at servers atomically. Such changes on
the data model will require rewriting 90\% of application functions from the scratch
which is extremely inconvinience. 
Moreover, now developers must face non-trivial problems associated with
session management and dynamic data models. For example, when a new
client arrives at a server 
and initiates a session, the server is required to generate a unique ID
for that session, and then add a new column to the
underlying database table, to record the associated meta-data.
Unfortunately, table alteration in ECDSs is not an atomic task, and forces developers to implement
appropriate guards to make sure that the new column is available at all servers, before
allowing the client to move on with the rest of her requests.  
%
% Paragraph 6: The problem of new consistency requirements
%

To make the matter worse, new integrity anomalies are discovered very
frequently. Assume the bulltin board application with RMW has been
successfully designed and implemented, however, after the developement
phase users report a new type of anomaly, where messages seem to be
disappeared after refreshing the browser. This undesirable behavior is
allowed under EC and even RMW; since two read operations can go to
different servers, where the second one does not include all the
messages read from the first server. This requires another consistency
level, called Monotonic Reads, that guarantees subsequent reads will
include everything that has been read before in a session.
 
 
 






\end{comment}




%================================ SECTION THREE: Motivation
\newpage
\section{Motivation}
\label {sec:motiv}
In this section, we explain the developement process of a highly
available application in \tool. We discuss a possible  anomily under
eventual consistency and then explain the difficulties associated 
with the manual approaches for prevention techniques.
Lastly we will explain how by automating the process, \tool can liberate
the developers from all those problems.
Our ideas mentioned here, are completed in  section
\ref{sec:ctrt_language}, where we extend the tool with a language, to
specify \emph{any} kind of anomalies. 


%
%--- What is the application, what are the requirements 
%
\subsection{RDTs in Eventually Consistent Stores }
\begin{figure}[t]
        \centering
	\begin{subfigure}[b]{0.45\textwidth}
	\begin{lstlisting}
	type Effect  String 
	type State = String 

	read :: State -> (String, Maybe Effect)
	read s = (s,Nothing)

	write :: String -> ((), Maybe Effect)
	write comment = ((), comment)

	apply :: State -> Effect -> State 
	apply s eff = in s ++ " - " ++ comment
	\end{lstlisting}
	\caption{Implementation}
	\label{subfig:comment_code}
	\end{subfigure}
	\hfill
	\begin{subfigure}[b]{0.49\textwidth}
	\includegraphics[scale=0.36]{Figures/comment_application.pdf}
	\caption{Example}
	\label{subfig:comment_example}
	\end{subfigure} 
\\ \hrulefill
\caption{A distributed application for comment section
management}
\label{fig:comment_app}
\end{figure}



Let's now consider a highly available comment
section management application, as part a photo sharing web site.
Figure \ref{subfig:comment_code} presnets implementation of such an application
in our system model. As explained in section \ref{sec:sys_model}, the
code is consisted of types \effectC{} and \stateC{}. Both types here are
defined as a strings, the former representing the text of a comment, and
the latter all the visible comments concatinated together.
Everytime a user calls the \writeC{} function to add a comment, an
\effectC{}
is generated and a \readC{} call simply
returns the \stateC{} of the object.

The \applyC{} function is given an effect, and is defined by the
developers to \emph{update} the objects' state.
Here the \applyC{} function simply pastes the 
included comment inside of  an effect, to the end of the current \stateC{}. As we mentioned earlier, we
completely separate the convergence semantics of the application  from the consistency
requirements. Since our focus is consistency here, we omit any conflict
resolution strategy in the code, however, developers (using roll-backs,
etc) can design the \applyC{} function to resolve conflicting
concurrent updates as they desire. 

Figure \ref{subfig:comment_example}, presents an example of how users
interact with the application. The example shows two clients, Bob and
Alice, that invoke operations on a comment section object. In the
setting Bob first writes a comment, which is routed to the replica A,
whose effect is then propagated and delivered to the replica B, where Alice's
first read operation is routed to next. Alice and Bob then keep talking
through more read and write events, whose order are marked in the
figure. 

Now let's assume Bob's read opration, instead of replica A, was routed
to another replica C, where the update from his first operation was not
present. This is \emph{lost-updates} anomaly, a very well-known
undesired behavior that is admitted in eventually consistent stores. 
Now developers are faced with the problem of preventing such an undesired
behavior, a task that as we will explain shortly, is difficult,
erorneous and heavily tangled with the application logic.
%
%--- What are the challenges implementing those reuquirements manually
%
\subsection{Ad-hoc Anomaly Prevention}
In this part, by referring to the modified code  presented in figure
\ref{fig:modified_code}, we will
explain a well-understood approach toward eliminating the lost\_update anomaly in
our comment manager application running on an EC store.

%tagging effects
First modification required in this technique is tagging effects with
unique identifiers, consisting of their originating sessions' id, and their
sequence number in them. This is used by replicas to
record the set of effects, that are alerady present locally
($line:1,2,5$). By this simple adjustment, the undesired anomaly would
be completely avoided, if operations would never be routed to replicas,
that do not contain all the effects from prior operations on that
session (let's call these effects the dependencies of
the operation). 

%blocking
Since operations do not have any control on which replica they are
routed to, the above property can be achieved, if operations that are
routed to a replica that does not contain their dependencies, wait
before execution until such effects become available at that replica.
This technique that is called {\bf blocking}, guarantees that the state
witnessed by oerations, 
is updated by a set of effects that is a \emph{superset} of the desired
dependencies.

Moreover, another technique called {\bf filteration} is used to further realize
the above idea. It basically separates the set off effects that have
arrived to the replica (available effects), and the effects who have
arrived and also been applied to the state (filtered effects).
By this separation, replicas can only apply effects to their state, if all
effects in session order with them, have already been applied to the
state. 
This way, replicas can record only the highest sequence
numbers from each session that they have applied to the state (since it
is guaranteed that the smaller ones are also applied)($line:3,6$). 
Figure \ref{fig:modified_code}, represents the blocking technique in the modified \readC{}
operation, where the result is only returned if the required
dependencies have already been applied to the state. Furthermore, the
filteration technique is used bu tge modified \applyC{} funciton, only
updates the state, if the sequence number of the given effect is
larger than the highest previously applied effect to the state precisly
by 1.
\begin{figure}[t]
	\centering
	\begin{subfigure}[t]{0.5\textwidth}
	\begin{lstlisting}
data Sess = Bob | Alice
type ID = (Sess,Int) 
type Effect= (ID,String)
type State = (String,Int,Int)
	
read :: ID -> State -> String
read (sess,seq) (st,sq1,sq2) = 
	case sess (*@\textcolor{blue}{of}@*) 
		Bob ->   if (seq==sq1+1) (*@\textcolor{blue}{then}@*) st
		         else read (sess,seq)(st,sq1,sq2)
		Alice -> if (seq==sq2+1) (*@\textcolor{blue}{then}@*) st
		         else read (sess,seq)(st,sq1,sq2)
	\end{lstlisting}		  
	\end{subfigure}
	%
	\hfill
        %
	\begin{subfigure}[t]{0.42\textwidth}
	\begin{lstlisting}[firstnumber=13]
	apply :: State -> Effect -> State 
	apply (st,sq1,sq2) ((sess,seq),cm) = 
	  case sess (*@\textcolor{blue}{of}@*) 
	    Bob ->   if (sq1==seq-1)
	             (*@\textcolor{blue}{then}@*) (st++cm,sq1+1,sq2)
	             else (st,sq1,sq2)
	    Alice -> if (sq2==seq-1)
	             (*@\textcolor{blue}{then}@*) (st++cm,sq1,sq2+1)
	             else (st,sq1,sq2)
	\end{lstlisting}		  
        \end{subfigure}

	\hrulefill
	\caption{Guarded Application to Prevent Lost-updates Anomaly
	When Serving Bob and Alice}
	\label{fig:modified_code}
\end{figure}




The above approach although is shown to work correctly, but as our
example showed, requires fundamentall changes in the code where 90\% of
the application was rewritten. Additionally, the modifications are
heavily tangled with the application logic which is problem for
developement of and reasoning about large
productions.

However, the major drawback of this approach is the fact that requires
constant alterations in the state of the application when the sessions
come and go. The application is now required to make
sure that a new field is created locally \emph{and} globally when 
new sessions are connected. This can extremely degrade the performance
of the system since it requires direct synchronization between replicas.
We have shown in detail how this adversely effects the performance in section
\ref{sec:eval}, where  we explain our experience in implementing
such an ad-hoc approach.

To make the matter worse, in addition to the above difficulties,
new anomalies are constantly found in the system, which requires
developers to come-up with non-trivial solutions. For example, in the
above application, another type of anomaly can occur when a third user
Chris, uses the application and submits a read, which is routed to a
replica D, that only contains the last write from Bob. Then Chris sees a
window containing "Me too!", which is an undesirable behavior. Now
developers are left with two options. First, they can try implementing
another non-trivial ad-hoc solution, which would further pollute the
application logic, making all the existing reasonings obsolete. Second,
they move the application completely to another store, that offers
stronger forms of consistency, such as causal consistency which would
prevent such anomalies. However, as we will show in section
\ref{sec:eval}, this would result in performance loss and potential
costumer loss.

%
%--- What is our alternative approach
%
\subsection{An Alternative}
We are now offering the developers with an alternative to the above
solutions. 
\tool, is a generic consistency managaement tool, running on
top of an off-the-shelf eventually consistent store, extending it to
a store with multiple configurable consistency levels. 
Developers can define a seperate environment for each operation at the
design stage, and configure each of them according to certain types of
anomalies, prevention of which is guaranteed by the tool. We have
relized this idea, following our observation that the prevention of basically all
types of anomalies, can be boiled down to two tasks,
filterataion and blocking. 


Our tool is equipped with a filteration mechanism, that periodically
refreshes the environments, and allows local effects at the replica to
enter each environment, if their presence would not cause the associated
anomaly occur. This way, \tool completely eliminates the
possibily of operations \emph{seeing effects that they are not supposed
to}, which is one important type of anomalies in distributed systems
(for example, in the above example Chris was not supposed to see Bob's
last comment since some effects were missing).


Similarly, the tool contains a blocker mechanism that makes sure all
operations are executed on their associated environment, only after the
environemnt contains the necessaity effects for preventing the associated
anomaly. Consequently, the operations will \emph{always see the effects
that they are supposed to}. This eliminates another type of anomalies in
the distributed systems, that the lost\_updates anomaly explained
previously, is an
example of. 


Using \tool, the developers are only required to configure environments
for each read operation, using a simple specification language that we
provide them with. The language is seeded with $\soZ$ $\visZ$ relations and allows
user to define different anomalous behaviors.  For example the
followings are the two consistency contracts written in our language,
regarding the two anomalies mentioned in this section 
\begin{smathpar}
\begin{array}{lllll}

1: & \forall (\eff,\eff'). & \eff \xrightarrow{\soZ} \eff' & \Rightarrow
& \eff
\xrightarrow {\visZ} \eff'  \\
2: & \forall(\eff,\eff'). & \eff \xrightarrow{\visZ;\visZ} \eff' &
\Rightarrow & \eff \xrightarrow {\visZ} \eff' 
\end{array}
\end{smathpar}


The first contract, that eliminates the possibility of lost\_updates,
tells the \tool to make sure that all effects that are in session order
to be also in vibility relation. This can be achieved by blocking
operations as explained above. 
The second contract, which prevents the second anomaly mentioned,
requires the system to make sure that if an effect is being made visible
to an operation, the operation should also witness all effects that were
visible to that effect.
In the next section, we will formally introduce our specification
language, and explain how two types of anomalies mentioned above, can be
specified in this language with different syntaxes.
\begin{wrapfigure}{i}{0.2\textwidth}
\centering
	\vspace{-10mm}
	\includegraphics[scale=0.42]{Figures/outline.pdf}
	\label{fig:syncope_outline}
	\caption{\footnotesize \tool}
\label{fig:syncope_outline}
\vspace{-5mm}
\end{wrapfigure}


























%================================ SECTION FOUR: Operational Semantics
\section {Specification Language} 
\label{sec:ctrt_language}
\begin{wrapfigure}{r}{0.48\textwidth}
\begin{minipage}{0.5\textwidth}
\centering
  \begin{smathpar}
  \begin{array}{lclcl}
		r & \in & \texttt{rel.seed} & \coloneqq & \visZ \ALT \soZ \ALT r \cup r \\
               R & \in & \texttt{relation} & \coloneq &  r \ALT r^*  \ALT r;R \ALT r^*;R   \\
	     \pi & \in & \texttt{prop} & \coloneqq & \forall(\eff,\eff').
      ~\eff \xrightarrow{R} \eff' ~\Rightarrow~ \eff \xrightarrow{\visZ} \eff'\\
		\psi & \in & \texttt{spec} & \coloneqq & \pi \ALT \pi \conj \pi\\
  \end{array}
  \end{smathpar}
\caption{Syntax of the Specification Language}
\label{fig:ctrt_syntax}
\end{minipage}
\end{wrapfigure}

The formal syntax of our specification (or contract) language, presented in
Fig.\ref{fig:ctrt_syntax}, allows definition of
\propS{}, that is a FOL formula
establishing dependency relations between effects,
which is necessary to determine effects an operation may witness, under
a consistency requirement.
The language is seeded with $\soZ$ and $\visZ$, respectively representing session
order and visibility over effects, 
and defines dependency \relationS{} as a sequence\footnote{\tool also allows
using closures of seeds, which is omitted here for
simplicity} of seeds,  
where 
({\footnotesize $a \xrightarrow{\rel_1;...;\rel_k} b$})
must be interpreted as 
{\footnotesize$\exists c. (a
\xrightarrow{\rel_1;...;\rel_{k-1}} c
\wedge c \xrightarrow {\rel_k} b)$}
in meta-language. We also define $\nullR{}$ as a relation that never
holds.
Additionally, the language allows definition of \specS{},
a conjunction of propositions, that is used to define a safe environment
free from \emph{multiple} inconsistencies. 
Our language is crafted to capture all fine-grained weak consistency
levels, including the famous session guarantees proposed by Terry et al. \cite{terry}, presented
in Fig.\ref{fig:ctrt_example}.

%
% UB and LB contracts
We finish this section by introducing two syntactical classes of 
contracts, and explaining how they can be
satisfied with different enforcement techniques.
\begin{description}
\item {\textsf LB}: If all the defined dependency relations for a
contract end with an \soZ, i.e. are of
the following form: ({\footnotesize $\forall a. a
\xrightarrow{r_1;r_2;...;\soZ} \hat{\eff}
\Rightarrow a \xrightarrow{\visZ} \hat{\eff}$}), we call it a \emph{lower
bound} (\LB{}) contract, since it specifies the smallest set of
effects that any operation should witness to maintain consistency, e.g.
\rmwCTRT{} and \mrCTRT{} in Fig.\ref{fig:ctrt_example}.

\item {\textsf UB}: Similarly, we define the \emph{upper bound} (\UB{})
contracts, as the ones with all dependency relations ending with a $\visZ$.
These contracts define constraints on the set of effects made visible
to each operation, by enforcing that if an effect is in the set,
certain dependencies of that effect must also be included, e.g.
\visCTRT{} and \mwCTRT{} in Fig.\ref{fig:ctrt_example}.
\end{description}
Our consistency enforcement approach is based on blocking operations with \LB{}
contracts to make sure that they witness \emph{all effects that they are
supposed to}, and filteration for \UB{} contracts to make sure that they
would not witness \emph{effects that they are not supposed
to}. A combination of both approaches is also taken for contracts that are
neither  \LB{} nor \UB{}, i.e hybrid contracts.


%================================ SECTION FIVE: Instrumented Algorithm
%========================================================================================================================
%========================================================================================================================
%========================================================================================================================
\section{Semantics}
\label{sec:semantics}
In this section we formalize our consistency enforcement algorithm with an
operational semantics, which is also a high-level abstraction of our
tool \tool.
Our approach is complete for the specification language defined
in section \ref{sec:ctrt_language}, however, here for simplicity reasons we present an operational semantics 
parametrized over a non-hybrid contract with a single prop. As we will explain
in section \ref{subsec:generalization}, the rules can be easily
generalized to cover multiple consistency levels, each specified by any
given contract. Therefore, in the rest of this section we will assume a given contract $\psi$ of the
following form:
\begin{fmathpar}
\begin{array}{lll}
\psi = \forall a. a \xrightarrow{R} \hat{\eta} \Rightarrow a 
\xrightarrow{\visZ} \hat{\eta}
&\qquad & \quad R = r_1;r_2;...;r_k 
\quad
\end{array}
\end{fmathpar}

The operational semantics is defined via a small-step relation over \emph{execution
states}, which are tuples of the form $\E=(\EffSoup,\visZ ,\soZ)$.
The \emph{effect soup} $\EffSoup$, represents the set of all
effects produced in the system, and  $\visZ$,
$\soZ$ $\subseteq \EffSoup \times \EffSoup$, respectively stand for the
visibility and session order relations
among such effects. Figures \ref{subfig:execution_graph} and
\ref{subfig:execution_example} represent a simple
execution state of a system consisting of 9 effects with associated
primitive relations, where we ommited drawing the transitive $\soZ$ edge between
$\eff_8$ and $\eff_1$, for better readability.
\begin{figure}[t]

	\hspace{-4 mm}
	\begin{subfigure}[b]{0.298 \textwidth}
	\includegraphics[scale=0.31]{Figures/execution.pdf}
 	\subcaption{\scriptsize An execution state \E}
	\label{subfig:execution_graph}
	\end{subfigure}
	%
	\quad \vrule \quad
	%
	\begin{subfigure}[b]{0.3 \textwidth}
	\begin{smathpar}
	\scriptsize    
	\begin{array}{lcl}
	\E.\EffSoup & = & 
	\{\eta_1,\eta_2,\eta_3,\eta_4,\eta_5,\\ & & \;\eta_6,\eta_7,\eta_8,\eta_9\}\\
	\E.\visZ & = & 
	\{(\eta_5,\eta_3),(\eta_4,\eta_3),
	\\ & &\;(\eta_3,\eta_1),(\eta_2,\eta_1),	
	\\ & &\; (\eta_6,\eta_2)\} \\ 
	\E.\soZ & = & \{(\eta_9,\eta_3),(\eta_8,\eta_6),
	\\ & &\; (\eta_6,\eta_1), (\eta_8,\eta_1),
	\\ & &\; (\eta_7,\eta_2) \} 
	\end{array}
	\end{smathpar}
	\subcaption{\scriptsize Effect soup and primitive relations}
	\label{subfig:execution_example}
	\end{subfigure}
	%
	\quad \vrule \quad
	%
	\begin{subfigure}[b]{0.26 \textwidth}
	\begin{smathpar}
	\scriptsize
	\begin{array}{lll}
	\visZ^{-1}  (\eta_1) & = & \{\eta_2,\eta_3\} \\ 
	\soZ^{-1}  (\eta_1) & = & \{\eta_6, \eta_8\} \\
	(\soZ\cup\visZ)^{-1} (\eta_1) & = &
	\{\eta_2,\eta_3, 
	\eta_6,\eta_8\} \\ 
	(\visZ;\visZ)^{-1} (\eta_1) & = &
	\{\eta_2,\eta_3,\eta_4,\eta_5,\eta_6\} \\ 
	(\soZ;\visZ)^{-1}(\eta_1) & = & \{\eta_7,\eta_9\}
	
	\end{array}
	\end{smathpar} \\
	\subcaption{\scriptsize Relation inverse examples}
	\label{subfig:inverse_example}
	\end{subfigure}
	\caption{A simple execution state }
\label{fig:execution_state}
\end{figure}

We use notation $\EffSoup_{(condition)}$
to represent a  subset of $\EffSoup$ consisting of effects that
satisfy the specified condition.
Note that \tool's contracs are in fact constraints over execution states,
where the domain of quantification is fixed to the effect soup
$\EffSoup$, and
interpretation for $\soZ$ and $\visZ$ relations (which occur free in the
contract formulae) are also provided. Thus, execution states are
potential
models for any first-order formula expressable in the contract
language. If an execution state $\E$ is in fact a valid model for a contract
$\psi$, we say that $\E$ satisfies $\psi$, written as $\E
\models \psi$. 


The samentics' reduction step is of the form
\begin{smathpar}
\scriptsize
(\E,\op_{<s,i>}) \;\xrightarrow{\V}\; (\E', \eff),
\end{smathpar}
which can be interpreted as a reduction of the initial execution state
$\E$, initiated by a replica with a local 
set of effects $\V$, after it executes an operation
$\op$, which is the $i^{th}$ request from the session $s$. 
During this reduction step a new effect $\eff$ is produced and added to
the system, resulting a new execution state $\E'$ with updated effect
soup and primitive relations.

Before introducing the operational semantics, we will first formally
present the required definitions in the next section. Namely, we will define  
notions of the \emph{inverse} of a given relation $R$, and the \emph{maximally
closed subset} of a given set of effects $V$ under a contract $\psi$. 




%=============================================================================================================
%--------- Definitions to be used in the semanrics
%=============================================================================================================
\subsection{Preliminaries}
\label{subsec:prelim}
We start by formally defining the inverse of a seed relation ($\rel \in$
\seedS{}) given an 
execution state $\E$:
\begin{equation}
\label{eq:r_inv}
\scriptsize
\rel^{-1}(\Set) = 
\begin{cases}
\begin{array}{lcl}
\bigcup\limits_{b\in \Set}\{a|(a,b) \in \E.\rel \} & \myif
&\rel\in\{\soZ,\visZ\}\ \\ 
\rel_1^{-1}(\Set)\cup \rel_2^{-1}(\Set) & \myif & \rel=\rel_1\cup \rel_2
\end{array}
\end{cases}
\end{equation}
Note that when the input of an inversed relation is a singleton
$\{\eta\}$, we drop the brackets and simply write it as
$\rel^{-1}(\eta)$.
We now present the definition of the inverse of 
sequences (of size larger than 1) of \seedS{} as follows:
\begin{equation}
\label{eq:seq_inv}
\scriptsize
\begin{array}{lllll}
b \in  (R';r)^{-1}(a) & \iff & \exists c. c \in r^{-1}(a)
& \wedge & b \in (R')^{-1}(c) 
\end{array}
\end{equation}
Inverse of sequences of length 1 is also implicitely defined as the
inverse of the enclosed \seedS{}.

Following definitions (\ref{eq:r_inv}) and (\ref{eq:seq_inv}), since
\relationS{} in our specification language is defined as 
 either a \seedS{}, or a sequence of them, we are
now ready to formally define the inverse of any given relation $R\in$
\relationS{}.
However, note that the definition (\ref{eq:seq_inv}) fails to capture the reality of distributed
systems, where all computations are done locally by replicas, which
might have access to only a \emph{subset of all effects} at any given
moment. For example, consider  $(\soZ;\visZ)^{-1}(\eta_1)$ of the
execution state in figure
\ref{fig:execution_state}. In order to compute this set, based on the
recursive defintion of (\ref{eq:seq_inv}) we have: 
\begin{smathpar}
\scriptsize
\begin{array}{lllll}
b \in  (\soZ;\visZ)^{-1}(\eta_1) & \iff & \exists c. c \in
\visZ^{-1}(\eta_1)
& \wedge & b \in (\soZ)^{-1}(c)
\end{array}
\end{smathpar}
Since
there exist \emph{mid-level} effects $\eta_2$ and $\eta_3$, such that satisfy the above
definition respectively 
for $b=\eta_7$ and $b=\eta_9$, we have: $(\soZ;\visZ)^{-1}(\eta_1) =
\{\eta_7,\eta_9\}$.
Now assume a replica only contains $\{\eta_1, \eta_6, \eta_7, \eta_8,
\eta_9\}$ and wants to check if the dependencies of $\eta_1$ are locally present 
or not. Even though based on the above definitions the
answer is yes (since the replica does contain $\{\eta_7,\eta_9\}$), but in
reality the replica would not be able to verify that, since the mid-level
effects $\eta_2$ and $\eta_3$ are not present at the replica yet. 

To capture the above property, we now present partial definition of the inverse of a given relation $R \in$
\relationS{} \emph{according to a set of available effects $V$}. 
We define the inverse, only if all the required mid-level effects are
present in $V$ using definition (\ref{eq:r_inv}) and a slightly different version of the 
definition (\ref{eq:seq_inv}).
\begin{equation}
\scriptsize
b \in R^{-1}_V(a) \iff
\begin{cases}
\begin{array} {lllll} 
\bot & \;\myif\; & R = \nullR& & \\
b \in \rel^{-1}(a) & \;\myif\; & R=\rel & & \\
\exists c. c \in
\rel^{-1}(a) \wedge b \in (R')^{-1}(c) \wedge
\rel^{-1}(a) \subseteq V   & \;\myif\; & R=R';\rel & & \;
\end{array}
\end{cases}
\end{equation}
Note that the only difference between the third case in above definition
and the definition (\ref{eq:seq_inv}), is the last conjunct which is
added to ensure the presence of mid-level
effects before performing the next recursive call.  

Now, we define \trunc{} as a function that
given  $R \in$ \relationS{}, removes the last element from the
sequence (if there is any) in $R$, i.e.
\begin{equation}
\scriptsize
\trunc{R} = 
\begin{cases}
\begin{array}{lcl}
\nullR & \myif & R = \rel \quad \mathtt{or} \quad R = \nullR \\
R' & \myif & R = R';\rel 
\end{array}
\end{cases}
\end{equation}
Finally, we define \emph{closed subsets} of a given set of
effects $V$ under the contract $\psi$, which the maxiamal element among such
subsets is also defined next\footnote{We abuse the previously defined notation slightly
and use a \emph{set} of effects as the input to the inverse of
$R\in $\relationS{}, which simply means 
the union of the results of apply the function for all the effects in
the input set}:
\begin{equation}
\scriptsize
\begin{array}{rlllll}
\mathtt{closed \; subsets:} &  V' \in \left \lfloor  V \right \rfloor & \iff & V' \subseteq V & \wedge &
(\trunc R)_V^{-1}(V') \subseteq V'   \\
\mathtt{maximally \; closed \; subset:} & V' = \left \lfloor  V \right
\rfloor_{\mathtt{max}} & \iff & V' \in \left \lfloor  V \right \rfloor &
\wedge & \not\exists V'' \in \left \lfloor  V \right \rfloor. |V''|>|V'|
\end{array}
\end{equation}


%=============================================================================================================
%--------- The operational semantics
%=============================================================================================================
\subsection{Core Operational Semantics}

%--- Section intro
In this part we present the reduction rules, representing our
consistency preservation approach.
Figure \ref{fig:semantics} presents the set of rules defining the
auxiliary relation ($\hookrightarrow$) and small-step reduction relation 
($\rightarrow$) over executions. The latter relation is parametrized
over a set $V$,
that represents the set of effects that are available at the replica 
taking the step. Obviously $V$ must be a subset of the effect soup 
of the initial execution, however, there is no other restrictions on $V$,
since we only assume eventual consistency at the underlying store.

%--- The [OPER] rule
The rule
\rulelabel{oper} represtns the procedure of producing a new effect $\eff$, by witnessing a set
of effects $S$. 
An effect is formally defined as a tuple $\eff=(s,op,v)$, representing the
session and the operation name 
whose execution created $\eff$, and the value
that the replica returns as the response to that operation.
The rule explains how the execution state changes after producing an
effect at a replica. Specifially, in the new state, the effect soup
$\EffSoup'$ contains the newly created effect $\eff$, and the relations $\visZ'$
and $\soZ'$ capture the fact that all effects in the set $S$ were made
visible to $\eta$, and all effects from the same session that were
already presenet in the intial execution state, should be in session
order with $\eff$ in the final execution state.

\begin{figure*}[h]
\raggedright
%
\textbf{Auxiliary Definitions}\\
%
\begin{minipage}{\columnwidth}
\begin{smathpar}
\stretcharraybig
\begin{array}{lclcl}
  \multicolumn{5}{c}{
    \hspace{-8mm}
    {op} \in \mathtt{Operation\; Name} \spc \spc
    {s} \in \mathtt{Session\; Id} \spc\spc
    {i} \in \mathtt{Effect\; Id} \spc\spc
  }\\
  \eff & \in & \mathtt{Effect} & \coloneqq &  (s,i,op,v)\\
  F_{op} & \in & \mathtt{Op.\,Def.} & \coloneqq & \set{\eff} \rightarrow \eta\\
  \EffSoup & \in & \mathtt{Eff\,Soup}	  & \coloneqq & \set{\eff} \\
  \visZ,\soZ &	\in & \mathtt{Relations} & \coloneqq & \set{(\eff,\eff)} \\
  {\E} 		& \in & \mathtt{Exec\;State}  & \coloneqq & \Exec \\
\end{array}
\end{smathpar}
\end{minipage}
%

\begin{smathpar}
\begin{array}{c}
\ssn(s,\_,\_,\_) = s \spc\spc
\id(\_,j,\_,\_) = j \spc\spc
\oper(\_,\_,op,\_) = op \spc\spc
\rval(\_,\_,\_,v) = v\\
\end{array}
\end{smathpar}

\textbf{Auxiliary Reduction} \;
  \fbox{\(\auxred{S} {(\E,op_{<s,i>})} {} {(\E',\eff)}\)}\\

\begin{minipage}{\textwidth}
\rulelabel{Oper}
\begin{smathpar}
\stretcharraybig
\begin{array}{l}
\RuleTwo
{
%\Theta(\rho \mapsto (v,cache)) \qquad
F_{op}(S) = \eta \qquad
\ssn(\eta) = s \qquad 
\id(\eta) = i \qquad
%\{\eff'\} = \EffSoup_{({\sf SessID}=s,\,{\sf SeqNo}=i-1)}\\
\EffSoup' = \EffSoup \cup \{\eff\} \qquad S \subseteq \EffSoup\\
\visZ' = \visZ \cup S \times\{\eff\}\qquad
\soZ' = \soZ \cup \{(\eta',\eta) \,|\, \eta'\in \EffSoup \conj 
    \ssn(\eta')=s \conj \id(\eta')<i\}\qquad
%\soZ' = (\soZ^{-1}(\eff') \cup \eff') \times\{\eff\} \cup \soZ
}
{
  \auxred {S} {((\EffSoup,\visZ,\soZ), op_{<s,i>}))}
  {} {((\EffSoup',\visZ',\soZ'),\eta)}
}
\end{array}
\end{smathpar}
\end{minipage}
\vspace{4mm}\\
\textbf{Operational Semantics} \;
  \fbox{\((\E,op_{<s,i>}) \;\xrightarrow{V}\; (\E',\eff)\)}\\
\begin{minipage}{2.8in}
\rulelabel{vis Exec}
\begin{smathpar}
\stretcharraybig
\begin{array}{l}
\RuleTwo
{
     V \subseteq E.A \spc  R^{-1}_{V}(\{\eta\}) \subseteq V \spc  (vis \subseteq r_k) \spc 
  V'= \left \lfloor V \right \rfloor_V \\
  \auxred {V'} {(E, op_{<s,i>}))}
    {} {(E',\eta)} 
}
{
  (\E,op_{<s,i>}) \;\xrightarrow{V}\; (\E', \eff)
}
\end{array}
\end{smathpar}
\end{minipage}
\hspace{12 mm}
\begin{minipage}{2.3in}
\rulelabel{non-vis Exec}
\begin{smathpar}
\stretcharraybig
\begin{array}{l}
\RuleTwo
{
 
  V \subseteq E.A \spc  R^{-1}_{V}(\{\eta\}) \subseteq V\spc (vis \nsubseteq r_k )\\
  \auxred {V} {(E,op_{<s,i>}))}
    {} {(E',\eta)}
}
{
  (\E,op_{<s,i>}) 
    \;\xrightarrow{V}\;
  (\E',\eff) 
}
\end{array}
\end{smathpar}
\end{minipage}


\caption{Core Operational semantics of a replicated data store.}
\label{sem:oper}
\end{figure*}

 %--- The Figure Containing the Rules

%--- rule for (->) relation
Now we explain the rules for reduction relation $(\xrightarrow{V})$,
starting with \rulelabel{ub exec}, which represents the execution of operations
in a replica that updates the global state and produces a new
effect under a UB contract. 
The rule requires operations witnessing only the maximally consistent
subset $V'$ of the local set of  available effects $V$. In other words,
the rule filters out the effects that may result anomalies and shows the
safe environment to the operation.

The next rule, \rulelabel{lb exec}, represents the step taken when an
operation is performed under an LB contract. The precondition 
$R_V^{-1}(\eff)\subseteq V$ in the rule, ensures that the reduction
happens only if the effects necessary to avoid the specified anomaly are
present in V. The operations performing under an LB contract must be
blocked, untill all the necessary effects (and possibly required
mid-level effects) become available in the locally available set of
effects $V$. Note that in this case effects are not filtered out, and
the operation witnesses all effects in set $V$.

%=============================================================================================================
%--------- Theorem on correctness of enforcement
%=============================================================================================================
\subsection{Soundness}
\label{subsec:sound}
\begin{definition}
For a given contract ($\psi=\forall(a,b). a
\xrightarrow{Q} b \Rightarrow a \xrightarrow{vis} b $), we define a set
of effects S, to be  $\psi$-consistent under an execution $E$, if and
only if: 
$\forall (\eff \in S). \forall(a\in E.\EffSoup). Q(a,\eta)
\Rightarrow a \in S$ 
\\(We use definition of $Q^{-1}$ from the previous section to formally
define: $Q(a,b) \iff a \in Q_{E.A}^{-1}(b)  $)
\end{definition}

\begin{theorem}
For all operation executions of the form: 
$$
(\E,op_{<s,i>}) 
    \;\xrightarrow{V}\;
  (\E',\eff) 
,$$ 
if $V$ is $\psi$-consistent then $(V\cup\{\eta\})$ is also
$\psi$-consistent.
\end{theorem}
\begin{proof}
Sine $\eff$ is the only effect added to $V$, we only need to check the
following property for $\eta$, in order to prove $\psi$-consistency of
$V\cup\{\eta\}$:
$$\forall (a \in E.A). Q(a,\eta) \Rightarrow vis(a,\eta)  $$
\\
(1) $Q(a,\eta) \spc$ (hypothesis)
\\(2) $a \in Q^{-1}_V (\eta) \spc$  (we need to prove this)
\\(3) $a \in V \spc$ (premise of the rules and (2))
\\(4) $V$ is $\psi$-consistent so from (3) and definition of
consistency, the property holds for $a$
\paragraph{} In order to prove (2), we \emph{probably} have to destruct the contract and
show that if it ends with $so$, the property trivially holds since the
vis set of $\eta$ is fixed. For the other case the premise of the rule
[VIS Exec] will have the enought information to prove.
\end{proof}
\newpage


%=============================================================================================================
%--------- Theorem on maxVis and minWait
%=============================================================================================================
\subsection{Optimality}
Now we will present theorems 2 and 3, the former showing that the set of effects made visible during each
operation execution is the largest one possible, and the later
presenting the liveness property of the semantics, which states that the
store will take a step, if the required dependencies are locally
present. This guarantees that the store would never get stuck, since the
eventual delivery of all updates to all replicas is guaranteed 
by the underlying ECDS.
%
%Subsection of the maximality of the set made visible and liveness
%properties 
%
\begin{theorem}
\label{theorem:two}
For all reduction steps
{\footnotesize $
(\E,op_{<s,i>}) 
    \xrightarrow{V}
  (\E',\eff) 
$},
the set of effects made visible to $\eta$ is maximal. i.e. for all
 {\footnotesize $a \in V$}, if 
 {\footnotesize $ \SC {\trunc{R}} a V$}, then 
\begin{fmathpar}
(a,\eta) \not\in \E'.\visZ \Rightarrow 
(\E'.A,\E'.\visZ \cup \{a,\eta\}, \E'.so) \not\models \psi[\eta/\hat{\eta}]
\end{fmathpar}
\end{theorem}


%
% THE MINIIMAL WAIT THEOREM
%

\begin{theorem}
\label{theorem:three}
For all execution states $E$, if there
exists ($S\subseteq E.A$) such that: 
\begin{smathpar}
\auxred{S} {(\E,op_{<s,i>})} {} {(\E',\eff)} \spc \wedge \spc (\mathtt{S \cup \{\eta\} \; is 
\;} \psi \mathtt{-consistent \; under \;} E' )  
\end{smathpar}
then there exist  $E''$, $\eff'$ and $V\subseteq E.A$ such that:
$((\E,op_{<s,i>})\;\xrightarrow{V}\;(\E'',\eff'))$
\end{theorem}
\begin{proof}
The proof is given by choosing set $V$ to be equal to $S$, and then
considering two cases, where either $S$ or $\left \lfloor S \right
\rfloor_S$ are made visible to operations, if the contract is respectively
waiting and non-waiting. In both cases, all premises of taking an step
are satisfied.
A detailed proof can be found in appendix
\ref{app:proof3}.
\\
\end{proof}



%=============================================================================================================
%--------- How it can be generalized for all contracts
%=============================================================================================================
\subsection{Generalization}
\label{subsec:generalization}
We finish this section by extendeding our ideas in two dimentions. 
We will first explain how to handle an arbitrary
contract $\psi$ of the following form:  
\begin{fmathpar}
\psi = \pi_1 \wedge \pi_2 \wedge ... \wedge \pi_m \qquad \qquad 
\pi_i = \forall (a,b). a \xrightarrow{R_i} b \Rightarrow a
\xrightarrow{\visZ} b
\end{fmathpar}
Later, we will
explain how to maintain multiple levels of consistency simultaneously,
each of which is defined for a different operation name. We will assume an arbitrary contract
$\psi_{\op}$ for every user-defined operation $\op$, and explain how to
modify our system model to preserve them all.

To begin with, as we mentioned earlier, all propositions in our specification language,
either put a maximal or a minimal bound on the subset of local effects 
to be made visibe to each opreation. 
This simply means that when the system is given a conjunction
of propositions, it should define the such subsets in a way, so it would not violate
\emph{any} of them. 
Therefore, by a few modifications we can extend the system to support
all contracts. Firstly, the single premise $R_V^{-1}(\eta) \subseteq V$
in the reduction rule should be replaced with the following
conjunction: 
\begin{fmathpar}
\bigwedge_{1 \leq i \leq m} (R_i)_V^{-1}\subseteq V
\end{fmathpar}
Secondly, the definition of the maximal closed subset of local effects
must also be modified to a subset that is closed under \emph{all} given
relations:
%---------------------------------------------------------------------
% I am not sure if we should include the formal definition here. It is
% unnecessarily complex
\begin{fmathpar}
\left \lfloor S \right \rfloor_V = S' \spc \iff \spc S'
\subseteq S \; \wedge \;
\bigwedge(R_i)_V^{-1}(S') \subseteq S' \; \wedge \; 
\not\exists
S''.(\bigwedge ((R_i)_V^{-1}(S''))\subseteq S''\wedge |S''|>|S'|)
\end{fmathpar}
%---------------------------------------------------------------------

Moreover, for modifying the system to handle multiple contracts
simultaneously, we can
extend the local effect set $V$, to a sequence
of sets $V_{\op_i}$, each maintaning  the consistency level for an
operation type $\op_i$. Now we define the modified form of execution steps as
follow:
\begin{fmathpar}
(\E,\op_{<s,i>}) 
    \;\xrightarrow{V_{\op}}\;
  (\E',\eff) 
\end{fmathpar}
The local effect set $V$ must also be replaced with $V_{\op_i}$ in the 
premises of the reduction rules, so each operation of type $\op_i$ would
only witness the associated subset for its own consistency requirements.
This abstractly represents our implementation, in the sense that all operations
work only on a specific subset of available effects at any replica. The subset, is
maintained according to the contract assosiated with each operation, and
is guaranteed to preserve the consistency requirements following the
theorems of sections \ref{subsec:sound} and \ref{subsec:opt}. 



%================================ SECTION SIX: Implementation and Evaluation
\section{Evaluation}
\subsection{Implementation}
\subsection{Complexity of Ad-hoc approach}
\subsection{Latency and Staleness Comparison}



%================================ SECTION SEVEN: Related Works
\newpage
\section{Evaluation}
\label{sec:eval}
%
\begin{figure}[b]
\centering
\begin{center}
\begin{scriptsize}
\begin{tabular}{|l |  l | l|} 
\hline
 \multicolumn{1}{|c}  {\bf Benchmark} & \multicolumn{1}{c} {\bf
 Consistency} & \multicolumn{1}{c|}  {\bf
 Description}\\ [0.5ex] 
\hline
Counter  & MR & {Monotonicly increasing counter, e.g.
YouTubes' watch 
count}\\ \hline
DynamoDB  & RMW & {Integer register allowing various conditional puts and gets} \\ \hline
Online Store & RMW &  {Online store with shopping carts
and modifiable item prices} \\ \hline
Bankaccount  & 2VIS $\wedge$ RMW & {Offering deposit, withdraw and get balance operations}\\ \hline
Shopping List   &  MW $\wedge$ RMW & {A shopping list with
concurrent adds and deletes functionality}\\ \hline 
Microblog  &  MW, RMW & {A Twitter-like application modeled after
Twissandra}\\\hline
Rubis  & RMW, RMW$\wedge$2VIS & {eBay-like
application with browsing, supporting user wallet} \\
\hline
\end{tabular}
\end{scriptsize}
\end{center}
\caption{Fine-grained consistency requirement in benchmark programs}
\label{fig:dist_table}
\end{figure}

%intro: benchmark programs
In this section we present an evaluation study of our implementation,
including a report on
benchmark applications that utilize fine-grained weak consistency
requirements, expressable
in \tool's specification language.
Fig.\ref{fig:dist_table} presents seven of such programs, including
individual data types as well as larger programs consisted of multiple
data types. 

%multiple consistency levels for each program
Each program offers various operations, each of which is assigned a
potentially different consistency requirement,
representing the need for a multi-consistnet environement for
efficient execution of the programs. Surprisingly, we found no program
intrinsically requiring causal consistency; all known consistency anomalies that operations
may be involved in, are expressable with simple fine-grained contracts
composed of
dependency relations of length 1 or 2,
which differs from what was knwon in the context before, where all such
operations were considered to require CC.

%conjunction of consistency requirements for even a single operation
Additionally, in many cases we found operations that may be involved in
multiple anomalies, requiring simultaneous enforcement of different
consistency guarantees, which shows the unfeasability of hand-writing
such guarantees, considering the vast set of known consistency
anomalies. 
%
%example
For example, consider  a bank account application, which offers
\dRV{}, \wdRV{} and \gbRV{} operations, where \wdRV{} is a
strongly consistent operation that succeeds only if there are sufficient
funds in the account. There are two annomalous scenarios associate with
\gbRV{} in this program:
\begin{enumerate*}[label=(\roman*)]
\item when a user performs a \dRV{}, which is however, not refleceted
in the subsequent \gbRV{}
\item when a \gbRV{} witnesses a \wdRV{} effect, without witnessing all
the \dRV{} effects that were visible to it,
which may result in \gbRV{} returning a
negative balance.
\end{enumerate*}
As it is presented in Fig.~\ref{fig:dist_table}, in order to preempt
these anomalies, \gbRV{} requires
both \rmwCTRT{} and \visCTRT{} guarantees simultaneously.
\begin{figure}[t]
        \centering
	\begin{subfigure}[b]{0.48\textwidth}
	\end{subfigure}
	\hfill
	\begin{subfigure}[b]{0.49\textwidth}
	\includegraphics[scale=0.36]{Figures/latency.pdf}
	\caption{Example execution.}
	\label{subfig:comment_example}
	\end{subfigure} 
\\ \hrulefill
\caption{A distributed application for comment section
management}
\label{fig:comment_app}
\end{figure}




% performance evaluation
For our performance evaluation, we deploy \tool on a cloud cluster,
consisting of three fully replicated Cassandra replicas, running on
seperate machines within the same
datacenter. 
Each machine is instantiated with a
\tool shim layer, that responds to clients,  
 which are instantiated on a virtual machine 
co-located with one of the replicas on a machine.
We deploy the cluster on three \texttt{m4.4xlarge} Amazon EC2 instances
in US-West (Oregon) region, with inter-machine communication time of 5ms.

% The problem with Cassandra
Inter-replica communications in Cassandra use TCP connections, 
causing all messages get delivered with no loss and reordering, which
is in practice, far  more consistent than EC,  and masks out the
performance gain from our fine-grained consistency guarantees.
Consequently, to simulate a
realistic EC environment, we inject artificial message loss at the shim
layers, where a message delivery is delayed for 1
second in case it is lost.

%The latency and staleness gain using fine-grained consistency
Fig.~\ref{fig:eval}(a) and \ref{fig:eval}(b) represent
our experimental results, with a workload generated 
by 50 concurrent clients repeatedly running sessions, each composed of three
operations, where operations uniformly choose from 5 objects and are
performed under the specified consistency level. 
We increase the
percentage of delayed messages from 0 to 14, where each experiment ran for
100 repeaded sessions per client. Additional to client perceived
latency, we also measure the staleness of operations, which we define as
the average ratio of the number of visible effects,
to the number of all available effects, at the time an operation is executed.

% latency result
In the first set of experiments, we measure latency under
three different \LB{} contracts, all implemented in \tool. As
expected,
causal consistency and RMW experience respectively the highest and the
lowest
performance loss as the percentage of lost messages is increased\footnote{In fact, 
they define the storngest and the weakest
\LB{} dependency relations expressable in our language:
$(\xrightarrow{\soZ})$ and $(\xrightarrow{(\soZ\cup\visZ)^*})$}.
At only $4$ percent message loss rate, we see $17\%$ higher latency under MR
contract compared to RMW, and similarly $67\%$ higher latency in CC
compared to MR, whereas with $10$ percent message loss, the numbers are
increased to $18\%$ and $87\%$.


%latency result
Similarly, we repeated the experiment with 3 \UB{} contracts, where
\emph{causal visibility} (CV) contract (i.e. {\footnotesize $ \forall a.
a\xrightarrow{(\soZ\cup\visZ)^*;\visZ}\hat{\eta}\Rightarrow a\xrightarrow{\visZ}
\hat{\eta} $}), offers the most stale data when the percentage of lost
messages is increased, whereas staleness in MW is the lowest and is
barely effected. We report $3\%$ ($6\%$) difference 
between staleness of data under MW and 2VIS, and $4\%$ ($7\%$)
difference between 2VIS and CV,
at four (ten) percante message
loss rate.



%handwritten compared
Finally, in order to evidence the practicality of \tool, we implemented
an ad-hoc mechanism to prevent lost-updates anomaly, for a simple
counter application. Fig.~\ref{fig:eval}(c) shows the latency results of
this application compared to the same in \tool, under the same
setting s before (albeit with no message loss). We report $78\%$
higher latency for the handwritten code compared to \tool with 50
concurrent clients.
We experienced many bookkeeping complications with the handwritten
implementation, mainly because of the lack of meta-data queries in
Cassandra which needs strongly consistent table alterations at the
beginning and the end of each session, as mentioned before.












































%================================ SECTION EIGHT: Conclusion 
\section{Related Works}
\label{sec:rel_works}

%RDTs
Distributed data structures composed of operation-based replicated
data types (RDTs) \cite{rdt,crdt} have been utilized in a number of
real-world systems \cite{tango,cassandra}.  However, these systems are
developed without assuming any principled notions of consistency, and
thus have goals different from \tool.  Like \cite{bolton}, \tool's
focus is entirely on consistency management, and leaves issues of
liveness and durability management to the underlying data store.


%coordination
The specification of consistency requirements of replicated
data-objects have been studied in several works
\cite{autoc,mahsa,bloom}, where multiple sufficient conditions and
analysis techniques are proposed to detect potential coordination
points in programs to enforce different notions of consistency.  \tool
shares similar goals, manifested within a lightweight runtime
enforcement mechanism that dynamically validates fine-grained
consistency specifications.

%Consistency on top of EC
Numerous systems \cite{geofast,petersen,cbs,chapar,bolton,quelea}
define and implement various levels of consistency guarantees in order
to protect applications from anomalies admitted under EC.
\cite{chapar} presents a verified implementation for a causally
consistent store, assuming a system model with \emph{session
  stickiness}, where unlike \tool, operations from a session are
always routed to the same replica. The idea of a causally consistent
shim layer on top of an off-the-shelf ECDS, is proposed in
\cite{bolton} and is also utilized in \cite{quelea}, which offers
three coarse-grained levels of consistency.  \tool extends the shim
layer in \cite{quelea} by maintaining \emph{multiple} fine-grained
weak consistency levels.

%% % contract language
%% \tool's contract language shares similarity to \cite{rdt,rdtabs}, but
%% similar to \cite{quelea} is crafted for consistency
%% specification. Compared to \cite{quelea}, \tool introduces a more
%% strictly structured syntax (without losing any generality), allowing
%% the classification of consistency guarantees, which enables a generic
%% consistency enforcement method, whereas \cite{quelea} relies on an SMT
%% solver to \emph{map} the operations, to the weakest \emph{pre-defined}
%% consistency guarantee that satisfies the given contract.













%% Acknowledgments
%\begin{acks}                            %% acks environment is optional
                                        %% contents suppressed with 'anonymous'
  %% Commands \grantsponsor{<sponsorID>}{<name>}{<url>} and
  %% \grantnum[<url>]{<sponsorID>}{<number>} should be used to
  %% acknowledge financial support and will be used by metadata
  %% extraction tools.
%  This material is based upon work supported by the
%  \grantsponsor{GS100000001}{National Science
%    Foundation}{http://dx.doi.org/10.13039/100000001} under Grant
%  No.~\grantnum{GS100000001}{nnnnnnn} and Grant
%  No.~\grantnum{GS100000001}{mmmmmmm}.  Any opinions, findings, and
%  conclusions or recommendations expressed in this material are those
%  of the author and do not necessarily reflect the views of the
%  National Science Foundation.
%\end{acks}


%% Bibliography
%\bibliography{bibfile}


%% Appendix
\appendix
\newpage

\section{Proofs}
\label{appendix:proofs}
Here, we present detailed proofs of the theorems of the paper by
first presenting a useful lemma.\\
\rule{\textwidth}{1pt}
\begin{lemma}
For all relations {\footnotesize $R\in$ \relationS{}} and execution steps:
{\footnotesize $
(\E,op_{<s,i>}) \;\xrightarrow{V}\; (\E',\eff)
$}
interpretatin of $R$ under $\E$ and $\E'$ (Simply denoted by $R$ and $R'$) differ only on the single
effect $\eta$,
i.e.  {\footnotesize $\forall a,b\not= \eta \Rightarrow (R'(a,b) \Leftrightarrow
R(a,b))$}.
\label{lemma1}
\end{lemma}
\rule{\textwidth}{0.25pt}
\begin{proof}
We prove $\Rightarrow$, the other direction can be shown
similarly. We have the following goal and hypotheses: 
\begin{fmathpar}
\begin{array}{ll}
H_0: & (\E,op_{<s,i>}) \;\xrightarrow{V}\; (\E',\eff) \\
H_1: & a,b \not= \eta\\ 
H_2: & R' (a,b)\\
G_0: & R (a,b)
\end{array}
\end{fmathpar}
Now by destructing $R$, we get the followings, in the only non-trivial
case:
\begin{fmathpar}
\begin{array}{ll}
H_3: & (\trunc{R};r)' (a,b) \\
G_1: & (\trunc{R};r) (a,b)
\end{array}
\end{fmathpar}
which by rewriting the definition in $H_4$ and $G_1$, we get that $y$ exists s.t.
\begin{fmathpar}
\begin{array}{ll}
H_4: & (\trunc{R})'(a,y)\\
H_5: & r'(y,b) \\
G_2: & \exists x. (\trunc{R})(a,x) \wedge (r)(x,b)
\end{array}
\end{fmathpar}
Now by instantiating $G_2$ with $y$ and by inversion: 
\begin{fmathpar}
\begin{array}{ll}
G_3: & (\trunc{R})(a,y) \\ 
G_4: & r(y,b)
\end{array}
\end{fmathpar}
Now by induction on the length of the relation $R$, $G_3$ is trivially
proved and we are left with the following: 
\begin{fmathpar}
\begin{array}{ll}
H_5: & r'(y,b) \\
G_4: & r (y,b)
\end{array}
\end{fmathpar}
Now by inversion on $H_0$ we get two cases. We show the \rulelabel{lb
exec} case, and the other case can be shown similarly (only difference
is $V$ being replaced by $V'$, which
has no effect on the proof):
\begin{fmathpar}
\begin{array}{ll}
H_7: & \visZ' = \visZ \cup V \times \{\eta\} \\
H_8: & \soZ' \; = \soZ \cup  (\EffSoup_{({\sf
SessID}=s)}\times \{\eta\}) 
\end{array}
\end{fmathpar}
Now, because of $H_1$ (and the fact that $y\not= \eta$) it is easy to get the following from $H_7$ and
$H_8$:
\begin{fmathpar}
\begin{array}{ll}
H_9: & \visZ'(y,b) \Rightarrow \visZ(y,b) \\
H_{10}: & \soZ'(y,b) \Rightarrow \soZ(y,b)
\end{array}
\end{fmathpar}
which directly prove $G_4$, in both cases derived by destructing $r$.
\end{proof}








\subsection{Proof of Theorem \ref{theorem:one}}
\label{app:proof1}
We have the following two hypotheses and the goal:
\begin{smathpar}
\begin{array}{ll}
H_0: & (\E,op_{<s,i>}) \;\xrightarrow{V}\; (\E',\eff)  
\\
H_1: & V \; \mathtt{ is }\; \psi\mathtt{-consistent \; under \; \E} \\
G_0: & V \cup \{\eta\} \; \mathtt{is} \; \psi\mathtt{-consistent \;
under \; \E'}
\end{array}
\end{smathpar}
Based on the definition of consistency, we should prove the following:
\begin{smathpar}
\begin{array}{ll}
G_1: & \forall(x\in V\cup\{\eta\}).\forall (a\in E'.A). R(a,x) \Rightarrow a \in V\cup\{\eta\}
\end{array}
\end{smathpar}
By intro (and by the fact that $\E'.\EffSoup = \E.\EffSoup \cup \{\eff\}$) we will have the following new hypotheses and goal:
\begin{smathpar}
\begin{array}{ll}
H_2: & x \in V\cup \{\eta\} 
\\
H_3: & a \in E.A \cup \{\eff\}\\
H_4: & R(a,x) \\
G2: & a \in V \cup \{\eta\}
\end{array}
\end{smathpar}
By destructing $H_2$ we have two cases:
\\{ (i)} $\; x \in V$: Since $V$ is $\psi$-consistent, from $H_4$ we have $a \in V$ which proves the goal $G_2$ 
\\{ (ii)} $x = \eta$: By replacing $x$ with $\eta$ in $H_4$ we will have the following:
\begin{smathpar}
\begin{array}{ll}
H_5: & R(a,\eta) \\
\end{array}
\end{smathpar}
Based on definitions, we will have the following:
\begin{smathpar}
\begin{array}{ll}
H_6: & a \in R_{E.A}^{-1}(\eta) \\
\end{array}
\end{smathpar}
We can now assume $R=r_1;r_2;...;r_k$, and derive from $H_6$ that there exists $c$ such that:
\begin{smathpar}
\begin{array}{ll}
H_7: & c \in r_k^{-1}(\eta) \\
H_8: & a \in (r_1;...;r_{k-1})_{E.A}^{-1}(c)  \\
H_9: & r_k^{-1}(\eta) \subseteq E.A\\
\end{array}
\end{smathpar}
Now by inversion on $H_0$ we will have the following:
\begin{smathpar}
\begin{array}{ll}
H_{10}: & R_V^{-1}(\eta) \subseteq V\\
\end{array}
\end{smathpar}
which by definition will result:
\begin{smathpar}
\begin{array}{ll}
H_{11}: & r_k^{-1}(\eta) \subseteq V\\
\end{array}
\end{smathpar}
Now from $H_7$, $H_8$ and $H_{11}$ we can derive: 
\begin{smathpar}
\begin{array}{ll}
H_{12}: & a \in R_V^{-1}(\eta)\\
\end{array}
\end{smathpar}
and from $H_{10}$ and $H_{12}$ we have $a \in V$ which proves the final
goal $G_2$. \\QED.


\subsection{Proof of Theorem \ref{theorem:two}}
\label{app:proof2}
\begin{footnotesize}
\vspace{-2mm}\rule{\textwidth}{1pt}\\ \vspace{0mm} \\
We prove the theorem by contradiction:
\begin{fmathpar}
\begin{array}{ll}
H_0: & (\E,op_{<s,i>}) \;\xrightarrow{V}\; (\E',\eff) \\
H_1: & a \in V \\ 
H_2: & (a,\eta) \not\in E'.\visZ\\
H_3: & (E'.A,E'.\visZ \cup \{(a,\eta)\},E'.\soZ) \models  \psi[\eta/\hat{\eta}]\\
H_4: & (\trunc{R}^{-1}_V(a) = \trunc{R}^{-1}_{E.A}(a)) \\
G_0: & \bot
\end{array}
\end{fmathpar}
Now we call {\scriptsize $(E'.A,E'.\visZ \cup \{(a,\eta)\},E'.\soZ)$} as
{\scriptsize $E''$} and derive
the following from $H_3$:
\begin{fmathpar}
\begin{array}{ll}
H_5: & E'' \models \forall x. x\xrightarrow{R}\eta \Rightarrow x
\xrightarrow{\visZ} \eta
\end{array}
\end{fmathpar}
because $E''$ defines the universe of quantification (and since
$E''.A=E'.A$), we get the
following:
\begin{fmathpar}
\begin{array}{ll}
H_6: & \forall (x\in E'.A). E'' \models x\xrightarrow{R}\eta \Rightarrow x
\xrightarrow{\visZ} \eta
\end{array}
\end{fmathpar}
and is rewritten as the following:
\begin{fmathpar}
\begin{array}{ll}
H_7: & \forall (x\in E'.A). (E'' \models x\xrightarrow{R}\eta) \Rightarrow
(E'' \models x
\xrightarrow{\visZ} \eta)
\end{array}
\end{fmathpar}
Now by inversion on $H_0$ we get two cases, one of which is trivial. We
skip the formal proof for it but it is easy to see that in [LB exec]
case, ALL effects in $V$ are made visible to $\eta$, so the set is
trivially 
maximal, i.e. $H_1$ and $H_2$ yield $\bot$. For the other case (UB
exec), we get the following:
\begin{fmathpar}
\begin{array}{ll}
H_8: & V' = \left \lfloor V  \right \rfloor_{\mathtt{max}}\\
H_9: & \auxred{V'} {(\E,op_{<s,i>})} {} {(\E',\eff)}
\end{array}
\end{fmathpar}
by inversion on $H_9$ we get $H_{10}$ and from that and from $H_2$, following a similar argument from the proof of
theorem 1, we get $H_{11}$: 
\begin{fmathpar}
\begin{array}{ll}
H_{10}: & \visZ' = \visZ \cup V' \times \{\eta\}\\
H_{11}: & a \not\in V'
\end{array}
\end{fmathpar}
Now by denoting the interpretation of $R$ under $E''$ as $R''$,
$H_7$ can be rewritten as follows: 
\begin{fmathpar}
\begin{array}{ll}
H_{12}: &  \forall(x\in E'.A). R''(x,\eta) \Rightarrow \visZ''(x,\eta)\\
\end{array}
\end{fmathpar}
Now by inversion on $H_8$, we get the following:
\begin{fmathpar}
\begin{array}{lll}
& H_{13}: & V' \in \left \lfloor V  \right \rfloor \\
& H_{14}: & \not\exists V'' \in \left \lfloor V  \right \rfloor.
|V''|>|V'|\\
(\mathtt{from \; H_{13}}) & H_{15}: & V' \subseteq V \; \wedge \;
(\trunc{R})^{-1}_V(V') \subseteq V' \; \wedge \\ & & (\trunc{R})^{-1}_V(V')  =
(\trunc{R})^{-1}_{E.A}(V') 
\end{array}
\end{fmathpar}
Now we can destruct $R$, where we get multiple cases, only two of which
are non-trivial, ($R=\visZ$) and ($R=\trunc{R};\visZ$)
\begin {itemize}
\item{\bf Case1}($R=\visZ$):\\
$\trunc{R}=\nullR$, thus V itself satisfies the requirements in $H_{15}$
and we get that {\scriptsize $(V=\left \lfloor V  \right \rfloor_{\mathtt{max}})$} and
the following holds:
\begin{fmathpar}
\begin{array}{ll}
H_{16}: & V = V'
\end{array}
\end{fmathpar}
which results in contradiction from $H_1$ and $H_{11}$.
\\
\item{\bf Case2}($R=\trunc{R};\visZ$):\\
Since {\scriptsize $|V'\cup\{a\}|>|V'|$} we have the following:
\begin{fmathpar}
\begin{array}{ll}
H_{17}: & (V'\cup\{a\}) \not\in \left \lfloor V  \right \rfloor 
\end{array}
\end{fmathpar}
which based on the definition yields that the conditiones for holding
the above relation are not true, i.e.
\begin{fmathpar}
\begin{array}{ll}
H_{18}: &  \neg ((V'\cup\{a\}) \subseteq V \; \wedge \;
(\trunc{R})^{-1}_V(V'\cup\{a\}) \subseteq (V'\cup\{a\}) \; \wedge \\ & 
(\trunc{R})^{-1}_V(V'\cup\{a\})  =
(\trunc{R})^{-1}_{E.A}(V'\cup\{a\}))
\end{array}
\end{fmathpar}
or equally:
\begin{fmathpar}
\begin{array}{ll}
H_{19}: &   (V'\cup\{a\}) \not\subseteq V\; \vee \\
        &  (\trunc{R})^{-1}_V(V'\cup\{a\}) \not\subseteq
	(V'\cup\{a\}) \; \vee \\ & 
(\trunc{R})^{-1}_V(V'\cup\{a\})  \not=
(\trunc{R})^{-1}_{E.A}(V'\cup\{a\})
\end{array}
\end{fmathpar}
By inversion on the above, we get three cases, two of which are
trivial. The last conjunct can't hold because of $H_4$ and the first one
also contradics with $H_1$ and $H_{15}$. Thus, we are left with only one
case: 
\begin{fmathpar}
\begin{array}{ll}
H_{20}: & (\trunc{R})^{-1}_V(V'\cup\{a\}) \not\subseteq
	(V'\cup\{a\})
\end{array}
\end{fmathpar}
Now, from the second conjunct in $H_{15}$ we know that it should be the
case that: 
\begin{fmathpar}
\begin{array}{lll}
\hspace{-30mm} (\mathtt{\tiny from\; H_{15}: (\trunc{R})^{-1}_V(V')
\subseteq V' }) \quad & H_{21}: & ((\trunc{R})^{-1}_V(a) \not\subseteq
	(V'\cup\{a\})) 
\end{array}
\end{fmathpar}
The above hypothesis yields the existance of $c\not= a$ such that:
\begin{fmathpar}
\begin{array}{ll}
H_{22}: & c \in (\trunc{R})^{-1}_V(a)\\
H_{23}: & c \not\in V'
\end{array}
\end{fmathpar}
Now, by rewriting ({\scriptsize $R=\trunc{R};\visZ$}) in $H_{12}$ we get $H_{24}$, which can be
rewritten again into $H_{25}$ from the definition:
\begin{fmathpar}
\begin{array}{lrll}
H_{24}: & \forall (x\in E'.A). ((\trunc{R};\visZ)''(x,\eta) & \Rightarrow
& \visZ''(x,\eta)) \\
H_{25}: & \forall (x\in E'.A). (\exists b.\trunc{R}''(x,b) \; \wedge & & \\
& \visZ''(b,\eta) & \Rightarrow & \visZ''(x,\eta))
\end{array}
\end{fmathpar}
Now, we instantiate $H_{25}$ with $x=c$:
\begin{fmathpar}
\begin{array}{ll}
H_{26}: & \exists b.\trunc{R}''(c,b) \; \wedge \;
 \visZ''(b,\eta)  \Rightarrow  \visZ''(c,\eta)
\end{array}
\end{fmathpar}
we can replace $\trunc{R}''$ with $\trunc{R}'$ in above definition,
since from $H_3$, the only difference in interpretation under $E'$ and
$E''$ is the extra element $(a,\eta)$ in $E''.\visZ$ which does not
effect $\trunc{R}''(c,b)$:
\begin{fmathpar}
\begin{array}{ll}
H_{27}: & \exists b.\trunc{R}'(c,b) \; \wedge \;
 \visZ''(b,\eta)  \Rightarrow  \visZ''(c,\eta)
\end{array}
\end{fmathpar}
Moreover, since $c\not= a$, we can replace $\visZ''(c,\eta)$ with
$\visZ'(c,\eta)$:
\begin{fmathpar}
\begin{array}{ll}
H_{28}: & \exists b.\trunc{R}'(c,b) \; \wedge \;
 \visZ''(b,\eta)  \Rightarrow  \visZ'(c,\eta)
\end{array}
\end{fmathpar}
From $H_{15}$ and $H_{22}$ we get $H_{29}$, and $H_{30}$ also holds
trivially from $H_3$:
\begin{fmathpar}
\begin{array}{ll}
H_{29}: & \trunc{R}'(c,a) \\
H_{30}: & \visZ''(a,\eta)
\end{array}
\end{fmathpar}
which can be used in instantiation of  $H_{28}$ with $b=a$ and derive the
following:
\begin{fmathpar}
\begin{array}{ll}
H_{31}: & \visZ' (c,\eta)
\end{array}
\end{fmathpar}
However, we know -from the previously explained argument- that $H_{31}$
results in $H_{32}$, which results in contradiction with $H_{23}$.
\begin{fmathpar}
\begin{array}{ll}
H_{32}: & c \in V'
\end{array}
\end{fmathpar}
\fcolorbox{red}{white}{\color{black} \footnotesize QED.}

\end{itemize}
\end{footnotesize}


\subsection{Proof of Theorem \ref{theorem:three}}
\label{app:proof3}
\begin{footnotesize}
\vspace{-2mm}\rule{\textwidth}{1pt}\\ \vspace{-4mm} \\
Before proving the theorem, we first present and prove a useful lemma
and then we will present 
a new definition, regarding sets of effects.
\begin{lemma}
Under an execution state E and for a given set $S \subseteq E.A$, if
$S$ is $\psi$-consistent under $E$, then $\forall(x\in S).R_S^{-1}(x)
\subseteq S$ under $E$.
\end{lemma}
\begin{proof}
\begin{fmathpar}
\begin{array}{ll}
H_0: & S is \psi\mathtt{-consistent} \\
G_0: & \forall(x\in S). R_{S}^{-1}(x) \subseteq S \\
\end{array}
\end{fmathpar}
after intros:
\begin{fmathpar}
\begin{array}{ll}
H_1: & x \in S\\
G_1: & R_{S}^{-1}(x) \subseteq S \\
\end{array}
\end{fmathpar}
inversion on $H_0$ gives the following:
\begin{fmathpar}
\begin{array}{ll}
H_2: & \forall (\eta \in S).\forall(a\in E.A). R(a,\eta) \Rightarrow a
\in S \\
\end{array}
\end{fmathpar}
which can be rewritten to:
\begin{fmathpar}
\begin{array}{ll}
H_3: & \forall (\eta \in S). R^{-1}(\eta) \subseteq S\\
\end{array}
\end{fmathpar}
however, since $S\subseteq E.A$ then\footnote{\scriptsize we skip the formal proof
of this claim, however, since the only difference in the definitions of
$R^{-1}$ and $R^{-1}_S$ is the extra requirement about mid-level
effects in the latter, it should be a subset of the former.}:
\begin{fmathpar}
\begin{array}{ll}
H_4: & \forall (a\in E.A).R^{-1}_S(a) \subseteq R^{-1}(a)  \\
\end{array}
\end{fmathpar}
Now we can instantiate $H_3$ and $H_4$ into:
\begin{fmathpar}
\begin{array}{ll}
H_5: & R^{-1}(x) \subseteq S \\
H_6: & R^{-1}_S(x) \subseteq R^{-1}(x) 
\end{array}
\end{fmathpar}
which trivially yields $G_1$ and the proof is completed.\\
\fcolorbox{red}{white}{\color{black} \footnotesize QED.}
\vspace{2mm}
\hrule
\end{proof}
\label {lemma1}
\end{footnotesize}
Now, in order to prove the theorem, we define $V=S$, so now we need to find $E''$ and $\eta'$, such that the following
hy
potheses and goal hold.
\begin{smathpar}
\begin{array}{ll}
H_0: & \auxred{S} {(\E,op_{<s,i>})} {} {(\E',\eff)}   \\
H_1: & \mathtt{S \cup \{\eta\} \; is \;} \psi \mathtt{-consistent \;
under \;} E'   \\
G_0: & ((\E,op_{<s,i>})\;\xrightarrow{S}\;(\E'',\eff'))
\end{array}
\end{smathpar}

\paragraph{Case I: $vis \subseteq r_k $}
We define $S'$ to be the maximal closed subset of $S$, i.e. $S' = \left
\lfloor S \right \rfloor_S $. Now we can $\eta'=F_{op}(S')$.
Moreover, in order to define $E''$, we first define the following: 
\begin{smathpar}
\begin{array}{lll}

A'' & = & E.A \cup \{\eta'\}  \\
so'' & = & E.so \cup \{(\eta'',\eta')|\eta''\in E.A\} \\
vis'' & = & E.vis \cup S'\times \{\eta'\}\\
\end{array}
\end{smathpar}
Finally, we define $E''=(A'',vis'',so'')$, and now by applying
\rulelabel{OPER} rule, we will have the following:
\begin{smathpar}
\begin{array}{ll}
H_2: &  \auxred {S'} {((\EffSoup,\visZ,\soZ), op_{<s,i>}))}
    {} {((\EffSoup'',\visZ'',\soZ''),\eta')}
\end{array}
\end{smathpar}
Now, by applying \rulelabel{exec} rule on $G_0$, we have the following
new goals:
\begin{smathpar}
\begin{array}{ll}
G_0: & S \subseteq E.A  \\
G_1: & R_S^{-1}(\eta) \subseteq S \\
G_2: & vis \subseteq r_k \\
G_3: & S' = S' = \left \lfloor S \right \rfloor_S \\
G_4: & \auxred{S'} {(\E,op_{<s,i>})} {} {(\E',\eff)} \\
\end{array}
\end{smathpar}
All the goals above are an already shown assumption but $G_1$, which is
the direct result of applying lemma \ref{lemma1} on $H_1$.
\paragraph{Case II: $vis \not\subseteq r_k $}
We pick $E'' = E'$ and $\eta'=\eta$, so by applying the
\rulelabel{Exec} we have the following:
\begin{smathpar}
\begin{array}{ll}
G_0: & S \subseteq E.A  \\
G_1: & R_S^{-1}(\eta) \subseteq S \\
G_2: & vis \not\subseteq r_k \\
G_3: & \auxred{S} {(\E,op_{<s,i>})} {} {(\E',\eff)} \\
\end{array}
\end{smathpar}
$G_1$ is the direct result of lemma \ref{lemma1}, and the rest of the goals
are in fact in the  assumptions, and thus the theorem is proved. 








\section{Operational Semantics of the Augmented algorithm}
\label{appendix:large_semantics}
Here, we explain our detailed operational semantics, to maintain
multi-consistent replicated stores. We assume a given function from
operation names, to consistency contracts: $\Psi :  op \mapsto \psi $
and for simplicity reasons (again, it can be easily generalized) we
consider contracts made by a single prop:
\begin{smathpar}
\Psi(op)=\forall (a,b). a \xrightarrow{R_{op}} \Rightarrow a \xrightarrow{vis} b. 
\end{smathpar}
For a given realtion $R$ we also define $R[m]$ to refer to the m'th
relation seed in $R$:
\begin{smathpar}
(r_1;r_2;...;r_m;...;r_k)[m] = r_m
\end{smathpar}
Each replica in this semantics, maintains a $\Pool$ of available
effects, and a $\Cache$ of filtered effects for each operation, each of
which is a subset of $\Pool$ that is closed under its associated
contract, i.e. $\forall \eff \in \Cache(op). (\trunc{R_{op}})_{\Pool}^{-1}(\eff)
\subseteq \Cache(op) $ 
We also define $\Avail$ of effects 
which is  maintained according to section
\ref{sec:alg}.
Following is the formal definitions and the operation semantics.
\begin{figure*}[t]
\raggedright
%
\textbf{Auxiliary Definitions}\\
%
\begin{minipage}{\columnwidth}
\begin{smathpar}
\stretcharraybig
\begin{array}{lclcl}
  \multicolumn{5}{c}{
    {\delta} \in \mathtt{Replicated\; Data\; Type} \spc\spc
    {v} \in \mathtt{Value}\spc\spc
    {op} \in \mathtt{Operation\; Name}
  }\\
  \multicolumn{5}{c}{
    {s} \in \mathtt{Session\; Id} \spc\spc
    {i} \in \mathtt{Effect\; Id} \spc\spc
    {r} \in \mathtt{Replica\; Id}
  }\\
  \eff & \in & \mathtt{Effect} & \coloneqq &  (s,i,op,v)\\
  F_{op} & \in & \mathtt{Op.\,Def.} & \coloneqq & \delta \rightarrow \eta\\
  \EffSoup & \in & \mathtt{Eff\,Soup}	  & \coloneqq & \set{\eff} \\
  \visZ,\soZ &	\in & \mathtt{Relations} & \coloneqq & \set{(\eff,\eff)} \\
  {\E} 		& \in & \mathtt{Exec\;State}  & \coloneqq & \Exec \\
  \Theta  & \in & \mathtt{Store}      & \coloneqq & r \mapsto (\delta,\set{\eff}) \\
  {\sigma} 	& \in & \mathtt{Session} 					 	& \coloneqq & \cdot \ALT op::\sigma \\
  \Sigma 		& \in & \mathtt{Session\;Soup}   	 	& \coloneqq &
        \langle s, i, \sigma \rangle \pll \Sigma \ALT \emptyset \\
\end{array}
\end{smathpar}
\end{minipage}
%

\begin{smathpar}
\begin{array}{c}
\ssn(s,\_,\_,\_) = s \spc\spc
\id(\_,j,\_,\_) = j \spc\spc
\oper(\_,\_,op,\_) = op \spc\spc
\rval(\_,\_,\_,n) = n\\
\end{array}
\end{smathpar}

\vspace{5mm}
\textbf{Auxiliary Reduction} \;
  \fbox{\(\auxred{\Theta} {(\E,\langle s,i,op \rangle)} {r} {(\E',\eff)}\)}\\

\begin{minipage}{\textwidth}
\rulelabel{Oper}
\begin{smathpar}
\stretcharraybig
\begin{array}{l}
\RuleTwo
{
\Theta(r \mapsto (v,S)) \qquad
F_{op}(v) = \eta \qquad
\ssn(\eta) = s \qquad 
\id(\eta) = i \qquad
%\{\eff'\} = \EffSoup_{({\sf SessID}=s,\,{\sf SeqNo}=i-1)}\\
\EffSoup' = \EffSoup \cup \{\eff\} \\
\visZ' = \visZ \cup S \times\{\eff\}\qquad
\soZ' = \soZ \cup \{(\eta',\eta) \,|\, \eta'\in \EffSoup \conj 
    \ssn(\eta')=s \conj \id(\eta')<i\}\qquad
%\soZ' = (\soZ^{-1}(\eff') \cup \eff') \times\{\eff\} \cup \soZ
}
{
  \auxred {\Theta} {((\EffSoup,\visZ,\soZ), \langle s,i,op \rangle}
  {r} {((\EffSoup',\visZ',\soZ'),\eta)}
}
\end{array}
\end{smathpar}
\end{minipage}


\vspace{5mm}
\textbf{Operational Semantics} \;
  \fbox{\((\E,\Theta,\Sigma) \;\xrightarrow{\eff}\; (\E',\Theta',\Sigma')\)}\\

\begin{minipage}{3in}
\rulelabel{EffVis}
\begin{smathpar}
\stretcharraybig
\begin{array}{l}
\RuleTwo
{
  \eta \in \E.\EffSoup \spc
  \Theta(r) = (v,S) \spc
  \eta \not\in S \\
  {\sf deps}_{\psi}(\eta) \subseteq  S \spc
  \Theta' = \Theta[r \mapsto (apply \;\eta \;v, \{\eta\} \cup S)]\\
}
{
  (\E,\Theta,\Sigma) \;\xrightarrow{\eff}\; (\E, \Theta', \Sigma)
}
\end{array}
\end{smathpar}
\end{minipage}
\begin{minipage}{2.3in}
\rulelabel{Exec}
\begin{smathpar}
\stretcharraybig
\begin{array}{l}
\RuleTwo
{
  \auxred{\Theta} {(\E,\langle s,i,op \rangle)} {r} {(\E',\eta)} \\
  \Theta(r) = (v,S) \spc {\sf deps}_{\psi}(\eta) \subseteq   S \spc
}
{
  (\E,\Theta,\langle s,i,op::\sigma \rangle \pll \Sigma) 
    \;\xrightarrow{\eff}\;
  (\E',\Theta,\langle s,i+1,\sigma \rangle \pll \Sigma) 
}
\end{array}
\end{smathpar}
\end{minipage}


\caption{Operational semantics of a replicated data store.}
\label{sem:oper}
\end{figure*}









%Text of appendix \ldots

\end{document}
