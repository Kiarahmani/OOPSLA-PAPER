Here, we present the detailed proofs of the theorems of the paper. 
Let's first present a useful lemma:
\begin{lemma}
For all relations $R$ and execution steps:
\begin{fmathpar}
(\E,op_{<s,i>}) \;\xrightarrow{V}\; (\E',\eff)
\end{fmathpar}
interpretatin of $R$ under $E$ and $E'$ only differs considering $\eta$,
i.e.  {\scriptsize $a,b\not= \eta \Rightarrow (R'(a,b) \Leftrightarrow
R(a,b))$}
\label{lemma1}
\end{lemma}
\begin{proof}
We only prove $\Rightarrow$, the other part can be done
similarly. We have the following goal and hypotheses: 
\begin{fmathpar}
\begin{array}{ll}
H_0: & (\E,op_{<s,i>}) \;\xrightarrow{V}\; (\E',\eff) \\
H_1: & a,b \not= \eta\\ 
H_2: & R' (a,b)\\
G_0: & R (a,b)
\end{array}
\end{fmathpar}
Now by destructing $R$ we have the followings from new hypothesis and
goal:
\begin{fmathpar}
\begin{array}{ll}
H_3: & (\trunc{R};r)' (a,b) \\
G_1: & (\trunc{R};r) (a,b)
\end{array}
\end{fmathpar}
which can be rewritten by the definition to  get that y exists s.t.
\begin{fmathpar}
\begin{array}{ll}
H_4: & (\trunc{R})'(a,y)\\
H_5: & r'(y,b) \\
G_1: & \exists x. (\trunc{R})(a,x) \wedge (r)(x,b)
\end{array}
\end{fmathpar}
Now we instantiate the goal with y itself and by using induction on the
length of R, the first conjunct is proved, and we are left with the
following:
\begin{fmathpar}
\begin{array}{ll}
H_6: & r'(y,b) \\
G_1: & r (y,b)
\end{array}
\end{fmathpar}
Now by inversion on $H_0$ we get two cases, at both of which the
following can be derived. (In one case V should be replaced by V' but
has no effect on the proof):
\begin{fmathpar}
\begin{array}{ll}
H_7: & \visZ' = \visZ \cup V \times \{\eta\} \\
H_8: & \soZ' = \soZ' = \soZ \cup \{(\eta',\eta) \,|\, \eta'\in \EffSoup_{({\sf
SessID}=s)} 
\end{array}
\end{fmathpar}
Now, because of $H_1$ (and the fact that $y\not= \eta$) it is easy to get the following from $H_7$ and
$H_8$:
\begin{fmathpar}
\begin{array}{ll}
H_9: & \visZ(y,b) \Rightarrow \visZ(y,b) = \\
H_{10}: & \soZ(y,b) \Rightarrow \soZ(y,b) =
\end{array}
\end{fmathpar}
Which directly prove the goal, after destructing $r$.
\end{proof}








\subsection{Proof of Theorem \ref{theorem:one}}
\label{app:proof1}
We have the following two hypotheses and the goal:
\begin{smathpar}
\begin{array}{ll}
H_0: & (\E,op_{<s,i>}) \;\xrightarrow{V}\; (\E',\eff)  
\\
H_1: & V \; \mathtt{ is }\; \psi\mathtt{-consistent \; under \; \E} \\
G_0: & V \cup \{\eta\} \; \mathtt{is} \; \psi\mathtt{-consistent \;
under \; \E'}
\end{array}
\end{smathpar}
Based on the definition of consistency, we should prove the following:
\begin{smathpar}
\begin{array}{ll}
G_1: & \forall(x\in V\cup\{\eta\}).\forall (a\in E'.A). R(a,x) \Rightarrow a \in V\cup\{\eta\}
\end{array}
\end{smathpar}
By intro (and by the fact that $\E'.\EffSoup = \E.\EffSoup \cup \{\eff\}$) we will have the following new hypotheses and goal:
\begin{smathpar}
\begin{array}{ll}
H_2: & x \in V\cup \{\eta\} 
\\
H_3: & a \in E.A \cup \{\eff\}\\
H_4: & R(a,x) \\
G2: & a \in V \cup \{\eta\}
\end{array}
\end{smathpar}
By destructing $H_2$ we have two cases:
\\{ (i)} $\; x \in V$: Since $V$ is $\psi$-consistent, from $H_4$ we have $a \in V$ which proves the goal $G_2$ 
\\{ (ii)} $x = \eta$: By replacing $x$ with $\eta$ in $H_4$ we will have the following:
\begin{smathpar}
\begin{array}{ll}
H_5: & R(a,\eta) \\
\end{array}
\end{smathpar}
Based on definitions, we will have the following:
\begin{smathpar}
\begin{array}{ll}
H_6: & a \in R_{E.A}^{-1}(\eta) \\
\end{array}
\end{smathpar}
We can now assume $R=r_1;r_2;...;r_k$, and derive from $H_6$ that there exists $c$ such that:
\begin{smathpar}
\begin{array}{ll}
H_7: & c \in r_k^{-1}(\eta) \\
H_8: & a \in (r_1;...;r_{k-1})_{E.A}^{-1}(c)  \\
H_9: & r_k^{-1}(\eta) \subseteq E.A\\
\end{array}
\end{smathpar}
Now by inversion on $H_0$ we will have the following:
\begin{smathpar}
\begin{array}{ll}
H_{10}: & R_V^{-1}(\eta) \subseteq V\\
\end{array}
\end{smathpar}
which by definition will result:
\begin{smathpar}
\begin{array}{ll}
H_{11}: & r_k^{-1}(\eta) \subseteq V\\
\end{array}
\end{smathpar}
Now from $H_7$, $H_8$ and $H_{11}$ we can derive: 
\begin{smathpar}
\begin{array}{ll}
H_{12}: & a \in R_V^{-1}(\eta)\\
\end{array}
\end{smathpar}
and from $H_{10}$ and $H_{12}$ we have $a \in V$ which proves the final
goal $G_2$. \\QED.


\subsection{Proof of Theorem \ref{theorem:two}}
\label{app:proof2}
\begin{footnotesize}
\vspace{-2mm}\rule{\textwidth}{1pt}\\ \vspace{0mm} \\
We prove the theorem by contradiction:
\begin{fmathpar}
\begin{array}{ll}
H_0: & (\E,op_{<s,i>}) \;\xrightarrow{V}\; (\E',\eff) \\
H_1: & a \in V \\ 
H_2: & (a,\eta) \not\in E'.\visZ\\
H_3: & (E'.A,E'.\visZ \cup \{(a,\eta)\},E'.\soZ) \models  \psi[\eta/\hat{\eta}]\\
H_4: & (\trunc{R}^{-1}_V(a) = \trunc{R}^{-1}_{E.A}(a)) \\
G_0: & \bot
\end{array}
\end{fmathpar}
Now we call {\scriptsize $(E'.A,E'.\visZ \cup \{(a,\eta)\},E'.\soZ)$} as
{\scriptsize $E''$} and derive
the following from $H_3$:
\begin{fmathpar}
\begin{array}{ll}
H_5: & E'' \models \forall x. x\xrightarrow{R}\eta \Rightarrow x
\xrightarrow{\visZ} \eta
\end{array}
\end{fmathpar}
because $E''$ defines the universe of quantification (and since
$E''.A=E'.A$), we get the
following:
\begin{fmathpar}
\begin{array}{ll}
H_6: & \forall (x\in E'.A). E'' \models x\xrightarrow{R}\eta \Rightarrow x
\xrightarrow{\visZ} \eta
\end{array}
\end{fmathpar}
and is rewritten as the following:
\begin{fmathpar}
\begin{array}{ll}
H_7: & \forall (x\in E'.A). (E'' \models x\xrightarrow{R}\eta) \Rightarrow
(E'' \models x
\xrightarrow{\visZ} \eta)
\end{array}
\end{fmathpar}
Now by inversion on $H_0$ we get two cases, one of which is trivial. We
skip the formal proof for it but it is easy to see that in [LB exec]
case, ALL effects in $V$ are made visible to $\eta$, so the set is
trivially 
maximal, i.e. $H_1$ and $H_2$ yield $\bot$. For the other case (UB
exec), we get the following:
\begin{fmathpar}
\begin{array}{ll}
H_8: & V' = \left \lfloor V  \right \rfloor_{\mathtt{max}}\\
H_9: & \auxred{V'} {(\E,op_{<s,i>})} {} {(\E',\eff)}
\end{array}
\end{fmathpar}
by inversion on $H_9$ we get $H_{10}$ and from that and from $H_2$, following a similar argument from the proof of
theorem 1, we get $H_{11}$: 
\begin{fmathpar}
\begin{array}{ll}
H_{10}: & \visZ' = \visZ \cup V' \times \{\eta\}\\
H_{11}: & a \not\in V'
\end{array}
\end{fmathpar}
Now by denoting the interpretation of $R$ under $E''$ as $R''$,
$H_7$ can be rewritten as follows: 
\begin{fmathpar}
\begin{array}{ll}
H_{12}: &  \forall(x\in E'.A). R''(x,\eta) \Rightarrow \visZ''(x,\eta)\\
\end{array}
\end{fmathpar}
Now by inversion on $H_8$, we get the following:
\begin{fmathpar}
\begin{array}{lll}
& H_{13}: & V' \in \left \lfloor V  \right \rfloor \\
& H_{14}: & \not\exists V'' \in \left \lfloor V  \right \rfloor.
|V''|>|V'|\\
(\mathtt{from \; H_{13}}) & H_{15}: & V' \subseteq V \; \wedge \;
(\trunc{R})^{-1}_V(V') \subseteq V' \; \wedge \\ & & (\trunc{R})^{-1}_V(V')  =
(\trunc{R})^{-1}_{E.A}(V') 
\end{array}
\end{fmathpar}
Now we can destruct $R$, where we get multiple cases, only two of which
are non-trivial, ($R=\visZ$) and ($R=\trunc{R};\visZ$)
\begin {itemize}
\item{\bf Case1}($R=\visZ$):\\
$\trunc{R}=\nullR$, thus V itself satisfies the requirements in $H_{15}$
and we get that {\scriptsize $(V=\left \lfloor V  \right \rfloor_{\mathtt{max}})$} and
the following holds:
\begin{fmathpar}
\begin{array}{ll}
H_{16}: & V = V'
\end{array}
\end{fmathpar}
which results in contradiction from $H_1$ and $H_{11}$.
\\
\item{\bf Case2}($R=\trunc{R};\visZ$):\\
Since {\scriptsize $|V'\cup\{a\}|>|V'|$} we have the following:
\begin{fmathpar}
\begin{array}{ll}
H_{17}: & (V'\cup\{a\}) \not\in \left \lfloor V  \right \rfloor 
\end{array}
\end{fmathpar}
which based on the definition yields that the conditiones for holding
the above relation are not true, i.e.
\begin{fmathpar}
\begin{array}{ll}
H_{18}: &  \neg ((V'\cup\{a\}) \subseteq V \; \wedge \;
(\trunc{R})^{-1}_V(V'\cup\{a\}) \subseteq (V'\cup\{a\}) \; \wedge \\ & 
(\trunc{R})^{-1}_V(V'\cup\{a\})  =
(\trunc{R})^{-1}_{E.A}(V'\cup\{a\}))
\end{array}
\end{fmathpar}
or equally:
\begin{fmathpar}
\begin{array}{ll}
H_{19}: &   (V'\cup\{a\}) \not\subseteq V\; \vee \\
        &  (\trunc{R})^{-1}_V(V'\cup\{a\}) \not\subseteq
	(V'\cup\{a\}) \; \vee \\ & 
(\trunc{R})^{-1}_V(V'\cup\{a\})  \not=
(\trunc{R})^{-1}_{E.A}(V'\cup\{a\})
\end{array}
\end{fmathpar}
By inversion on the above, we get three cases, two of which are
trivial. The last conjunct can't hold because of $H_4$ and the first one
also contradics with $H_1$ and $H_{15}$. Thus, we are left with only one
case: 
\begin{fmathpar}
\begin{array}{ll}
H_{20}: & (\trunc{R})^{-1}_V(V'\cup\{a\}) \not\subseteq
	(V'\cup\{a\})
\end{array}
\end{fmathpar}
Now, from the second conjunct in $H_{15}$ we know that it should be the
case that: 
\begin{fmathpar}
\begin{array}{lll}
\hspace{-30mm} (\mathtt{\tiny from\; H_{15}: (\trunc{R})^{-1}_V(V')
\subseteq V' }) \quad & H_{21}: & ((\trunc{R})^{-1}_V(a) \not\subseteq
	(V'\cup\{a\})) 
\end{array}
\end{fmathpar}
The above hypothesis yields the existance of $c\not= a$ such that:
\begin{fmathpar}
\begin{array}{ll}
H_{22}: & c \in (\trunc{R})^{-1}_V(a)\\
H_{23}: & c \not\in V'
\end{array}
\end{fmathpar}
Now, by rewriting ({\scriptsize $R=\trunc{R};\visZ$}) in $H_{12}$ we get $H_{24}$, which can be
rewritten again into $H_{25}$ from the definition:
\begin{fmathpar}
\begin{array}{lrll}
H_{24}: & \forall (x\in E'.A). ((\trunc{R};\visZ)''(x,\eta) & \Rightarrow
& \visZ''(x,\eta)) \\
H_{25}: & \forall (x\in E'.A). (\exists b.\trunc{R}''(x,b) \; \wedge & & \\
& \visZ''(b,\eta) & \Rightarrow & \visZ''(x,\eta))
\end{array}
\end{fmathpar}
Now, we instantiate $H_{25}$ with $x=c$:
\begin{fmathpar}
\begin{array}{ll}
H_{26}: & \exists b.\trunc{R}''(c,b) \; \wedge \;
 \visZ''(b,\eta)  \Rightarrow  \visZ''(c,\eta)
\end{array}
\end{fmathpar}
we can replace $\trunc{R}''$ with $\trunc{R}'$ in above definition,
since from $H_3$, the only difference in interpretation under $E'$ and
$E''$ is the extra element $(a,\eta)$ in $E''.\visZ$ which does not
effect $\trunc{R}''(c,b)$:
\begin{fmathpar}
\begin{array}{ll}
H_{27}: & \exists b.\trunc{R}'(c,b) \; \wedge \;
 \visZ''(b,\eta)  \Rightarrow  \visZ''(c,\eta)
\end{array}
\end{fmathpar}
Moreover, since $c\not= a$, we can replace $\visZ''(c,\eta)$ with
$\visZ'(c,\eta)$:
\begin{fmathpar}
\begin{array}{ll}
H_{28}: & \exists b.\trunc{R}'(c,b) \; \wedge \;
 \visZ''(b,\eta)  \Rightarrow  \visZ'(c,\eta)
\end{array}
\end{fmathpar}
From $H_{15}$ and $H_{22}$ we get $H_{29}$, and $H_{30}$ also holds
trivially from $H_3$:
\begin{fmathpar}
\begin{array}{ll}
H_{29}: & \trunc{R}'(c,a) \\
H_{30}: & \visZ''(a,\eta)
\end{array}
\end{fmathpar}
which can be used in instantiation of  $H_{28}$ with $b=a$ and derive the
following:
\begin{fmathpar}
\begin{array}{ll}
H_{31}: & \visZ' (c,\eta)
\end{array}
\end{fmathpar}
However, we know -from the previously explained argument- that $H_{31}$
results in $H_{32}$, which results in contradiction with $H_{23}$.
\begin{fmathpar}
\begin{array}{ll}
H_{32}: & c \in V'
\end{array}
\end{fmathpar}
\fcolorbox{red}{white}{\color{black} \footnotesize QED.}

\end{itemize}
\end{footnotesize}


\subsection{Proof of Theorem \ref{theorem:three}}
\label{app:proof3}
Before proving the theorem, we first present and prove a useful lemma
and then we will present 
a new definition, regarding sets of effects.
\begin{lemma}
Under an execution state E and for a given set $S \subseteq E.A$, if
$S$ is $\psi$-consistent under $E$, then $\forall(x\in S).R_S^{-1}(x)
\subseteq S$ under $E$.
\begin{proof}
From $\psi$-consistency of $S$ we have the following:
\begin{smathpar}
\begin{array}{ll}
H_0: & \forall(x\in S). \forall (y\in E.A). R(y,x) \in S \\
\end{array}
\end{smathpar}
Now based on the definition of $R$, we have the following: 
\begin{smathpar}
\begin{array}{ll}
H_1: & \forall(x\in S). R_{E.A}^{-1}(x) \subseteq S \\
\end{array}
\end{smathpar}
However, since $S\subseteq E.A$ we have $R_{S}^{-1}(x) \subseteq
R_{E.A}^{-1}(x)$ and the following will be derived: 
\begin{smathpar}
\begin{array}{ll}
H_2: & \forall(x\in S). R_{S}^{-1}(x) \subseteq S \\
\end{array}
\end{smathpar}
which completes the proof of the lemma.
\\ QED.
\end{proof}
\end{lemma}
\label {lemma1}
Now, in order to prove the theorem, we define $V=S$, so now we need to find $E''$ and $\eta'$, such that the following
hy
potheses and goal hold.
\begin{smathpar}
\begin{array}{ll}
H_0: & \auxred{S} {(\E,op_{<s,i>})} {} {(\E',\eff)}   \\
H_1: & \mathtt{S \cup \{\eta\} \; is \;} \psi \mathtt{-consistent \;
under \;} E'   \\
G_0: & ((\E,op_{<s,i>})\;\xrightarrow{S}\;(\E'',\eff'))
\end{array}
\end{smathpar}

\paragraph{Case I: $vis \subseteq r_k $}
We define $S'$ to be the maximal closed subset of $S$, i.e. $S' = \left
\lfloor S \right \rfloor_S $. Now we can $\eta'=F_{op}(S')$.
Moreover, in order to define $E''$, we first define the following: 
\begin{smathpar}
\begin{array}{lll}

A'' & = & E.A \cup \{\eta'\}  \\
so'' & = & E.so \cup \{(\eta'',\eta')|\eta''\in E.A\} \\
vis'' & = & E.vis \cup S'\times \{\eta'\}\\
\end{array}
\end{smathpar}
Finally, we define $E''=(A'',vis'',so'')$, and now by applying
\rulelabel{OPER} rule, we will have the following:
\begin{smathpar}
\begin{array}{ll}
H_2: &  \auxred {S'} {((\EffSoup,\visZ,\soZ), op_{<s,i>}))}
    {} {((\EffSoup'',\visZ'',\soZ''),\eta')}
\end{array}
\end{smathpar}
Now, by applying \rulelabel{exec} rule on $G_0$, we have the following
new goals:
\begin{smathpar}
\begin{array}{ll}
G_0: & S \subseteq E.A  \\
G_1: & R_S^{-1}(\eta) \subseteq S \\
G_2: & vis \subseteq r_k \\
G_3: & S' = S' = \left \lfloor S \right \rfloor_S \\
G_4: & \auxred{S'} {(\E,op_{<s,i>})} {} {(\E',\eff)} \\
\end{array}
\end{smathpar}
All the goals above are an already shown assumption but $G_1$, which is
the direct result of applying lemma \ref{lemma1} on $H_1$.
\paragraph{Case II: $vis \not\subseteq r_k $}
We pick $E'' = E'$ and $\eta'=\eta$, so by applying the
\rulelabel{Exec} we have the following:
\begin{smathpar}
\begin{array}{ll}
G_0: & S \subseteq E.A  \\
G_1: & R_S^{-1}(\eta) \subseteq S \\
G_2: & vis \not\subseteq r_k \\
G_3: & \auxred{S} {(\E,op_{<s,i>})} {} {(\E',\eff)} \\
\end{array}
\end{smathpar}
$G_1$ is the direct result of lemma \ref{lemma1}, and the rest of the goals
are in fact in the  assumptions, and thus the theorem is proved. 





