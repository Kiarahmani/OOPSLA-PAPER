\begin{figure}[h]
	\centering
	\begin{subfigure}[t]{0.5\textwidth}
	\begin{lstlisting}
data Sess = Bob | Alice
type ID = (Sess,Int) 
type Effect= (ID,String)
type State = (String,Int,Int)
	
read :: ID -> State -> String
read (sess,seq) (st,sq1,sq2) = 
	case sess (*@\textcolor{blue}{of}@*) 
		Bob ->   if (seq==sq1+1) (*@\textcolor{blue}{then}@*) st
		         else read (sess,seq)(st,sq1,sq2)
		Alice -> if (seq==sq2+1) (*@\textcolor{blue}{then}@*) st
		         else read (sess,seq)(st,sq1,sq2)
	\end{lstlisting}		  
	\end{subfigure}
	%
	\hfill
        %
	\begin{subfigure}[t]{0.42\textwidth}
	\begin{lstlisting}[firstnumber=13]
	apply :: State -> Effect -> State 
	apply (st,sq1,sq2) ((sess,seq),cm) = 
	  case sess (*@\textcolor{blue}{of}@*) 
	    Bob ->   if (sq1==seq-1)
	             (*@\textcolor{blue}{then}@*) (st++cm,sq1+1,sq2)
	             else (st,sq1,sq2)
	    Alice -> if (sq2==seq-1)
	             (*@\textcolor{blue}{then}@*) (st++cm,sq1,sq2+1)
	             else (st,sq1,sq2)
	\end{lstlisting}		  
        \end{subfigure}

	\hrulefill
	\caption*{The guarded application to prevent the lost-updates
	anomaly,
	assuming 2 known clients Bob and Alice. Even with
	2 known clients, the application has become much more complex
	with changed logic. This
	is much worse in reality where clients constantly join and leave the
	system}
	\label{fig:modified_code}
\end{figure}


