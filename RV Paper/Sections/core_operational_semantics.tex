Before intorducing the operational semantics, 
we need to formally define the inverse of a relation, \emph{given a set
of local effects $\V$}.
In correspondence with real
distributed replicas, where at any moment there is only a \emph{subset}
of all produced effects present at a replica, we want to define
the inverse of a relation,
only if the necessary information is present. For example, assume a
replica wants to compute the set $D$ of all effects satisfying
$\xrightarrow{r_1;r_2}$
relation to an effect $\eff$. This is done in two steps, by first finding
the set $D'$ of all effects satisfying relation $\xrightarrow{r_2}$ to
$\eff$, and only then finding effects satisfying $\xrightarrow{r_1}$
relation to a member of set $D'$. We should formalize our definition of the
inverse relation, so that it captures the fact that the replica cannot
compute $D$, if some effects in $D'$ are not locally available yet.
In order to do so, we first formally define the inverse of a seed relation
$r$ for a given set of effects $S$:
\begin{smathpar}
\rel^{-1}(\Set) = 
\begin{cases}
\begin{array}{lll}
\bigcup^{}_{\eta\in \Set}. \{\eta'|(\eta',\eta) \in \E.\rel \} & \myif
&\rel\in\{\soZ,\visZ\} \\ 
\rel_1^{-1}(\Set)\cup \rel_2^{-1}(\Set) & \myif & \rel=\rel_1\cup \rel_2
%\\G_{r}(S,\emptyset) & if &  r=r^* 
\end{array}
\end{cases}
\end{smathpar}
Now, using a helping recursive function $G_r$, we extend the above
definition to also cover the closure of a seed relation:
\begin{smathpar}
(r^{*})^{-1}(S) = G_{r}(S,\emptyset) 
\end{smathpar}
where,
\begin{smathpar}
G_r(S,P) =
\begin{cases}
\begin{array} {lll}
G_r(r^{-1}(S) , P\cup r^{-1}(S)) &if& r^{-1}(S) \neq \emptyset  \\
P  & &  otherwise
\end{array}
\end{cases}
\end{smathpar}
Now we are ready to formally define the inverse of the given relation
$R=r_1;r_2;...;r_k$, according to a set $\V$ of local effects. This is
done by logically specifying the set of effects satisfying the inverse
relation with a given effect $\eff$:
\begin{smathpar}
\begin{array}{ccc}
   \eta' \in (r_1;r_2;...;r_k)^{-1}_V (\eta) & \iff & \exists
   \eta''.(\eta''\in r_k^{-1}
   (\eta))\; \wedge \;(\eta' \in (r_1;r_2;...;r_{k-1})^{-1}
   (\eta''))\;  \wedge \; (r_k^{-1}(\eta) \subseteq V)
\end{array}
\end{smathpar}
Note that the first two conjuncts are the straightforward stepwise definition of
the closure of relations. However, the last conjunct is also added, to
make sure that before going any further in recursive computation, $V$
includes all
effects of the very next step. This is how we capture the
necessaity of the presence of mid-way effects in real world systems for
computing the final set of dependencies. 

Lastly, we finish this part, by formally defining the \emph{maximally
closed subset} of a set of effects $S$, given a relation $R$ and a set
of local effects $V$
as follows:
\begin{smathpar}
\left \lfloor S \right \rfloor_V = S' \spc \iff \spc S'
\subseteq S \; \wedge \;
R_V^{-1}(S') \subseteq S' \; \wedge \; 
\not\exists
S''.((R_V^{-1}(S''))\subseteq S''\wedge |S''|>|S'|)
\end{smathpar}
The above statement, simply requires $\left \lfloor S \right \rfloor_V $
to be closed under $R_V^{-1}$ and to be the largest among such closed
subsets.
