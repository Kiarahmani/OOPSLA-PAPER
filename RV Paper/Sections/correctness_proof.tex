\begin{footnotesize}

{\bf (Part i)} We have the following two hypotheses and the goal:
\begin{fmathpar}
\begin{array}{ll}
H_0: & (\E,op_{<s,i>}) \;\xrightarrow{V}\; (\E',\eff)  
\\
H_1: & V \; \mathtt{ is }\; \psi\mathtt{-consistent \; under \; \E} \\
\hline
G_0: & V \cup \{\eta\} \; \mathtt{is} \; \psi\mathtt{-consistent \;
under \; \E'}
\end{array}
\end{fmathpar}
Rewriting the definition in $G_0$ results in the following. We denote
the interpretation of $R$ under $E'$ as $R'$:
\begin{fmathpar}
\begin{array}{ll}
G_1: & \forall (b\in V\cup \{\eta\}).\forall (a \in E'.A). R'(a,b)
\Rightarrow a \in V \cup \{\eta\}
\end{array}
\end{fmathpar}
By intros we have: 
\begin{fmathpar}
\begin{array}{ll}
H_2: & b \in V \cup \{\eta\} \\
H_3: & a \in E'.A \\
H_4: & R' (a,b)\\
\hline
G_2: & a \in V \cup \{\eta\}
\end{array}
\end{fmathpar}
by inversion on $H_0$, there is two cases, in case one (UB reduction) we
have the 
following:
\begin{fmathpar}
\begin{array}{ll}
T_1: &  \auxred{V'} {(\E,op_{<s,i>})} {} {(\E',\eff)}\
\end{array}
\end{fmathpar}
by inversion on $T_1$ we will have the following:
\begin{fmathpar}
\begin{array}{ll}
T_2: & E'.A = E.A \cup \{\eta\}
\end{array}
\end{fmathpar}
Since the other case (LB reduction) also includes similar premises which
yields $T_2$, we can add it to the hypothesis:
\begin{fmathpar}
\begin{array}{ll}
H_5: & E'.A = E.A \cup \{\eta\}
\end{array}
\end{fmathpar}
by rewriting $H_5$ in $H_3$  and by inversion, we get two cases:
$a=\eta$ and $a \in E.A$. The first case immediatly proves $G_2$, so we
only consider the second case where we have: 
\begin{fmathpar}
\begin{array}{ll}
H_6: & a \in E.A
\end{array}
\end{fmathpar}
Now, by inversion on $H_2$, we have two cases: 
\begin{itemize}
\item {\bf Case 1:} \\
\begin{fmathpar}
\begin{array}{ll}
& b \in V 
\end{array}
\end{fmathpar}
by inversion in $H_1$ we have:
\begin{fmathpar}
\begin{array}{ll}
H_7: & \forall (x \in V). \forall (y \in E.A). R(y,x) \Rightarrow y \in
V
\end{array}
\end{fmathpar}
by instantiation with a and b: 
\begin{fmathpar}
\begin{array}{ll}
H_8: & R(a,b) \Rightarrow a \in V
\end{array}
\end{fmathpar}
Now by applying the lemma 1 on $H_4$ we get that $R(a,b)$ holds (since
$a,b \neq \eta$), which can be applied on $H_8$ to get $a \in V$ which
proves the goal $G_2$.
%
%
%
%
\item {\bf Case 2:} \\
\begin{fmathpar}
\begin{array}{lll}
& H_9: & b = \eta \\
\hspace{-35 mm} \mathtt{(by \; rewriting \;} H_9 \mathtt{\;in\; } H_4) & H_{10}: & R'(a,\eta)
\end{array}
\end{fmathpar}
Now we use inversion on $H_0$ and get two cases: (LB exec) and (UB exec)
\subitem {\bf \footnotesize SCase (LB exec):}
we have $H_{11}$ and $H_{12}$ from the reduction rule premises:
\begin{fmathpar}
\begin{array}{ll}
H_{11}: & R^{-1}_V(\eta) = R^{-1}_{E'.A}(\eta)   \\ \vspace{-2mm}\\
H_{12}: & R_V^{-1}(\eta) \subseteq V
\end{array}
\end{fmathpar}
now from $H_{10}$ we have $H_{13}$ which can be rewritten by $H_{11}$ to
get $H_{H14}$:
\begin{fmathpar}
\begin{array}{ll}
H_{13}: & a \in R^{-1}_{E'.A}(\eta)\\
H_{14}: & a \in R^{-1}_{V}(\eta)
\end{array}
\end{fmathpar}
The goal $G_2$ is now proved from $H_{12}$ and $H_{14}$.

\subitem {\bf \footnotesize SCase (UB exec):}
We have the following from the premises: 
\begin{fmathpar}
\begin{array}{ll}
H_{15}: & V' = \left \lfloor V  \right \rfloor_{\mathtt{max}}\\
H_{16}: & V' \subseteq V
\end{array}
\end{fmathpar}
now destruct $R$, the only non-trivial cases are
$(R=\trunc{R};\visZ)$ and 
($R=\visZ$):


{\bf SSCase }($R=\trunc{R};\visZ$):  \\
From $H_{10}$ we get $H_{17}$ which based on the definition, yields that
there exists $c$ such that $H_{18}$, $H_{19}$ and $H_{20}$ hold:
\begin{fmathpar}
\begin{array}{ll}
H_{17}: & a \in (\trunc{R}';\visZ')_{E'.A}^{-1}(\eta)\\
H_{18}: & c \in \visZ'^{-1}(\eta)\\
H_{19}: & a \in \trunc{R}'^{-1}(c)\\
H_{20}: & \visZ'^{-1}(\eta) \subseteq E'.A \\
\end{array}
\end{fmathpar}
from $H_{15}$ we have:
\begin{fmathpar}
\begin{array}{ll}
H_{21}: & (\trunc{R})^{-1}_V(V') \subseteq V'
\end{array}
\end{fmathpar}
Now from $H_{18}$ is straightforward to get:
\begin{fmathpar}
\begin{array}{ll}
H_{22}: & c \in V'
\end{array}
\end{fmathpar}
which after appying the lemma 1 on $H_{19}$, and by $H_{21}$ yields the following, which proves
the goal $G_2$:
\begin{fmathpar}
\begin{array}{ll}
H_{23}: & a \in V'
\end{array}
\end{fmathpar}

{\bf SSCase }($R=\visZ$): 
From $H_{10}$ we get that $\visZ' (a,\eta)$, which -with a similar
argument to the previous subcase- yields the following and the goal is
proved: 
\begin{fmathpar}
\begin{array}{ll}
H_{24}: & a \in V'
\end{array}
\end{fmathpar}
\end{itemize} 
\fcolorbox{red}{white}{\color{black} \footnotesize QED.}  \vspace {10mm} \\ 
%
%
%
%
%
%
%
{\bf (Part ii)}  \\
For this part we have the  following hypothesis and the goal:
\begin{fmathpar}
\begin{array}{ll}
H_{0}: & (\E,op_{<s,i>}) \;\xrightarrow{V}\; (\E',\eff) \\ 
G_{0}: & E' \models [\eta/\hat{\eta}]
\end{array}
\end{fmathpar}
By inversion on $H_0$, we have two cases: \\
{\footnotesize \bf Case1} (UB exec):\\
\begin{fmathpar}
\begin{array}{ll}
H_{1}: & r_k = \visZ\\ 
H_{2}: & V \subseteq E.A \\
H_{3}: & V'= \left \lfloor V  \right \rfloor_{\mathtt{max}}\\
H_{4}: & \auxred{V'} {(\E,op_{<s,i>})} {} {(\E',\eff)}\\
\end{array}
\end{fmathpar}
The goal $G_0$ can be rewritten as: 
\begin{fmathpar}
\begin{array}{ll}
G_{1}: & E' \models \forall a. a \xrightarrow{R} \eta \Rightarrow a
\xrightarrow{vis} \eta 
\end{array}
\end{fmathpar}
Since the $E'.A$ gives the interpretation for the universe of
quantification:
\begin{fmathpar}
\begin{array}{ll}
G_{2}: & \forall (a\in E'.A). E' \models a \xrightarrow{R} \eta \Rightarrow a
\xrightarrow{vis} \eta 
\end{array}
\end{fmathpar}
by intros: 
\begin{fmathpar}
\begin{array}{ll}
H_{5}: & a \in E'.A\\
G_{3}: & E' \models a \xrightarrow{R} \eta \Rightarrow a
\xrightarrow{vis} \eta 
\end{array}
\end{fmathpar}
Now since ({\scriptsize $(\mathcal{M} \models A \Rightarrow B) \Leftrightarrow
(\mathcal{M} \models A   \Rightarrow  \mathcal{M} \models B)$}) we can rewrite $G_3$ as:
\begin{fmathpar}
\begin{array}{ll}
G_{4}: & (E' \models a \xrightarrow{R} \eta) \Rightarrow (E' \models a
\xrightarrow{vis} \eta) 
\end{array}
\end{fmathpar}
intros: 
\begin{fmathpar}
\begin{array}{ll}
H_{6}: & E' \models a \xrightarrow{R} \eta \\
G_{5}: & E' \models a \xrightarrow{vis} \eta 
\end{array}
\end{fmathpar}
Now we use the interpretation given by $E'$, to rewrite the relations as
follows. Note that we denote the interpretation of $R$ under $E'$ as
$R'$ and $E.\visZ$ as $\visZ'$.
\begin{fmathpar}
\begin{array}{ll}
H_{7}: & R'(a,\eta) \\
G_{6}: & \visZ'(a,\eta)
\end{array}
\end{fmathpar}
by inversion on $H_4$:
\begin{fmathpar}
\begin{array}{ll}
H_{8}: & \visZ' = \visZ \cup V' \times \{\eta\}
\end{array}
\end{fmathpar}
Now since $\eta$ is a fresh effect, we get that $a \in V' \Rightarrow
\visZ'(a,\eta)$ which can be applied to $G_6$ to get the following:
\begin{fmathpar}
\begin{array}{ll}
G_{7}: & a \in V'
\end{array}
\end{fmathpar}
Now, destructing R yileds multiple cases, only one of which is
non-trivial:  
$\scriptsize R=\trunc{R};\visZ$, which can be rewritten in $H_7$ to get:
\begin{fmathpar}
\begin{array}{ll}
H_{9}: & (\trunc{R};\visZ)'(a,\eta)
\end{array}
\end{fmathpar}
Now we can rewrite the definition in $H_9$, and derive that there exists
$b$ such that:
\begin{fmathpar}
\begin{array}{ll}
H_{10}: & \trunc{R}'(a,b)\\
H_{11}: & \visZ'(b,\eta)\\
\end{array}
\end{fmathpar}
Now using a similar argument, from $H_8$ and $H_{11}$ we get: 
\begin{fmathpar}
\begin{array}{ll}
H_{12}: & b \in V'\\
\end{array}
\end{fmathpar}
Now by applying the lemma 1 on $H_{10}$ we get: 
\begin{fmathpar}
\begin{array}{ll}
H_{13}: & \trunc{R}(a,b)
\end{array}
\end{fmathpar}
since we have {\scriptsize $V' \in \left \lfloor V  \right \rfloor$}, 
we get the following: 
\begin{fmathpar}
\begin{array}{ll}
H_{14}: & \forall (x\in V'). (\trunc{R})_{E.A}^{-1}(V') \Rightarrow x \in V'
\end{array}
\end{fmathpar}
which yields the following from $H_{12}$ and $H_{13}$:
\begin{fmathpar}
\begin{array}{ll}
H_{15}: & a \in V'
\end{array}
\end{fmathpar}
which proves the goal $G_7$.\\ \vspace{3 mm} \\
{\footnotesize \bf Case2} (LB exec):\\
We prove this case by induction on the length of the given relation $R$.
We have the followings, from the premises of the reduction rule:
\begin{fmathpar}
\begin{array}{ll}
H_{1}: & r_k = \soZ\\ 
H_{2}: & V \subseteq E.A \\
H_{3}: & R^{-1}_V(\eta) = R^{-1}_{E.A}(\eta)\\
H_{4}: & R^{-1}_V(\eta) \subseteq V \\
H_{5}: & \auxred{V} {(\E,op_{<s,i>})} {} {(\E',\eff)}\\
\end{array}
\end{fmathpar}
Using the same argument as the previous section, we get the following
new goal and hypotheses:
\begin{fmathpar}
\begin{array}{ll}
H_{6}: & a \in E'.A\\ 
H_{7}: & R'(a,\eta) \\
G_{1}: & \visZ'(a,\eta)
\end{array}
\end{fmathpar}
We now destruct R to get $H_8$ from $H_7$, and rewrite the definition in it to get the next
two hypotheses.
Note that by destructing $R$, there are only two non-trivial cases
$R=\trunc{R};\soZ$ and $R=\soZ$, which we are only consideing the
former, since the latter can be proved similarly.
\begin{fmathpar}
\begin{array}{ll}
H_{8}: & (\trunc{R};\soZ)'(a,\eta) \\ 
H_{9}: & \trunc{R}'(a,b)\\
H_{10}: & \soZ'(b,\eta)
\end{array}
\end{fmathpar}
Now, from the previous section we know that $(so')^{-1}(\eta) \subseteq V$
which yields the following from $H_{10}$:
\begin{fmathpar}
\begin{array}{ll}
H_{11}: & b \in V
\end{array}
\end{fmathpar}
The goal is proved by the induction hypothesis, $H_9$ and $H_{11}$.
\\ \fcolorbox{red}{white}{\color{black} \footnotesize QED.}
\end{footnotesize}























