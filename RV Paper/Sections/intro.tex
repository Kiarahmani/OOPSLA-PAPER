\section{Introduction}
\label{sec:intro}

Modern web-based applications are typically implemented as multiple
agents simultaneously serving clients, operating over shared data
objects replicated across geographically distributed machines.
Historically, replication transparency (i.e. requiring distributed
systems to \emph{appear} as a single compute and storage server to
users), has been the \emph{de facto} system abstraction used to
program in these environments.  This abstraction has resulted in the
development of standardized implementation and reasoning techniques
around \emph{strongly consistent} (SC) distributed stores.  Although
strong notions of consistency, such as \emph{linearizability} and
\emph{serializability}, are ideal to develop and reasoning about
distributed applications, they come at the price of availability and
low-latency.  Extensive synchronization overhead often necessary to
realize strong consistency is unacceptable for web-scale applications
that wish to be ``always-on'' despite network partitions.  Such
applications are therefore usually designed to tolerate certain
inconsistencies, allowing them to adopt weaker notions of consistency
that impose less synchronization overhead. An extreme example is
\emph{eventual consistency} (EC), where the application responds to
user requests using just the local state of the server to which the
client connects; this state is \emph{some} subset of the global state
(i.e., it includes an unspecified subset of writes submitted to the
application in an unspecified order).  Applications that may not
tolerate the level of inconsistency imposed by EC strengthen it as
needed, resulting in various instantiations that are stronger than EC,
but weaker than SC. The term \emph{weak consistency} is a catch-all
term used to refer to such application-specific weak consistency
guarantees.

Unfortunately, the \emph{ad hoc} nature of weak consistency confounds
standardization, with different implementations defining different
mechanisms for achieving weakly consistent behavior.  Oftentimes,
implementations are closely tied to application logic, complicating
maintainability and reuse.  To illustrate, consider a web application
that stores user passwords (encrypted or otherwise) in an
off-the-shelf EC data store (e.g., Cassandra~\cite{cassandra}). The
application allows an authenticated user to change her password,
following which the current authentication expires, and the user is
required to re-login.  Now, consider the scenario shown in
Fig.~\ref{fig:rmw-anomaly} where Alice changes her password, and
subsequently tries to login with the new password. This involves a
write of a new password to the store, followed by a read during
authentication.  However, because of transient system properties
(e.g., load balancing, or network partitions), Alice's write and the
read could be served by different replicas of the store, say $R_1$ and
$R_2$ (resp.), where $R_2$ may not (yet) contain the latest writes
from $R_1$. Consequently, Alice login attempt fails, even though she
types the correct password.

To preempt the scenario described above, applications might want to
enforce a stronger consistency guarantee that ensures reads from a
client session witness previous writes from the same session. The
consistency guarantee, known as \emph{Read-My-Writes/Read-Your-Writes}
(RMW/RYW), is one of several well-understood session
guarantees~\cite{terry-pdis94}, yet the methods used for its
enforcement are often store -and application- dependent. For instance,
Oracle's replicated implementation of Berkeley DB suggests application
developers implement RMW by querying various metadata associated with
writes~\cite{oracle-ryw}.  Each successful write to the store returns
a commit token, which is then passed with the subsequent reads to help
the store identify the last write preceding the read. The read
succeeds only if the write is present at the replica serving the read,
failing which the application has to retry the read, preferably after
some delay.  Fig.~\ref{fig:rmw-oracle} illustrates this idea.

The RMW implementation described above already requires considerable
re-engineering of the application (to store and pass commit tokens for
each object accessed), and conflates application logic with concerns
orthogonal to its semantics. On stores that do not admit metadata
queries (e.g., Cassandra), the implementation is even more complicated
as we describe in Sec.~\ref{sec:motivation}. Moreover, applications
sometimes require different consistency guarantees for different
objects.  In such cases different enforcement mechanisms must be
developed, forcing developers to simultaneously reason about their
respective properties \emph{in conjunction with} the application
state. This is clearly an onerous task.  Other alternatives such as
forgoing application integrity, or resorting to strong consistency,
sacrifice correctness or availability, both unappealing options.

In this paper, we propose an alternative to the aforementioned
approaches that overcomes their weaknesses.  \tool is a lightweight
runtime system for Haskell that allows application developers to take
advantage of weak consistency without having to re-engineer their code
to accommodate consistency enforcement logic.  The key insight that
drives \tool's design is that the hardness of reasoning about the
integrity of a distributed application stems from conflating
application logic with the consistency enforcement logic, reasoning
about both \emph{operationally}.  By separating application semantics
from consistency enforcement semantics, admitting operational
reasoning for the former, and declarative reasoning for the latter
frees programmers from having to worry about implementation details of
consistency guarantees, and instead focus on reasoning about
application semantics under the assumption that specified consistency
guarantees are automatically enforced by the data store runtime.  Our
approach admits declarative reasoning for consistency enforcement via
a specification language that lets programmers formally specify the
consistency requirements of their application. The design of our
specification language is based on the observation that various forms
of weak consistency guarantees differ only in terms of how and what
they mark as \emph{dependencies} among operations.  When all
dependencies are present on a replica to an operation, then the effect
of the operation is guaranteed to be correct with respect to the
specification.  For example, RMW marks all previous writes from the
same session as dependencies of subsequent read operations, so an RMW
read succeeds only if all the previous writes are visible (i.e.,
present on the replica on which the read is performed) . A different
consistency guarantee (e.g., \emph{Monotonic Reads} (MR)) imposes a
different set of dependencies, as do various combinations of (e.g.,
MR+RMW).  By allowing a runtime system to monitor dependencies defined
by consistency specifications, we realize a generic weak consistency
enforcement mechanism framework.  Such a runtime, working in tandem
with a consistency specification language, contributes to the novelty
of our approach.


% here users send requests to a cloud of servers
% to post a message or
% to read the current messages on the board. Each user request is sent
% to an available server, which itself is working on top of an instance
% of an off-the-shelf data store (e.g. Facebook's Cassandra).  Since the
% underlying stores, usually satisfy the {\bf eventual consistency}
% model, developers are guaranteed  that every write to a local instance
% of the data store will eventually  be delivered at all other
% instances. However, most of the desirable application-level properties
% are not met under this model.          For example, assume the
% developers wants to make sure that all the $\mathtt{read}$ requests
% from a user, would necessarily include prior writes by the same user.
% This guarantee which is known as {\bf read my writes (RMW)} can be
% violated in eventually consistent stores.  
% \begin{figure}
\centering
\begin{subfigure}[b]{0.3\textwidth}
\includegraphics[scale=0.2]{../Figures/System_example.pdf}
\label{fig:rmw_falsified}
\caption{A behavior allowd in EC stores that falsifies RMW guarantee}
\end{subfigure}\quad
\begin{subfigure}[b]{0.3\textwidth}
A figure showing tokens being passed around for enforcing RMW
\label{fig:addhoc_impl}
\caption{An ad-hoc implementation of RMW consistency guarantee, using
tokens and guards}
\end{subfigure}\quad
\begin{subfigure}[b]{0.3\textwidth}
A figure showing how consistency management can be separated from the
applications and underlying stores,using shim layers (our shims are
synthesized according to the given contract)
\label{fig:shim_impl}
\caption{Consistency management shim layer that guarantees desired
levels of consistency, based on the given specifications}
\end{subfigure}~
\caption{Running applications on EC store, ad-hoc consistency
implementations, general-purpose shim layers}



\end{figure}


% Here, developers are forced to come up with their own implementation of
% RMW. They must modify client and server applications to generate, send
% and receive special tokens, to maintain the set of updates available at
% each server, and block user requests, if some necessary updates are
% still missing. Now, in order to have a correct application, the
% developer must prove certain non-trivial safety and liveness
% properties for this implementation. To make the matter worse, the
% proof will becmoe obsolete after each modification in the application or
% the consistency requirements. For example, after users reporting some unaccepatble
% behaviors, in order to disallow those executions, developers 
% implement and add another consistency guarantee to RMW; a task that
% would clearly make the original correctness proofs obsolete. 


% In this paper, we address these issues by introducing a compositional, principled approach to 
% derive enforcement mechanisms for various forms of weak consistency 
% guarantees. 
% We offer developers with a tool that generates a shim layer that works
% with an eventually consistent store and extends it to a key-value store
% with multiple environment, each of which shows a certain level of
% consistency, specified by the developers. 
% We also introduce a language for specifying application-level consistency 
% requirements as logical formulae that express legal execution states of the distributed store. 
% Our consistency enforcement methodology is independent of the underlying store 
% and only assumes the well-established guarantee of "eventual delivery". 
% Moreover, our enforcement tool is complete for the specification
% language, which itself is powerful enough to express all the known consistency 
% guarantees in the literature.

% We argue that all the consistency guarantees basically specify, \emph{when} an 
% application instance should block a user request and for arrival of 
% \emph{what} remote updates it should wait. For obvious reasons, any implementation of these 
% consistency guarantees must be \emph{correct} and \emph{optimal}. The former property states that 
% all possible behaviors of the implementation are allowed by the given specification and the later 
% ensures that application instances do not engage in 
% any unnecessary synchronization before responding to a user request (i.e. users are only blocked if 
% it is absolutely necessary).
% \\ Our technique is based on tracking relationships between update effects, and maintaining multiple 
% (logical) caches at the shim layer, each of which is enforced to satisfy a specific level of consistency, 
% derived from the given specifications. We believe that our approach is the first ever principled 
% reasoning and implementation framework for weak consistency enforcement techniques which 
% is also proven to be \emph{correct} and \emph{optimal}. 
% By separating all the consistency management procedures from the application level, we are taking 
% the non-trivial task of proving soundness criteria for ad-hoc consistency enforcement techniques 
% off of the developer's shoulders. 
% Contributions of the paper
A summary of our contributions is given below:
\begin{itemize}
  \item We propose a consistency specification language that lets
    programmers express the consistency requirements of their
    applications in terms of the dependencies between operations.

  \item We describe a generic consistency enforcement runtime that
    analyzes an operation's consistency specification, and ensures
    that its dependencies are satisfied before it is executed. We
    formalize the operational semantics of the runtime, and prove its
    correctness and optimality guarantees. Optimality includes
    \emph{minimum wait}, which guarantees that an operation waits (on
    arriving communication) only until its dependencies are satisfied,
    and \emph{minimum staleness}, which guarantees that among various
    states that satisfy an operation's dependencies, operation
    witnesses the latest state.
  
  \item We describe an implementation of our specification language and
    consistency enforcement runtime in a tool called \tool, which
    works on top of an off-the-shelf EC data store. We evaluate \tool
    over realistic applications and microbenchmarks, and present
    results that demonstrate the performance benefits of making
    fine-grained distinctions between consistency guarantees, and the
    ease of doing so via our specification language.

\end{itemize}

The remainder of the paper is organized as follows.  The next section
presents a system model that describes key notions of consistency and
replication.  Sec.~\ref{sec:motivation} provides additional
motivation.  Sec.~\ref{sec:contract-lang} introduces an abstract store
model that forms the basis of our specification language, along with
the language itself. We describe the semantis of our consistency
enforcement runtime, and formalize its correctness and optimality
guarantees in Sec.~\ref{sec:formalization}.  Sec.~\ref{sec:algorithm}
elaborates on the algorithmic aspects of our runtime that is key to
its efficient realization.  Sec.~\ref{sec:evaluation} describes \tool,
an implementation of the specification language and enforcement
mechanism, and evaluates its applicability and practical utility.
Related work and conclusions are presented in
Sec.~\ref{sec:related-work}.


%% . While there are a few well-defined
%% models of weak consistency~\cite{terry-pdis94}, the list is not
%% exhaustive as applications often define consistency models to suit
%% their needs. Furthermore, unlike the standardized implementation
%% techniques, such as 2PL~\cite{2pl}, to enforce strong consistency,
%% enforcement of weak consistency guarantees, including the well-defined
%% ones, is often done via \emph{ad hoc} implementation techniques that
%% are strongly coupled with the the application logic. 
% weak consitency is often enforced using \emph{ad hoc} implementation
% techniques that are tailor-made Even though weak models of
% consistency (e.g. session guarantees from Terry et. al.) have been
% known for more than two decades, they suffer from the lack of
% standardized definitions and enforcement methodologies, which forces
% developers to modify their applications with ad-hoc fixes in order
% to enforce their desired levels of consistency. The lack of a
% general reasoning framework for weak consistency models, has
% resulted in the difficulty of proving correctness and optimality
% properties of such highly error-prone implementations. To make the
% matter worse, many of distributed applications require different
% \emph{combinations} of known consistency guarantees, or might even
% face new consistency requirements after the developement phase is
% over.
