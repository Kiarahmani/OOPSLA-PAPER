\begin{footnotesize}
\vspace{-2mm}\rule{\textwidth}{1pt}\\ \vspace{0mm} \\
We prove the theorem by contradiction:
\begin{fmathpar}
\begin{array}{ll}
H_0: & (\E,op_{<s,i>}) \;\xrightarrow{V}\; (\E',\eff) \\
H_1: & a \in V \\ 
H_2: & (a,\eta) \not\in E'.\visZ\\
H_3: & (E'.A,E'.\visZ \cup \{(a,\eta)\},E'.\soZ) \models  \psi[\eta/\hat{\eta}]\\
H_4: & (\trunc{R}^{-1}_V(a) = \trunc{R}^{-1}_{E.A}(a)) \\
G_0: & \bot
\end{array}
\end{fmathpar}
Now we call {\scriptsize $(E'.A,E'.\visZ \cup \{(a,\eta)\},E'.\soZ)$} as
{\scriptsize $E''$} and derive
the following from $H_3$:
\begin{fmathpar}
\begin{array}{ll}
H_5: & E'' \models \forall x. x\xrightarrow{R}\eta \Rightarrow x
\xrightarrow{\visZ} \eta
\end{array}
\end{fmathpar}
because $E''$ defines the universe of quantification (and since
$E''.A=E'.A$), we get the
following:
\begin{fmathpar}
\begin{array}{ll}
H_6: & \forall (x\in E'.A). E'' \models x\xrightarrow{R}\eta \Rightarrow x
\xrightarrow{\visZ} \eta
\end{array}
\end{fmathpar}
and is rewritten as the following:
\begin{fmathpar}
\begin{array}{ll}
H_7: & \forall (x\in E'.A). (E'' \models x\xrightarrow{R}\eta) \Rightarrow
(E'' \models x
\xrightarrow{\visZ} \eta)
\end{array}
\end{fmathpar}
Now by inversion on $H_0$ we get two cases, one of which is trivial. We
skip the formal proof for it but it is easy to see that in [LB exec]
case, ALL effects in $V$ are made visible to $\eta$, so the set is
trivially 
maximal, i.e. $H_1$ and $H_2$ yield $\bot$. For the other case (UB
exec), we get the following:
\begin{fmathpar}
\begin{array}{ll}
H_8: & V' = \left \lfloor V  \right \rfloor_{\mathtt{max}}\\
H_9: & \auxred{V'} {(\E,op_{<s,i>})} {} {(\E',\eff)}
\end{array}
\end{fmathpar}
by inversion on $H_9$ we get $H_{10}$ and from that and from $H_2$, following a similar argument from the proof of
theorem 1, we get $H_{11}$: 
\begin{fmathpar}
\begin{array}{ll}
H_{10}: & \visZ' = \visZ \cup V' \times \{\eta\}\\
H_{11}: & a \not\in V'
\end{array}
\end{fmathpar}
Now by denoting the interpretation of $R$ under $E''$ as $R''$,
$H_7$ can be rewritten as follows: 
\begin{fmathpar}
\begin{array}{ll}
H_{12}: &  \forall(x\in E'.A). R''(x,\eta) \Rightarrow \visZ''(x,\eta)\\
\end{array}
\end{fmathpar}
Now by inversion on $H_8$, we get the following:
\begin{fmathpar}
\begin{array}{lll}
& H_{13}: & V' \in \left \lfloor V  \right \rfloor \\
& H_{14}: & \not\exists V'' \in \left \lfloor V  \right \rfloor.
|V''|>|V'|\\
(\mathtt{from \; H_{13}}) & H_{15}: & V' \subseteq V \; \wedge \;
(\trunc{R})^{-1}_V(V') \subseteq V' \; \wedge \\ & & (\trunc{R})^{-1}_V(V')  =
(\trunc{R})^{-1}_{E.A}(V') 
\end{array}
\end{fmathpar}
Now we can destruct $R$, where we get multiple cases, only two of which
are non-trivial, ($R=\visZ$) and ($R=\trunc{R};\visZ$)
\begin {itemize}
\item{\bf Case1}($R=\visZ$):\\
$\trunc{R}=\nullR$, thus V itself satisfies the requirements in $H_{15}$
and we get that {\scriptsize $(V=\left \lfloor V  \right \rfloor_{\mathtt{max}})$} and
the following holds:
\begin{fmathpar}
\begin{array}{ll}
H_{16}: & V = V'
\end{array}
\end{fmathpar}
which results in contradiction from $H_1$ and $H_{11}$.
\\
\item{\bf Case2}($R=\trunc{R};\visZ$):\\
Since {\scriptsize $|V'\cup\{a\}|>|V'|$} we have the following:
\begin{fmathpar}
\begin{array}{ll}
H_{17}: & (V'\cup\{a\}) \not\in \left \lfloor V  \right \rfloor 
\end{array}
\end{fmathpar}
which based on the definition yields that the conditiones for holding
the above relation are not true, i.e.
\begin{fmathpar}
\begin{array}{ll}
H_{18}: &  \neg ((V'\cup\{a\}) \subseteq V \; \wedge \;
(\trunc{R})^{-1}_V(V'\cup\{a\}) \subseteq (V'\cup\{a\}) \; \wedge \\ & 
(\trunc{R})^{-1}_V(V'\cup\{a\})  =
(\trunc{R})^{-1}_{E.A}(V'\cup\{a\}))
\end{array}
\end{fmathpar}
or equally:
\begin{fmathpar}
\begin{array}{ll}
H_{19}: &   (V'\cup\{a\}) \not\subseteq V\; \vee \\
        &  (\trunc{R})^{-1}_V(V'\cup\{a\}) \not\subseteq
	(V'\cup\{a\}) \; \vee \\ & 
(\trunc{R})^{-1}_V(V'\cup\{a\})  \not=
(\trunc{R})^{-1}_{E.A}(V'\cup\{a\})
\end{array}
\end{fmathpar}
By inversion on the above, we get three cases, two of which are
trivial. The last conjunct can't hold because of $H_4$ and the first one
also contradics with $H_1$ and $H_{15}$. Thus, we are left with only one
case: 
\begin{fmathpar}
\begin{array}{ll}
H_{20}: & (\trunc{R})^{-1}_V(V'\cup\{a\}) \not\subseteq
	(V'\cup\{a\})
\end{array}
\end{fmathpar}
Now, from the second conjunct in $H_{15}$ we know that it should be the
case that: 
\begin{fmathpar}
\begin{array}{lll}
\hspace{-30mm} (\mathtt{\tiny from\; H_{15}: (\trunc{R})^{-1}_V(V')
\subseteq V' }) \quad & H_{21}: & ((\trunc{R})^{-1}_V(a) \not\subseteq
	(V'\cup\{a\})) 
\end{array}
\end{fmathpar}
The above hypothesis yields the existance of $c\not= a$ such that:
\begin{fmathpar}
\begin{array}{ll}
H_{22}: & c \in (\trunc{R})^{-1}_V(a)\\
H_{23}: & c \not\in V'
\end{array}
\end{fmathpar}
Now, by rewriting ({\scriptsize $R=\trunc{R};\visZ$}) in $H_{12}$ we get $H_{24}$, which can be
rewritten again into $H_{25}$ from the definition:
\begin{fmathpar}
\begin{array}{lrll}
H_{24}: & \forall (x\in E'.A). ((\trunc{R};\visZ)''(x,\eta) & \Rightarrow
& \visZ''(x,\eta)) \\
H_{25}: & \forall (x\in E'.A). (\exists b.\trunc{R}''(x,b) \; \wedge & & \\
& \visZ''(b,\eta) & \Rightarrow & \visZ''(x,\eta))
\end{array}
\end{fmathpar}
Now, we instantiate $H_{25}$ with $x=c$:
\begin{fmathpar}
\begin{array}{ll}
H_{26}: & \exists b.\trunc{R}''(c,b) \; \wedge \;
 \visZ''(b,\eta)  \Rightarrow  \visZ''(c,\eta)
\end{array}
\end{fmathpar}
we can replace $\trunc{R}''$ with $\trunc{R}'$ in above definition,
since from $H_3$, the only difference in interpretation under $E'$ and
$E''$ is the extra element $(a,\eta)$ in $E''.\visZ$ which does not
effect $\trunc{R}''(c,b)$:
\begin{fmathpar}
\begin{array}{ll}
H_{27}: & \exists b.\trunc{R}'(c,b) \; \wedge \;
 \visZ''(b,\eta)  \Rightarrow  \visZ''(c,\eta)
\end{array}
\end{fmathpar}
Moreover, since $c\not= a$, we can replace $\visZ''(c,\eta)$ with
$\visZ'(c,\eta)$:
\begin{fmathpar}
\begin{array}{ll}
H_{28}: & \exists b.\trunc{R}'(c,b) \; \wedge \;
 \visZ''(b,\eta)  \Rightarrow  \visZ'(c,\eta)
\end{array}
\end{fmathpar}
From $H_{15}$ and $H_{22}$ we get $H_{29}$, and $H_{30}$ also holds
trivially from $H_3$:
\begin{fmathpar}
\begin{array}{ll}
H_{29}: & \trunc{R}'(c,a) \\
H_{30}: & \visZ''(a,\eta)
\end{array}
\end{fmathpar}
which can be used in instantiation of  $H_{28}$ with $b=a$ and derive the
following:
\begin{fmathpar}
\begin{array}{ll}
H_{31}: & \visZ' (c,\eta)
\end{array}
\end{fmathpar}
However, we know -from the previously explained argument- that $H_{31}$
results in $H_{32}$, which results in contradiction with $H_{23}$.
\begin{fmathpar}
\begin{array}{ll}
H_{32}: & c \in V'
\end{array}
\end{fmathpar}
\fcolorbox{red}{white}{\color{black} \footnotesize QED.}

\end{itemize}
\end{footnotesize}
