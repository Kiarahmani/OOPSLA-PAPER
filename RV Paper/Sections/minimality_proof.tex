\begin{footnotesize}
\begin{definition} 
We define the completment of a given set of effects $S$ (under an
execution state $E$) as the super set of $S$, containing ALL the
mid-level effects required to determine ALL the dependencies of the
effects in $S$, i.e.
\begin{fmathpar}
\begin{array}{lll}
S' \in  \left \lceil S \right \rceil \iff  R^{-1}_{S'}(S) = R^{-1}_{E.A}(S)
\end{array}
\end{fmathpar}
\end{definition}
\hrule
\vspace{4mm}

Now, using the above theorem and lemma, we present the proof of the
theorem 3, which starts by listing the following hypotheses and the goal:
\begin{fmathpar}
\begin{array}{ll}
H_0: & \auxred{S} {(\E,op_{<s,i>})} {} {(\E',\eff)}    \\
H_1: & S \cup \{\eta\} \; is \;  \psi\mathtt{-consistent}\\
G_0: & \exists E''.\exists \eta'. \exists V.
((\E,op_{<s,i>})\;\xrightarrow{V}\;(\E'',\eff'))
\end{array}
\end{fmathpar}
Now, by destructing $R$ we get two non-trivial cases:
\begin{itemize}
\item {\bf Case1}($R=\trunc R ;\visZ$):\\
In this case\footnote{\scriptsize Note that in this case the goal $G_0$,
trivially holds. That is because the contract in this case is
\rulelabel{ub}, which represents executions whitout blocking or waiting,
that can always make progress  by showing \emph{some} set of effects to
the operations}, we generate the premises of the \rulelabel{ub exec} to
achieve the goal as follows. Firstly, we define $S'$ and present $\eta'$: 
\begin{fmathpar}
\begin{array}{ll}
H_3: & S' = \left
\lfloor S \right \rfloor_{\mathtt{max}} \\
H_4: & \eta' = (s,op,F_{op}(S')) 
\end{array}
\end{fmathpar}
Moreover, we will define the followings, which will be used when
presenting $E''$:
\begin{fmathpar}
\begin{array}{ll}
H_5: & \soZ'' = \soZ \cup A_{\mathtt{(SessID = s)}}\times \{\eta'\}  \\
H_6: & \visZ'' = \visZ \cup S' \times \{\eta'\} \\
H_7: & A'' = E.A \cup \{\eta'\}
\end{array}
\end{fmathpar}
Now we present $V$ and $E''$ as follows and rewrite the goal:
\begin{fmathpar}
\begin{array}{ll}
H_8: & V = S\\
H_9: & E''=(A'',\soZ'',\visZ'')\\
G_1: & (\E,op_{<s,i>})\;\xrightarrow{V}\;(\E'',\eff')
\end{array}
\end{fmathpar}
by applying \rulelabel{ub exec} on $G_1$ we get the following new goals
(after rewriting $H_9$ and $H_3$): 
\begin{fmathpar}
\begin{array}{ll}
G_2: & r_k = \visZ\\
G_3: & S \subseteq E.A \\
G_4: & S' = \left \lfloor S \right \rfloor\\
G_5: & \auxred{S'} {(\E,op_{<s,i>})} {} {(\E'',\eff')}  
\end{array}
\end{fmathpar}
first three goals are proved via the assumptions, and the last one can
be easily shown to hold by applying \rulelabel{oper} and deriving the
following new goals:
\begin{fmathpar}
\begin{array}{ll}
G_6: & S' \subseteq E.A\\
G_7: & F_{op}(S') = v\\
G_8: & \eta' = (s,op,v)\\ 
G_9: & \eta \not\in S'\\
G_{10}: & E''.A = E.A \cup \{\eta'\}\\
G_{11}: & E''.\visZ = E.\visZ \cup S' \times \{\eta\}\\
G_{12}: & E''.\soZ = E.\soZ \cup (A_{\mathtt{(SessID=s)}}) \times \{\eta\}
\end{array}
\end{fmathpar}
all the above goals have already been shown in the assumptions and the
case is proved.
\\
\item {\bf Case2}($R=\trunc R;\soZ$):\\
Similarly in this case we define the following: 
\begin{fmathpar}
\begin{array}{ll}
H_{13}: & V = \left \lceil S\cup\{\eta\} \right \rceil \\
\end{array}
\end{fmathpar}
which yields: 
\begin{fmathpar}
\begin{array}{ll}
H_{14}: & \forall (x \in S\cup\{\eta\}).R^{-1}_V(x)=R^{-1}_{E'.A}(x)\\
\end{array}
\end{fmathpar}
and also: 
\begin{fmathpar}
\begin{array}{ll}
H_{15}: & R^{-1}_V(\eta)=R^{-1}_{E'.A}(\eta)\\
\end{array}
\end{fmathpar}
Similar to the previous case, we now define the followings: 
\begin{fmathpar}
\begin{array}{ll}
H_{16}: & \eta' = (s,op,F_{op}(V))\\
H_{17}: & \soZ'' = \soZ \cup A_{\mathtt{(SessID = s)}}\times \{\eta'\}  \\
H_{18}: & \visZ'' = \visZ \cup V \times \{\eta'\} \\
\end{array}
\end{fmathpar}
Now we present$E''$ as follows and rewrite the goal:
\begin{fmathpar}
\begin{array}{ll}
H_{19}: & E''=(A'',\soZ'',\visZ'')\\
G_{1}: & (\E,op_{<s,i>})\;\xrightarrow{V}\;(\E'',\eff')
\end{array}
\end{fmathpar}
by applying \rulelabel{lb exec} on $G_1$ we get the following new goals
\begin{fmathpar}
\begin{array}{ll}
G_2: & r_k = \soZ\\
G_3: & V \subseteq E.A \\
G_4: & R^{-1}_V(\eta') = R^{-1}_{E''.A} (\eta')   \\
G_5: & R^{-1}_V (\eta') \subseteq V \\ 
G_6: & \auxred{V} {(\E,op_{<s,i>})} {} {(\E'',\eff')}  
\end{array}
\end{fmathpar}
Now, $G_2$ and $G_3$ are trivially proved from the assumptions, and
$G_6$ also can be easily proved following the argument from the previous case. 
We prove  $G_4$ and $G_5$, by a new  claim that $R^{-1}_{E'.A}(\eta) =
R_{E''.A}^{-1}(\eta')$ which will be proved separately.
Thus, we can rewrite the goals and add the new claim:
\begin{fmathpar}
\begin{array}{ll}
G_7: & R^{-1}(\eta) = R^{-1}_{E'.A} (\eta) \\
G_8: & R^{-1}(\eta) \subseteq V \\ 
G_9: & R^{-1}_{E'.A}(\eta) = R^{-1}_{E''.A}(\eta')
\end{array}
\end{fmathpar}
Now $G_7$ is equal to  the assumption $H_{15}$, and $G_8$ is the direct result
of applying the lemma 2 on $H_1$.
Now by rewriting $R=\trunc{R};\soZ$ in $G_9$ we have the following: 
\begin{fmathpar}
\begin{array}{ll}
G_{10}: & (\trunc{R};\soZ)^{-1}_{E'.A}(\eta) =
(\trunc{R};\soZ)^{-1}_{E''.A}(\eta')
\end{array}
\end{fmathpar}
Now, note that the only difference in $E'$ and $E''$ is in how the update
the $\visZ$ relation from $E$, the former makes the set $S$ visible to the
operation and the latter the set $\left \lceil S \cup \{\eta\}\right
\rceil $. Now since the given relation $R$ ends with an $\soZ$ relation,
it is straigtforward to show that $G_10$ holds and thus the case (and
the theorem) is proved. 
\end{itemize}
\fcolorbox{red}{white}{\color{black} \footnotesize QED.}
\end{footnotesize}




















