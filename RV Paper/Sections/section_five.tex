%========================================================================================================================
%========================================================================================================================
%========================================================================================================================
\section{Semantics}
\label{sec:semantics}
In this section, we present the consistency enforcement mechanism of
\tool, abstracted as a formal operational semantics. Our approach is
complete for the specification language defined in
Sec.\ref{sec:ctrt_language}.  However for better comprehensibility, we
present the semantics and the theorems paramterized over a contract
consisting of a single proposition.  Therefore, in the rest of this
section, we will assume a given contracat $\psi$ of the following
form:
\begin{fmathpar}\footnotesize
\begin{array}{lll}
\psi = \forall a. a \xrightarrow{\rel_1;\rel_2;...;\rel_k} \hat{\eta} \Rightarrow a 
\xrightarrow{\visZ} \hat{\eta}
&\qquad 
\quad & \rel_i \in \{\visZ;\soZ\}
\end{array}
\end{fmathpar}
%
%

The operational semantics defines a small-step relation over \emph{execution
states}, which are tuples of the form {\footnotesize $\E=(\EffSoup,\visZ
,\soZ)$}.
The \emph{effect soup} $\EffSoup$ stands for the set of all
effects produced in the system, and \emph{primitive relations}
{$\visZ,
\soZ \subseteq \EffSoup \times \EffSoup$}, respectively represent the
visibility and session order 
among such effects. Figs.~\ref{subfig:execution_graph} and
\ref{subfig:execution_example} present a simple
execution state consisting of 9 effects with associated
primitive relations\footnote{We omit drawing transitive $\soZ$
edges (e.g. between
$\eff_8$ and $\eff_1$) for better readability.}.
\begin{figure}[t]

	\hspace{-4 mm}
	\begin{subfigure}[b]{0.298 \textwidth}
	\includegraphics[scale=0.31]{Figures/execution.pdf}
 	\subcaption{\scriptsize An execution state \E}
	\label{subfig:execution_graph}
	\end{subfigure}
	%
	\quad \vrule \quad
	%
	\begin{subfigure}[b]{0.3 \textwidth}
	\begin{smathpar}
	\scriptsize    
	\begin{array}{lcl}
	\E.\EffSoup & = & 
	\{\eta_1,\eta_2,\eta_3,\eta_4,\eta_5,\\ & & \;\eta_6,\eta_7,\eta_8,\eta_9\}\\
	\E.\visZ & = & 
	\{(\eta_5,\eta_3),(\eta_4,\eta_3),
	\\ & &\;(\eta_3,\eta_1),(\eta_2,\eta_1),	
	\\ & &\; (\eta_6,\eta_2)\} \\ 
	\E.\soZ & = & \{(\eta_9,\eta_3),(\eta_8,\eta_6),
	\\ & &\; (\eta_6,\eta_1), (\eta_8,\eta_1),
	\\ & &\; (\eta_7,\eta_2) \} 
	\end{array}
	\end{smathpar}
	\subcaption{\scriptsize Effect soup and primitive relations}
	\label{subfig:execution_example}
	\end{subfigure}
	%
	\quad \vrule \quad
	%
	\begin{subfigure}[b]{0.26 \textwidth}
	\begin{smathpar}
	\scriptsize
	\begin{array}{lll}
	\visZ^{-1}  (\eta_1) & = & \{\eta_2,\eta_3\} \\ 
	\soZ^{-1}  (\eta_1) & = & \{\eta_6, \eta_8\} \\
	(\soZ\cup\visZ)^{-1} (\eta_1) & = &
	\{\eta_2,\eta_3, 
	\eta_6,\eta_8\} \\ 
	(\visZ;\visZ)^{-1} (\eta_1) & = &
	\{\eta_2,\eta_3,\eta_4,\eta_5,\eta_6\} \\ 
	(\soZ;\visZ)^{-1}(\eta_1) & = & \{\eta_7,\eta_9\}
	
	\end{array}
	\end{smathpar} \\
	\subcaption{\scriptsize Relation inverse examples}
	\label{subfig:inverse_example}
	\end{subfigure}
	\caption{A simple execution state }
\label{fig:execution_state}
\end{figure}

We denote the subset of $\EffSoup$ consisting of effects that
satisfy a certain condition as $\EffSoup_{(\mathtt{condition})}$.

% models
Note that \tool's contracts are in fact constraints over execution states,
where the domain of quantification is fixed to the effect soup
$\EffSoup$, and
interpretation for $\soZ$ and $\visZ$ relations (which occur free in the
contract formulae) are also provided. Thus, execution states are
potential models for any first-order formula expressible in the
specification language. If an execution state $\E$ is a valid model
for a contract $\psi$, we say that $\E$ satisfies $\psi$ ($\E
\models \psi$). 

%reduction relation
The reduction relation in our semantics is of the form
{\footnotesize $
(\E,\op_{<s,i>}) \;\xrightarrow{V}\; (\E', \eff),
$}
which can be interpreted as a transformation of the initial execution state
$\E$, caused by a replica with a local 
set of effects $V$, when it executes
$\op$, the $i^{th}$ operation from the session $s$. 
During this reduction step, a new effect $\eff$ is produced and added to
the system, resulting in a new execution state $\E'$ composed of an updated effect
soup and new primitive relations.





%=============================================================================================================
%=============================================================================================================
%--------- Definitions to be used in the semantics
%=============================================================================================================
\subsection{Preliminaries}
\label{subsec:prelim}
Before presenting the operational semantics, we first introduce
supporting definitions and notations.
We start by defining the interpretation of an \emph{inversed} 
dependency relation 
{\footnotesize $\Rel^{-1}$} under an execution state $\E$, 
which is utilized in the basis of our consistency enforcement mechanism.
We previously mentioned our interpretation for $\soZ$ and $\visZ$
between
effects under  $\E$; this can now be straightforwardly extended to their inverse as
follows\footnote{Note that when the input of an inversed relation is a singleton
$\{\eta\}$, we drop the brackets and simply write it as
$\rel^{-1}(\eta)$}:
\begin{equation}
\label{eq:r_inv}
\scriptsize
\rel^{-1}(S) = 
\begin{array}{lcl}
\bigcup\limits_{b\in S}\{a|(a,b) \in \E.\rel \} & \qquad & \rel\in\{\soZ,\visZ\}
\end{array}
\end{equation}
Additionally, based on our interpretation of the sequences of seed
relations given in Sec.\ref{sec:ctrt_language}, we can
extend the above definition:
\begin{equation}
\label{eq:seq_inv}
\scriptsize
\begin{array}{lllll}
b \in  (\Rel';\rel)^{-1}(a) & \iff & \exists c. c \in \rel^{-1}(a)
& \wedge & b \in (\Rel')^{-1}(c) 
\end{array}
\end{equation}
It might seem that we are ready to define any $\Rel^{-1}$ 
based on  the two definitions above; 
however, note that definition (\ref{eq:seq_inv}) fails to capture
the reality of our system model, where all computations are
performed by replicas independently; at any given moment, a replica might have access to only a
\emph{subset of all produced effects} in the system.
For example, consider  {\footnotesize$(\soZ;\visZ)^{-1}(\eta_1)$} under the
execution state presented in Fig.\ref{fig:execution_state}. 
In order to compute this set, based on (\ref{eq:seq_inv}) we have: 
\begin{smathpar}
\scriptsize
\begin{array}{lllll}
b \in  (\soZ;\visZ)^{-1}(\eta_1) & \iff & \exists c. c \in
\visZ^{-1}(\eta_1)
& \wedge & b \in (\soZ)^{-1}(c)
\end{array}
\end{smathpar}
%%%SJ: This is confusing.
Now, since there exists \emph{mid-level} effects $c=\eta_2$ and 
$c=\eta_3$,  that satisfy the above definition respectively 
for $\eta_7$ and $\eta_9$, we can conclude: {\footnotesize$(\soZ;\visZ)^{-1}(\eta_1) =
\{\eta_7,\eta_9\}$}.
Consider a replica that contains {\footnotesize$\{\eta_1, \eta_6, \eta_7,
\eta_9\}$} at the moment, and wants to check if the dependencies of $\eta_1$ are locally present 
or not. Even though based on the above definition, the
answer is affirmative (since the replica does contain $\{\eta_7,\eta_9\}$), 
the replica has no way to verify it, since the mid-level
effects $\eta_2$ and $\eta_3$ are not present at the replica yet. 

To capture the above property, we redefine the 
inverse of $\Rel$, \emph{according to a set of available effects $V$}, 
that considers whether all required mid-level effects are present in $V$.
The following definition is based on (\ref{eq:r_inv}) and a more
strict version of (\ref{eq:seq_inv}):
\begin{equation}
\label{eq:R_inv}
\scriptsize
b \in \Rel^{-1}_V(a) \iff
\begin{cases}
\begin{array} {lllll} 
\bot & \;\myif\; & \Rel = \nullR& & \\
b \in \rel^{-1}(a) & \;\myif\; & \Rel=\rel & & \\
\exists c. c \in
\rel^{-1}(a) \wedge b \in (\Rel')^{-1}(c) \wedge
\rel^{-1}(a) \subseteq V   & \;\myif\; & \Rel=\Rel';\rel & & \;
\end{array}
\end{cases}
\end{equation}
For example, in Fig.\ref{fig:execution_state}, 
{\footnotesize  ($\eta_9 \in
(\soZ;\visZ)_{\{\eta_1,\eta_3\}}^{-1}(\eta_1)$)}
holds, but {\footnotesize($\eta_9 \not\in (\soZ;\visZ)_{\{\eta_1\}}^{-1}(\eta_1)$)}. 

We define a set $V$ to be \emph{self-contained} for
a given effect $\eta$,
written as {\footnotesize $\SC
R \eta V$}, if $V$ contains all the required mid-level effects to compute
$R$ inverse of $\eta$ in totality, i.e.
\begin{equation}
\scriptsize 
\SC R \eta {V} \iff R^{-1}_V(\eta) = R^{-1}_{E.A}(\eta)
\end{equation}
For example in Fig.\ref{fig:execution_state}, {\footnotesize $\SC R
{\eta_1} V$} holds for an arbitrary  $R$ and for any $V$ that is a superset of
{\footnotesize $\{\eta_1,\eta_2,\eta_3,\eta_4,\eta_5\}$}.

We define {\footnotesize \trunc{}} as a function that
given  $R \in$ \relationS{}, 
returns a new relation by removing the last element from the sequence
in R:
\begin{equation}
\scriptsize
\trunc{R} = 
\begin{cases}
\begin{array}{lcl}
\nullR & \quad \myif \quad & R = \rel \quad \mathtt{or} \quad R = \nullR \\
R' & \quad \myif \quad & R = R';\rel 
\end{array}
\end{cases}
\end{equation}

Finally, we define \emph{closed subsets} of a given set $V$, 
as the subsets that are closed under {\footnotesize$(\trunc{R}^{-1}_V)$}, that
also contain all the required mid-level effects to compute
{\footnotesize $\trunc{R}^{-1}$}.
Moreover, we define the largest element among such
subsets, as the \emph{maximally closed subset} of V as follows\footnote{We slightly abuse the previously defined notation
in (\ref{eq:R_inv})
and use a \emph{set} of effects as the input of
$R^{-1}$, which is defined as:
$x \in R^{-1}_V(S) \iff \exists (y\in S). x \in R^{-1}_V(y)$.}:
%
\begin{fmathpar}
\begin{array}{rlllll}
\mathtt{closed \; subsets:} &  V' \in \left \lfloor  V \right \rfloor & \iff & V' \subseteq V & \wedge &
(\trunc R)_V^{-1}(V') \subseteq V' ~\wedge ~ \SC {\trunc{R}} \eta {V'}    \\
\mathtt{maximally \; closed \; subset:} & V' = \left \lfloor  V \right
\rfloor_{\mathtt{max}} & \iff & V' \in \left \lfloor  V \right \rfloor &
\wedge & \not\exists V'' \in \left \lfloor  V \right \rfloor. |V''|>|V'|
\end{array}
\end{fmathpar}






%=============================================================================================================
%--------- The operational semantics
%=============================================================================================================
\subsection{Core Operational Semantics}

\begin{figure*}[h]
\raggedright
%
\textbf{Auxiliary Definitions}\\
%
\begin{minipage}{\columnwidth}
\begin{smathpar}
\stretcharraybig
\begin{array}{lclcl}
  \multicolumn{5}{c}{
    \hspace{-8mm}
    {op} \in \mathtt{Operation\; Name} \spc \spc
    {s} \in \mathtt{Session\; Id} \spc\spc
    {i} \in \mathtt{Effect\; Id} \spc\spc
  }\\
  \eff & \in & \mathtt{Effect} & \coloneqq &  (s,i,op,v)\\
  F_{op} & \in & \mathtt{Op.\,Def.} & \coloneqq & \set{\eff} \rightarrow \eta\\
  \EffSoup & \in & \mathtt{Eff\,Soup}	  & \coloneqq & \set{\eff} \\
  \visZ,\soZ &	\in & \mathtt{Relations} & \coloneqq & \set{(\eff,\eff)} \\
  {\E} 		& \in & \mathtt{Exec\;State}  & \coloneqq & \Exec \\
\end{array}
\end{smathpar}
\end{minipage}
%

\begin{smathpar}
\begin{array}{c}
\ssn(s,\_,\_,\_) = s \spc\spc
\id(\_,j,\_,\_) = j \spc\spc
\oper(\_,\_,op,\_) = op \spc\spc
\rval(\_,\_,\_,v) = v\\
\end{array}
\end{smathpar}

\textbf{Auxiliary Reduction} \;
  \fbox{\(\auxred{S} {(\E,op_{<s,i>})} {} {(\E',\eff)}\)}\\

\begin{minipage}{\textwidth}
\rulelabel{Oper}
\begin{smathpar}
\stretcharraybig
\begin{array}{l}
\RuleTwo
{
%\Theta(\rho \mapsto (v,cache)) \qquad
F_{op}(S) = \eta \qquad
\ssn(\eta) = s \qquad 
\id(\eta) = i \qquad
%\{\eff'\} = \EffSoup_{({\sf SessID}=s,\,{\sf SeqNo}=i-1)}\\
\EffSoup' = \EffSoup \cup \{\eff\} \qquad S \subseteq \EffSoup\\
\visZ' = \visZ \cup S \times\{\eff\}\qquad
\soZ' = \soZ \cup \{(\eta',\eta) \,|\, \eta'\in \EffSoup \conj 
    \ssn(\eta')=s \conj \id(\eta')<i\}\qquad
%\soZ' = (\soZ^{-1}(\eff') \cup \eff') \times\{\eff\} \cup \soZ
}
{
  \auxred {S} {((\EffSoup,\visZ,\soZ), op_{<s,i>}))}
  {} {((\EffSoup',\visZ',\soZ'),\eta)}
}
\end{array}
\end{smathpar}
\end{minipage}
\vspace{4mm}\\
\textbf{Operational Semantics} \;
  \fbox{\((\E,op_{<s,i>}) \;\xrightarrow{V}\; (\E',\eff)\)}\\
\begin{minipage}{2.8in}
\rulelabel{vis Exec}
\begin{smathpar}
\stretcharraybig
\begin{array}{l}
\RuleTwo
{
     V \subseteq E.A \spc  R^{-1}_{V}(\{\eta\}) \subseteq V \spc  (vis \subseteq r_k) \spc 
  V'= \left \lfloor V \right \rfloor_V \\
  \auxred {V'} {(E, op_{<s,i>}))}
    {} {(E',\eta)} 
}
{
  (\E,op_{<s,i>}) \;\xrightarrow{V}\; (\E', \eff)
}
\end{array}
\end{smathpar}
\end{minipage}
\hspace{12 mm}
\begin{minipage}{2.3in}
\rulelabel{non-vis Exec}
\begin{smathpar}
\stretcharraybig
\begin{array}{l}
\RuleTwo
{
 
  V \subseteq E.A \spc  R^{-1}_{V}(\{\eta\}) \subseteq V\spc (vis \nsubseteq r_k )\\
  \auxred {V} {(E,op_{<s,i>}))}
    {} {(E',\eta)}
}
{
  (\E,op_{<s,i>}) 
    \;\xrightarrow{V}\;
  (\E',\eff) 
}
\end{array}
\end{smathpar}
\end{minipage}


\caption{Core Operational semantics of a replicated data store.}
\label{sem:oper}
\end{figure*}

 %--- The Figure Containing the Rules
%--- Section intro
Our operational semantics is defined as a set of reduction rules
representing our consistency enforcement approach (see
Fig.\ref{fig:semantics}). The small-step reduction relation
($\rightarrow$) is parametrized over a set $V$, which stands for the
locally available set of effects at the replica taking the reduction
step.  Trivially, $V$ must be a subset of all effects in the system at
the initial execution state, however, there is no other restrictions
on $V$, since we only assume eventual consistency in the underlying
store.

%--- The Aux [OPER] rule
The rule
\rulelabel{oper} defines the abstract procedure of generating a new effect $\eff$, by witnessing a set
of effects $S$, using a user-defined function $F_{op}$. 
We formally define an effect as a tuple $\eff=(s,op,v)$, representing the
session $s$, operation name \emph{op} 
whose execution created $\eff$, and the value $v$
that the replica returns, responding to that operation.
%
Moreover, the rule explains how the execution state changes after a new
effect is produced. Specifically, in the new execution state, 
the effect soup
$\EffSoup'$ contains the newly created effect $\eff$,
the relation $\visZ'$
captures the fact that all effects in the set $S$ were made
visible to $\eta$, and $\soZ'$ states that all effects from the same
session as the current operation that are
already presenet in the system, should be in session
order with $\eff$ in the final execution state.

%--- rule for (->) relation
%UB
Rule\rulelabel{ub exec}, defines the execution of an operation in a
replica under a \UB{} contract.  The rule requires operations to only
witness $V'$, the maximally closed subset of $V$.  Thus,, the rule
governs how replicas create safe environments for operations, by
filtering out undesirable or uwanted effects.

%LB
Rule \rulelabel{lb exec} defines the step taken by a replica when an
operation is executed under an \LB{} contract. The precondition
$R_V^{-1}(\eff)\subseteq V$ in the rule ensures that the reduction
happens only if the effects necessary to avoid the specified anomaly
are present in $V$, assuming that $V$ contains all the mid-level
effects to determine dependencies of the newly created effect $\eta$
(i.e.  is a self contained set). Thus, the rule governs replicas to
block execution of an operation under an \LB{} contract, if the
replica is unable to verify the presence of all necessary dependent
effects.


%=============================================================================================================
%--------- Theorem on correctness of enforcement
%=============================================================================================================
\subsection{Soundness and Optimality}
\label{subsec:sound}
\begin{definition}
For a given contract ($\psi=\forall(a,b). a
\xrightarrow{Q} b \Rightarrow a \xrightarrow{vis} b $), we define a set
of effects S, to be  $\psi$-consistent under an execution $E$, if and
only if: 
$\forall (\eff \in S). \forall(a\in E.\EffSoup). Q(a,\eta)
\Rightarrow a \in S$ 
\\(We use definition of $Q^{-1}$ from the previous section to formally
define: $Q(a,b) \iff a \in Q_{E.A}^{-1}(b)  $)
\end{definition}

\begin{theorem}
For all operation executions of the form: 
$$
(\E,op_{<s,i>}) 
    \;\xrightarrow{V}\;
  (\E',\eff) 
,$$ 
if $V$ is $\psi$-consistent then $(V\cup\{\eta\})$ is also
$\psi$-consistent.
\end{theorem}
\begin{proof}
Sine $\eff$ is the only effect added to $V$, we only need to check the
following property for $\eta$, in order to prove $\psi$-consistency of
$V\cup\{\eta\}$:
$$\forall (a \in E.A). Q(a,\eta) \Rightarrow vis(a,\eta)  $$
\\
(1) $Q(a,\eta) \spc$ (hypothesis)
\\(2) $a \in Q^{-1}_V (\eta) \spc$  (we need to prove this)
\\(3) $a \in V \spc$ (premise of the rules and (2))
\\(4) $V$ is $\psi$-consistent so from (3) and definition of
consistency, the property holds for $a$
\paragraph{} In order to prove (2), we \emph{probably} have to destruct the contract and
show that if it ends with $so$, the property trivially holds since the
vis set of $\eta$ is fixed. For the other case the premise of the rule
[VIS Exec] will have the enought information to prove.
\end{proof}
\newpage


%=============================================================================================================
%--------- Theorem on maxVis and minWait
%=============================================================================================================
%
%Subsection of the maximality of the set made visible and liveness
%properties 
%
\begin{theorem}
\label{theorem:two}
For all reduction steps
{\footnotesize $
(\E,op_{<s,i>}) 
    \xrightarrow{V}
  (\E',\eff) 
$},
the set of effects made visible to $\eta$ is maximal. i.e. for all
 {\footnotesize $a \in V$}, if 
 {\footnotesize $ \SC {\trunc{R}} a V$}, then 
\begin{fmathpar}
(a,\eta) \not\in \E'.\visZ \Rightarrow 
(\E'.A,\E'.\visZ \cup \{a,\eta\}, \E'.so) \not\models \psi[\eta/\hat{\eta}]
\end{fmathpar}
\end{theorem}


%
% THE MINIIMAL WAIT THEOREM
%

\begin{theorem}
\label{theorem:three}
For all execution states $E$, if there
exists ($S\subseteq E.A$) such that: 
\begin{smathpar}
\auxred{S} {(\E,op_{<s,i>})} {} {(\E',\eff)} \spc \wedge \spc (\mathtt{S \cup \{\eta\} \; is 
\;} \psi \mathtt{-consistent \; under \;} E' )  
\end{smathpar}
then there exist  $E''$, $\eff'$ and $V\subseteq E.A$ such that:
$((\E,op_{<s,i>})\;\xrightarrow{V}\;(\E'',\eff'))$
\end{theorem}
\begin{proof}
The proof is given by choosing set $V$ to be equal to $S$, and then
considering two cases, where either $S$ or $\left \lfloor S \right
\rfloor_S$ are made visible to operations, if the contract is respectively
waiting and non-waiting. In both cases, all premises of taking an step
are satisfied.
A detailed proof can be found in appendix
\ref{app:proof3}.
\\
\end{proof}


























% GENERALIZATION OF THE SEMANTICS WHICH I DONT THINK SHOULD BE INCLUDED
% BECAUSE OF LACK OF SPACE
%=============================================================================================================
%=============================================================================================================
\begin{comment}
%=============================================================================================================
%--------- How it can be generalized for all contracts
%=============================================================================================================
\subsection{Generalization}
\label{subsec:generalization}
We finish this section by extendeding our ideas in two dimentions. 
We will first explain how to handle an arbitrary
contract $\psi$ of the following form:  
\begin{fmathpar}
\psi = \pi_1 \wedge \pi_2 \wedge ... \wedge \pi_m \qquad \qquad 
\pi_i = \forall (a,b). a \xrightarrow{R_i} b \Rightarrow a
\xrightarrow{\visZ} b
\end{fmathpar}
Later, we will
explain how to maintain multiple levels of consistency simultaneously,
each of which is defined for a different operation name. We will assume an arbitrary contract
$\psi_{\op}$ for every user-defined operation $\op$, and explain how to
modify our system model to preserve them all.

To begin with, as we mentioned earlier, all propositions in our specification language,
either put a maximal or a minimal bound on the subset of local effects 
to be made visibe to each opreation. 
This simply means that when the system is given a conjunction
of propositions, it should define the such subsets in a way, so it would not violate
\emph{any} of them. 
Therefore, by a few modifications we can extend the system to support
all contracts. Firstly, the single premise $R_V^{-1}(\eta) \subseteq V$
in the reduction rule should be replaced with the following
conjunction: 
\begin{fmathpar}
\bigwedge_{1 \leq i \leq m} (R_i)_V^{-1}\subseteq V
\end{fmathpar}
Secondly, the definition of the maximal closed subset of local effects
must also be modified to a subset that is closed under \emph{all} given
relations:
%---------------------------------------------------------------------
% I am not sure if we should include the formal definition here. It is
% unnecessarily complex
\begin{fmathpar}
\left \lfloor S \right \rfloor_V = S' \spc \iff \spc S'
\subseteq S \; \wedge \;
\bigwedge(R_i)_V^{-1}(S') \subseteq S' \; \wedge \; 
\not\exists
S''.(\bigwedge ((R_i)_V^{-1}(S''))\subseteq S''\wedge |S''|>|S'|)
\end{fmathpar}
%---------------------------------------------------------------------

Moreover, for modifying the system to handle multiple contracts
simultaneously, we can
extend the local effect set $V$, to a sequence
of sets $V_{\op_i}$, each maintaning  the consistency level for an
operation type $\op_i$. Now we define the modified form of execution steps as
follow:
\begin{fmathpar}
(\E,\op_{<s,i>}) 
    \;\xrightarrow{V_{\op}}\;
  (\E',\eff) 
\end{fmathpar}
The local effect set $V$ must also be replaced with $V_{\op_i}$ in the 
premises of the reduction rules, so each operation of type $\op_i$ would
only witness the associated subset for its own consistency requirements.
This abstractly represents our implementation, in the sense that all operations
work only on a specific subset of available effects at any replica. The subset, is
maintained according to the contract assosiated with each operation, and
is guaranteed to preserve the consistency requirements following the
theorems of sections \ref{subsec:sound} and \ref{subsec:opt}. 
\end{comment}
