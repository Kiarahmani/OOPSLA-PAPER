\section {Specification Language} 
\label{sec:ctrt_language}
\begin{wrapfigure}{r}{0.48\textwidth}
\begin{minipage}{0.5\textwidth}
\centering
  \begin{smathpar}
  \begin{array}{lclcl}
		r & \in & \texttt{rel.seed} & \coloneqq & \visZ \ALT \soZ \ALT r \cup r \\
               R & \in & \texttt{relation} & \coloneq &  r \ALT r^*  \ALT r;R \ALT r^*;R   \\
	     \pi & \in & \texttt{prop} & \coloneqq & \forall(\eff,\eff').
      ~\eff \xrightarrow{R} \eff' ~\Rightarrow~ \eff \xrightarrow{\visZ} \eff'\\
		\psi & \in & \texttt{spec} & \coloneqq & \pi \ALT \pi \conj \pi\\
  \end{array}
  \end{smathpar}
\caption{Syntax of the Specification Language}
\label{fig:ctrt_syntax}
\end{minipage}
\end{wrapfigure}

The formal syntax of our specification (or contract) language, presented in
Fig.\ref{fig:ctrt_syntax}, allows definition of
\propS{}, that is a FOL formula
establishing dependency relations between effects,
which is necessary to determine effects an operation may witness, under
a consistency requirement.
The language is seeded with $\soZ$ and $\visZ$, respectively representing session
order and visibility over effects, 
and defines dependency \relationS{} as a sequence\footnote{\tool also allows
using closures of seeds, which is omitted here for
simplicity} of seeds,  
where 
({\footnotesize $a \xrightarrow{\rel_1;...;\rel_k} b$})
must be interpreted as 
{\footnotesize$\exists c. (a
\xrightarrow{\rel_1;...;\rel_{k-1}} c
\wedge c \xrightarrow {\rel_k} b)$}
in meta-language. We also define $\nullR{}$ as a relation that never
holds.
Additionally, the language allows definition of \specS{},
a conjunction of propositions, that is used to define a safe environment
free from \emph{multiple} inconsistencies. 
Our language is crafted to capture all fine-grained weak consistency
levels, including the famous session guarantees proposed by Terry et al. \cite{terry}, presented
in Fig.\ref{fig:ctrt_example}.

%
% UB and LB contracts
We finish this section by introducing two syntactical classes of 
contracts, and explaining how they can be
satisfied with different enforcement techniques.
\begin{description}
\item {\textsf LB}: If all the defined dependency relations for a
contract end with an \soZ, i.e. are of
the following form: ({\footnotesize $\forall a. a
\xrightarrow{r_1;r_2;...;\soZ} \hat{\eff}
\Rightarrow a \xrightarrow{\visZ} \hat{\eff}$}), we call it a \emph{lower
bound} (\LB{}) contract, since it specifies the smallest set of
effects that any operation should witness to maintain consistency, e.g.
\rmwCTRT{} and \mrCTRT{} in Fig.\ref{fig:ctrt_example}.

\item {\textsf UB}: Similarly, we define the \emph{upper bound} (\UB{})
contracts, as the ones with all dependency relations ending with a $\visZ$.
These contracts define constraints on the set of effects made visible
to each operation, by enforcing that if an effect is in the set,
certain dependencies of that effect must also be included, e.g.
\visCTRT{} and \mwCTRT{} in Fig.\ref{fig:ctrt_example}.
\end{description}
Our consistency enforcement approach is based on blocking operations with \LB{}
contracts to make sure that they witness \emph{all effects that they are
supposed to}, and filteration for \UB{} contracts to make sure that they
would not witness \emph{effects that they are not supposed
to}. A combination of both approaches is also taken for contracts that are
neither  \LB{} nor \UB{}, i.e hybrid contracts.
