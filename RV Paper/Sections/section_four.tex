\section {Specification Language} 
\label{sec:ctrt_language}
\begin{wrapfigure}{r}{0.48\textwidth}
\begin{minipage}{0.5\textwidth}
\centering
  \begin{smathpar}
  \begin{array}{lclcl}
		r & \in & \texttt{rel.seed} & \coloneqq & \visZ \ALT \soZ \ALT r \cup r \\
               R & \in & \texttt{relation} & \coloneq &  r \ALT r^*  \ALT r;R \ALT r^*;R   \\
	     \pi & \in & \texttt{prop} & \coloneqq & \forall(\eff,\eff').
      ~\eff \xrightarrow{R} \eff' ~\Rightarrow~ \eff \xrightarrow{\visZ} \eff'\\
		\psi & \in & \texttt{spec} & \coloneqq & \pi \ALT \pi \conj \pi\\
  \end{array}
  \end{smathpar}
\caption{Syntax of the Specification Language}
\label{fig:ctrt_syntax}
\end{minipage}
\end{wrapfigure}

The formal syntax of our specification (or contract) language, presented in
Fig.\ref{fig:ctrt_syntax}, allows definitiosn of
\propS{}, a first-order formula
that establishes dependency relations between effects,
necessary to determine the effects an operation may witness, under
a given consistency level.
The language is seeded with $\soZ$ and $\visZ$, respectively representing session
order and visibility relations over effects, 
and defines dependency \relationS{} as a sequence\footnote{\tool also allows
using closures of seeds, which is omitted here for
simplicity.} of seeds,  
where 
({\footnotesize $a \xrightarrow{\rel_1;...;\rel_k} b$})
is interpreted as 
{\footnotesize$\exists c. (a
\xrightarrow{\rel_1;...;\rel_{k-1}} c
\wedge c \xrightarrow {\rel_k} b)$}.
$\nullR{}$ is the empty relation.
Additionally, the language allows conjunctions of propositions, \specS{},
used to define a safe environment
free from \emph{multiple} inconsistencies. 
Our language is crafted to capture all fine-grained weak consistency
levels, including well-known ones such as those explicated by Terry et al. \cite{terry}
(see e.g., Fig.\ref{fig:ctrt_example}).

%
% UB and LB contracts
We provide two important classes of 
contracts, and explain how they can be
satisfied with different enforcement techniques.
\begin{description}
\item {\textsf LB}: A \emph{lower bound} (\LB{}) contract is one in
  which all defined dependency relations end with an \soZ, i.e. are of
  the following form: ({\footnotesize $\forall a. a
    \xrightarrow{r_1;r_2;...;\soZ} \hat{\eff} \Rightarrow a
    \xrightarrow{\visZ} \hat{\eff}$}). It specifies the smallest set
  of effects that any operation should witness to maintain
  consistency, e.g.  \rmwCTRT{} and \mrCTRT{} in
  Fig.\ref{fig:ctrt_example}.

\item {\textsf UB}: Similarly, we define \emph{upper bound} (\UB{})
  contracts as those whose dependency relations end with a $\visZ$.
  These contracts define constraints on the set of effects made
  visible to each operation; if an effect is in the set, certain
  dependencies of that effect must also be included, e.g.  \visCTRT{}
  and \mwCTRT{} in Fig.\ref{fig:ctrt_example}.
\end{description}
Our consistency enforcement approach is based on blocking operations
with \LB{} contracts to make sure that they witness \emph{all effects
  that they are supposed to}, and filtering for \UB{} contracts to
make sure that they do not witness \emph{effects that they are not
  supposed to}.  
