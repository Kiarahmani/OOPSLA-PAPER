\section{Algorithm}
\label{sec:alg}

%intro
In Sec.~\ref{sec:semantics}, we presented a high-level
abstraction of our system behavior, where we explained \emph{what} subset of
effects at a replica, must be witnessed by every operations. In this
section, we explain \tool's algorithm to efficiently maintain a \emph{consistent
cache}, in order to avoid redundancies in filteration mechanism.

%how cache works
\tool maintains a consistent cache on top of each replica, by periodic
reads from the underlying ECDS, where an effect $\eta$ is moved to the
cache, only  if the cache already includes {\footnotesize
$\trunc{R}_V^{-1}(\eta)$}. Consequently, all operations under \UB{}
contracts can be immediately executed
by witnessing the cache, which is a closed subset (not necessarily maximal all the time) 
of effects present at the replica. 
Moreover, \LB{} contracts can be satisfied, if operations are
blocked until effects of all previous operations from the same session enter
the cache, in which case current operation can proceed and witness \emph{all}
effects present at the replica.

%the memoization technique
Additionally, we implemented a simple memoization technique in \tool,
that extends the binary notion of dependency presence to the
\emph{degree of dependency presence} (\DDP{}), that represents 
the maximum \emph{depth} of dependencies, whose presence have been
verified so far. 
Consequently, when verifying the presence of
dependencies for an effect fails, the runtime system can avoid redundant
computations, next time it tries to verify the same property for the
same effect.
\tool's runtime, by performing
periodic \DDP{} refreshes, tries to assign larger numbers to each effect
while more effects arrive at the replica. 
Specifically, at each refresh the $DDP$ of an effect $\eff$ is increased from $i$ to $i+1$ if
all effects in $r_{i+1}^{-1}(\eff)$ already have $DDP$ values at least
equal to $i$.

% example
For example, consider a contract with dependency relation
$\Rel=\soZ;\visZ;\soZ$, and a newly arrived effect $\eta$ to the
replica, whose \DDP{} is initially set to 0. 
During the next refreshes, $\eta$ is given the value 1, if all
effects in $\soZ^{-1}(\eta)$ have \DDP{} equal to 0 (i.e. are present at the
replica). Similarly, $\eta$ is given the value 2, if all effects in
$\visZ^{-1}(\eta)$ have \DDP{} value of at least 1, which means that now
$(\soZ;\visZ)^{-1}_V(\eta)$ is now present at the replica and $\eta$ can
be safely added to the consistent cache (Fig.~\ref{fig:avail_deg})
\begin{figure}[t]
	\centering
	\includegraphics[scale = 0.4]{Figures/Availability_deg.pdf}
\caption{Representation of the stepwise process where effects become closer to
the cache, before actually entering it}
\end{figure}

