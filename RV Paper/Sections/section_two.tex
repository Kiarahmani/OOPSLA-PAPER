%==========================================================
%--- What programs are and how they are written in our tool
%==========================================================
\section{System Model}
\label{sec:sys_model}

A data store in our system model is a collection of \emph{replicas}
(\#1,\#2,...), each of which maintains an instance of a set of
replicated \emph{data object} ($x$,$y$,...).  These objects, which are
defined by application developers, contain a \emph{state}
($v$,$v'$,...) and are equipped with a set of \emph{operations}
($op$,$op'$,...), each of which are designated as either read-only or
effectful.  The former characterizes operations that just read from
the store, and receive an instance of an object's state, while the
latter characterizes operations that modify an object's state by
generating \emph{update effects} ($\eff$,$\eff'$,...).  Update effects
or simply effects are associated with an \emph{apply} function that
executes asynchronously, and modifies object state on different
replicas.  An effectful operation is handled by a replica that updates
the object's state, and guarantees eventual delivery of the effect to
all other replicas in the system. Each recipient uses the apply
function to modify its local instance of the object;
Fig.\ref{fig:system_model} illustrates this behavior.

Observe there is no direct synchronization between replicas when an
operation is executed, which means there can be conflicting updates on
an object, that are generated at different replicas concurrently.  We
do not bound the system to a particular conflict resolution strategy
Consequently, this model admits all inconsistencies and anomalies
associated with eventual consistency~\cite{quelea}; our goal is equip
applications and implementations with mechanisms to specify and avoid
such inconsistencies.

Clients interact with the store by invoking operations on objects.
A \emph{session} is a sequence of operations invoked by a particular
client. Consequently, operations (and update effects) can be uniquely
identified by their invoking \emph{session id} and
their \emph{sequence number} in that particular session. The data
store is typically accessed by a large number of clients concurrently
and as a result of the load balancing regulations of the store,
operations might be routed to different replicas, even if they are
from the same session (Figures \ref{fig:sys_model1} and
\ref{fig:sys_model3}).  Operations belonging to a given session are
not required to be handled by the same replica.

We new define two relations over effects created in the store.
\emph{Session order} ($\xrightarrow{\soZ}$) is an irreflexive, transitive relation that relates
effects from the same session following the integer \emph{smaller
than} relation over their sequence numbers.  \emph{Visibility} ($\eff
\xrightarrow{\visZ} \eff'$) is an
irreflexive and assymetric relation that holds if effect $\eff$ has
been applied to a replica $R$ before $\eff'$ is applied to $R$.  In
Fig.\ref{fig:sys_model3} for example, $\eff'$ (the effect of executing
op') will witness the updates associated with effect $\eff$
($\eff \xrightarrow{\visZ} \eff'$), since $\eff$ is already present at
the replica4, when op' is submitted).

\begin{figure}[t]
    \centering
    \begin{subfigure}[t]{0.3\textwidth}
    \centering
        \includegraphics[scale=0.32]{Figures/system_model1.pdf}
        \caption{\scriptsize A client submits an operation $\op$ to the store, which
	is routed to the replica \texttt{\#1}.}
        \label{fig:sys_model1}
    \end{subfigure}
    \hfill
    \vline
    \hfill
    \begin{subfigure}[t]{0.3\textwidth}
        \centering
	\includegraphics[scale=0.32]{Figures/system_model2.pdf}
        \caption{\scriptsize The state of the replica \texttt{\#1} is updated, an
	effect is created and is being propagated}
        \label{fig:sys_model2}
    \end{subfigure}
    \hfill
    \vline
    \hfill
    \begin{subfigure}[t]{0.3\textwidth}
        \centering
	\includegraphics[scale=0.32]{Figures/system_model3.pdf}
        \caption{\scriptsize Second operation $\op'$ is submitted to the store, which
	is routed to the replica \texttt{\#4}}
        \label{fig:sys_model3}
    \end{subfigure}
    \caption{system model of \tool}\label{fig:system_model}
\end{figure}




















%==========================================================




























































\begin{comment}

%
% Initial Paragraph explaining the running example
%

Let's consider the developement of a modern-day large scale \emph{Bulletin
Board} web application, that allows managers to initiate multiple boards for their
organization, where employees can either write something on the
board, or request to read all the messages on it.
In order to achieve high availablity, developers decide to implement the
application as distributed servers, working
with replicated data objects on top of an 
Eventually Consistent Data Store ({ECDS}). The data model here dictates
each row at the data base level, to include a board identifier
and a set of messages 
(Fig. \ref{fig:simple_bb}).
In this fashion, employees can start a session to send requests to geographically distributed
servers and make updates to the rows, and the underlying store,
would guarantee the eventual delivery of all updates at every server.
\begin{figure}[h]
    \begin{subfigure}[b]{0.48\textwidth}
        \centering
	\includegraphics[scale = 0.5]{Figures/SimpleBulletinApp.pdf}
        \caption{Simple Data Model}
    \end{subfigure} 
    \begin{subfigure}[b]{0.48\textwidth}
        \centering
	\includegraphics[scale=0.5]{Figures/ModifiedBulletinBoard.pdf}
	\caption{Modified Data Model}
    \end{subfigure}
    \\ \hrulefill \\
\caption{Low-Level Data Model of a Simple Bulletin Board Application
(Left) and the Modified Version Including Session Counters to be Used
for Providing }

\end{figure}

%
%===========================================================================SUBSECTION
% Paragraph 2: Explaining the possible anomaly under EC
%
\subsection{Ad-hoc Anomaly Prevention Mechanisms}

Developers cannot solely rely on the eventual consistency guarantee that
is provided by the underlying data stores for application correctness. There are application integrity anomalies
that must be thought of at the developement stage and be handled. For
example, assume an employee signs into the system, writes some
messages on the board, and immediately refreshes her browser hoping to see
her messages on the board, which are however not there. This is obviously
not desirable and as mentioned in
the previous section,  can occure if
the write messages were sent to a server, and the subsequent read to
another, where her previous write updates are  not available yet. 
\begin{wrapfigure}{l}{0.3\textwidth}
	\centering
	\includegraphics[scale = 0.4]{Figures/metadatapresent.pdf}
\vspace{1 mm} 
\hrule
\caption{Incorrect implementation of RMW. The meta data referring to the
first operation of session A can be present at server \#2, when the
actuall message is not}
\end{wrapfigure}

%
% Paragraph 3: Ad-hoc prevention methods
%

To prevent this anomaly, developers must provide RMW consistency
guarantee for their application, where user requests sent to a server are guaranteed to see
all the previous updates made by the same client and session. The
conventional technique for acheiving this,
is to tag each session and its contained requests, respectively with unique identifiers and
sequence numbers and to include this meta-data in the database, to
record what requests have been seen by the server at any time. 
Moreover, there must be some mechanisms to temporarily block user requests, when the database
does not include the updates from previous requests.
Since developers are not interested in synchronized and direct
communication between servers (which obviously contradics the idea of using
an ECDS), they have to record the meta data in the key value store and rely on
the provided delivery mechanisms. 
%
%Paragraph 4: The problem with simply adding the metadata on top of the
%implementation
%

However, the desired behavior cannot
be achieved by simply creating meta data "add-ons" to the current implementation.
The reason is that, in EC stores, no order of delivery is guaranteed,
and some servers might receive the meta-data regarding some updates that
are not present at the server yet. 
Developers need to make sure that the meta-data recording sequence
number of a request, becomes available at a server, at the same time
when the corresponding update arrives. This can only be done by
modifying the low-level initial data models, to also record the sequence
numbers seen from each client. 
%
% Paragraph 5: The difficulties of low-level modifications: complex
% and redundant
%

Now developers are inevitebaly required to modify the origianl low level
data model to also include the required meta data, so the changes in the
data and meta-data would arrive at servers atomically. Such changes on
the data model will require rewriting 90\% of application functions from the scratch
which is extremely inconvinience. 
Moreover, now developers must face non-trivial problems associated with
session management and dynamic data models. For example, when a new
client arrives at a server 
and initiates a session, the server is required to generate a unique ID
for that session, and then add a new column to the
underlying database table, to record the associated meta-data.
Unfortunately, table alteration in ECDSs is not an atomic task, and forces developers to implement
appropriate guards to make sure that the new column is available at all servers, before
allowing the client to move on with the rest of her requests.  
%
% Paragraph 6: The problem of new consistency requirements
%

To make the matter worse, new integrity anomalies are discovered very
frequently. Assume the bulltin board application with RMW has been
successfully designed and implemented, however, after the developement
phase users report a new type of anomaly, where messages seem to be
disappeared after refreshing the browser. This undesirable behavior is
allowed under EC and even RMW; since two read operations can go to
different servers, where the second one does not include all the
messages read from the first server. This requires another consistency
level, called Monotonic Reads, that guarantees subsequent reads will
include everything that has been read before in a session.
 
 
 






\end{comment}


